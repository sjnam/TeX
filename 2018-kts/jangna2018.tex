
% 2018 KTS
% Dancing links package

\documentclass[xcolor=svgnames]{beamer}

\usetheme[titleformat=smallcaps,subsectionpage=progressbar]{metropolis}
\metroset{outer/progressbar=head}

\usepackage{kotex}
\usepackage{pentominoes}
\usepackage{sudoku-dlx}
\usepackage{queens}


% title
\title{댄싱 링크 패키지}
\subtitle{루아텍으로 퍼즐 조판하기}
\date{2018년 2월 3일}
\author{남수진}
\institute{
  2018 한국텍학회 학술대회 및 정기총회 \\
  판교 스타트업캠퍼스 1동 2층, 세미나실 1}
\titlegraphic{\hfill\includegraphics[height=30mm]{imgs/meta.pdf}}


%%
\begin{document}

\maketitle

%
\begin{frame}{목차}
  \setbeamertemplate{section in toc}[sections numbered]
  \tableofcontents
\end{frame}


%%
\section{댄싱 링크}

\let\a\alert
%
\begin{frame}{집합 덮개 문제}
  다음 행렬에서, 모든 열이 오직 1을 갖도록 하는 행들의 집합을 구하여라.
  {\Large\boldmath
    $$
    \bordermatrix{
      \mathstrut&&&&&&&\cr
      &0&0&1&0&1&1&0\cr
      &1&0&0&1&0&0&1\cr
      &0&1&1&0&0&1&0\cr
      &1&0&0&1&0&0&0\cr
      &0&1&0&0&0&0&1\cr
      &0&0&0&1&1&0&1}
    $$}
\end{frame}

%
\begin{frame}{집합 덮개 문제}
  다음 행렬에서, 모든 열이 오직 1을 갖도록 하는 행들의 집합을 구하여라.
  \alert{(1, 4, 5 번째 행)}
  {\Large\boldmath
    $$
    \bordermatrix{
      \mathstrut&&&&&&&\cr  
      &\a0&\a0&\a1&\a0&\a1&\a1&\a0\cr
      &1&0&0&1&0&0&1\cr
      &0&1&1&0&0&1&0\cr
      &\a1&\a0&\a0&\a1&\a0&\a0&\a0\cr
      &\a0&\a1&\a0&\a0&\a0&\a0&\a1\cr
      &0&0&0&1&1&0&1}
    $$}
\end{frame}

%
\begin{frame}{집합 덮개 문제}
\Large\boldmath
  \begin{figure}[!htb]
    \hskip-17mm\begin{minipage}{.7\textwidth}
      \centering
      $\bordermatrix{
  \strut&&&&&&&\cr
  &0&0&1&0&1&1&0\cr
  &1&0&0&1&0&0&1\cr
  &0&1&1&0&0&1&0\cr
  &1&0&0&1&0&0&0\cr
  &0&1&0&0&0&0&1\cr
  &0&0&0&1&1&0&1}\ \Rightarrow$
    \end{minipage}%
    \begin{minipage}{.3\textwidth}
      \centering
  $
  \begin{array}{ccccccc}
    A & B & C & D & E & F & G\\
    \hline
    C & E & F &&&&\\
    A & D & G &&&&\\
    B & C & F &&&&\\
    A & D &&&&&\\
    B & G &&&&&\\
    D & E & G &&&&
  \end{array}
  $
    \end{minipage}
\end{figure}
\end{frame}

\renewcommand\arraystretch{1.1}
%
\begin{frame}{집합 덮개 문제}
  어떠한 전체 집합과 그 집합의 부분 집합들이 주어졌을 때, 부분 집합들 중에서 가능한 한
  적은 집합을 골라서 그 집합들의 합집합이 원래 전체 집합이 되도록, 즉 그 집합들이
  원래 전제집합을 \alert{덮도록} 집합을 선택하는 문제이다.
  {\large\boldmath
  $$
  \begin{array}{ccccccc}
    \a A & \a B & \a C & \a D & \a E & \a F & \a G\\
    \hline
    \a C & \a E & \a F &&&&\\
    A & D & G &&&&\\
    B & C & F &&&&\\
    \a A & \a D &&&&&\\
    \a B & \a G &&&&&\\
    D & E & G &&&&
  \end{array}
  $$}
\end{frame}

%
\begin{frame}{댄싱 링크}
  \begin{itemize}
  \item 집합 덮개 문제를 해결하는 알고리즘을 효율적으로 구현하는 하나의 기법
  \item \href{https://www.youtube.com/watch?v=pN76VICZiKU&start=100}
    {알고리즘이 진행되면서 노드와 링크로 구현되는 자료구조의 상태 변화가
      마치 춤을 연상하는 것 같다.}
  \item \href{https://github.com/sjnam/lua-dancing-links}
    {lua-dancing-links}
  \end{itemize}
    
  \centering\includegraphics[height=50mm]{imgs/cdance-4.png}
\end{frame}


%%
\section{펜토미노}

%
\begin{frame}{펜토미노}
  \begin{center}
  {\Large 12 조각} \\
  \includegraphics[height=5cm]{imgs/pentominoes.png}
  \end{center}
\end{frame}

%
\begin{frame}[fragile]{펜토미노 3x20}
\begin{verbatim}
\usepackage{pentominoes}
\pentominoes{5mm}{3}{20}{1}.
\end{verbatim}
\pentominoes{5mm}{3}{20}{1}.
\begin{verbatim}
\pentominoes{5mm}{4}{15}{1}.
\end{verbatim}
\pentominoes{5mm}{4}{15}{1}.
\end{frame}


%%
\section{수도쿠}

%
\begin{frame}[fragile]{수도쿠}
\begin{verbatim}
\sudoku {9.......6.3.4....9...915.8..8.5..7..%
..3.9.4....2..1.9.32176....6..1.2.3.8...5...1}
\end{verbatim}  
\begin{center}
  \sudoku{9.......6.3.4....9...915.8..8.5..7..%
    ..3.9.4....2..1.9.32176....6..1.2.3.8...5...1}
\end{center}
\end{frame}

%
\begin{frame}[fragile]{수도쿠}
\begin{verbatim}
\sudoku*{9.......6.3.4....9...915.8..8.5..7..%
..3.9.4....2..1.9.32176....6..1.2.3.8...5...1}
\end{verbatim}
\begin{center}
  \sudoku*{9.......6.3.4....9...915.8..8.5..7..%
    ..3.9.4....2..1.9.32176....6..1.2.3.8...5...1}
\end{center}
\end{frame}


%%
\section{여왕 배치 문제}

%
\begin{frame}[fragile]{여왕 배치 문제}
\begin{verbatim}
\usepackage{queens}
\queens{8}{2}.
\end{verbatim}
\vspace{-10mm}
\queens{8}{2}.
\end{frame}


%
\begin{frame}{참고}
  \begin{itemize}
  \item \href{https://goo.gl/44hiLo}{집합 덮개 문제}
  \item \href{http://www-cs-faculty.stanford.edu/~knuth/fasc5c.ps.gz}
    {\textsc{The Art of Computer Programming Pre-Fascicle 5c}}
  \item \href{https://github.com/sjnam/lua-dancing-links}
    {lua-dancing-links}
  \end{itemize}
\end{frame}

%
\begin{frame}[standout]
  감사합니다
\end{frame}

\end{document}

