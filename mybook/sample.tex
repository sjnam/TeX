% luatex sample.tex

\input mymac

\def\ptenv#1#2{\noindent{\bf #1 포인트}\smallskip
  \begingroup #2 \noindent\sampletext\bigskip\endgroup}
\def\sampletext{젊은 시절에 신진사대부로서 강직한 성품을 지니고 있었던 것으로 전해지는데
과거 급제 후 젊은 신진파로서 공민왕의 총애를 받기도 했으나 공민왕 시해 이후 귀양을 가게
된다. 중국에서는 원나라가 몰락하고 있었다. 이 시기에 {\bf 정도전}은 권문세족들을 비판하고
원나라와는 단교하며 {\tt 명나라와 친해지자는 주장을 펴고 있었는데 결국 권문세족의 눈 밖에
나 관리 자리를 잃고 낙향하게 된다.}
$가^{나^다}.$\par
이러한 전통을 계승한 {\sl 바빌론(Babyion)\/}의 승려들은 지구라트(Ziggurat)라는 신전의
대탑에서 별들의 운행을 관측했다. 이집트의 대 피라미드가 완성된지 1000년이 지난 시절, 그리고
법전으로 유명한 함무라비(Hammurabi)대왕이 서거한지 한 세대가 지난 시절인 B.C.1600년경,
그들은 {\sl 인류 역사상 처음으로 별의 목표를 작성하는 데 성공하였다.} 이것은 천문학의
역사에 있어서 획기적인 일이었다. 이 별의 목록표를 바탕으로 신전의 천문학자들은 별들의
운행을 좀더 정확하게 예측할 수 있게 되었고 그만큼 그들에 대한 군주의 신뢰는 증가되었다.}

%%

^^{정도전}
^^{바빌론}
^^{Ziggurat}

\ptenv{10}\noindent

\ptenv9\ninepoint

\ptenv8\eightpoint

\sampletext

\bye
