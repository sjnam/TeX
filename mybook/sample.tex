% luatex sample.tex

\input mymac
\input luamplib.sty

\def\section#1.{\penalty-500\smallskip
  \noindent{\bf #1}\nobreak\smallskip}
\def\ptenv#1.{\section{#1 포인트}.
  \ifnum#1=9\def\size{\ninepoint}
    \else\ifnum#1=8\def\size{\eightpoint}
    \else\def\size{\tenpoint}\fi\fi
  \bgroup\size\noindent\sampletext\bigskip\egroup}
\def\sampletext{젊은 시절에 신진사대부로서 강직한 성품을 지니고 있었던 것으로 전해지는데
  과거 급제 후 젊은 신진파로서 공민왕의 총애를 받기도 했으나 공민왕 시해 이후 귀양을 가게
  된다. 중국에서는 원나라가 몰락하고 있었다. 이 시기에 {\bf 정도전}은 권문세족들을 비판하고
  원나라와는 단교하며 {\tt 명나라와 친해지자는 주장을 펴고 있었는데 결국 권문세족의 눈 밖에
  나 관리 자리를 잃고 낙향하게 된다.}
  $가^{나^다}.$\par
  이러한 전통을 계승한 {\sl 바빌론(Babyion)\/}의 승려들은 지구라트(Ziggurat)라는 신전의
  대탑에서 별들의 운행을 관측했다. 이집트의 대 피라미드가 완성된지 1000년이 지난 시절, 그리고
  법전으로 유명한 함무라비(Hammurabi)대왕이 서거한지 한 세대가 지난 시절인 B.C.1600년경,
  그들은 {\sl 인류 역사상 처음으로 별의 목표를 작성하는 데 성공하였다.} 이것은 천문학의
  역사에 있어서 획기적인 일이었다. 이 별의 목록표를 바탕으로 신전의 천문학자들은 별들의
  운행을 좀더 정확하게 예측할 수 있게 되었고 그만큼 그들에 대한 군주의 신뢰는 증가되었다.}

\centerline{\twelvemj 플레인텍 한글 환경 테스트}
\bigskip

\ptenv 10.
\ptenv 9.
\ptenv 8.

\section 이미지 삽입 테스트.
$$
\includegraphics width 1.5in {images/1.png}\qquad
\includegraphics width 1.5in {images/2.png}\qquad
\includegraphics width 1.5in {images/5.png}
$$

\section \.{luamplib} 테스트.
\begindisplay
\vbox to7cm{\mplibcode
beginfig(1)
phisq = (0.5*(1+sqrt(5)))**2;
for n = 1 upto 800:
  r := 3*sqrt(n);
  theta := 360*((n/phisq) mod 1);
  filldraw fullcircle scaled 3bp shifted (r*cosd(theta),r*sind(theta));
endfor;
endfig
\endmplibcode\vfill}\quad\vbox to7cm{\vskip-10pt
  \begintt
  \mplibcode
  beginfig(1)
  phisq = (0.5*(1+sqrt(5)))**2;
  for n = 1 upto 800:
    r := 3*sqrt(n);
    theta := 360*((n/phisq) mod 1);
    filldraw fullcircle scaled 3bp shifted
      (r*cosd(theta),r*sind(theta));
  endfor;
  endfig
  \endmplibcode
  \endtt\vfill}
\enddisplay

\bye

