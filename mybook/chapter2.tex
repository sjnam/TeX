\input mymac

\titletrue

\section 머리말

연구 업적이 많고 학문이 높은 학자의 강의라고 해서 그것이 반드시 훌륭한 것은 아니다.
한번은 "중력" 이라는 규명한 책을 쓴 저자들 중에서 가장 유명한 사람의 강의에 들어간 적이
있는데, 그 강의는 마침 들고 들어간 커피 한잔만이 유일한 위로였을 만큼 지루하고 답답했다.
그의 작은 목소리는 참을 수 있을지라도 내내 평범한 사실을 맴도는 말솜씨에 그저 이마에 물을
한방을 떨어뜨리는 중국식 물 고문을 당하는 것 같았다. 이와 반대의 경우도 있다. 흔히 강의를
아주 잘하기로 이름난 사람들의 연구 업적을 살펴보면 그것이 의외로 평범하다는 데 놀라곤
한다.

그렇다면 그 사람의 강의력과 연구 업적과는 아무런 관련이 없는 말일까. 물론 그렇지 않다.
훌륭한 업적을 이룬 사람일수록 지식이 해박하며 강의도 자신감에 차 있는 법이다. 실상 앞의
경우는 강의에 쏟는 열정에 못 미치는 데에서 기인하는 것이다. 학자들에게 강의 활동이란 연구
활동의 연장이나 다름없는 것이다. 사람마다 차이는 있겠지만 일반적으로 전혀 성격이 다른
두개의 연구를 동시에 못하는 사람들이 많다. 그래서 훌륭한 학자가 훌륭한 강의자로 탈바꿈하지
못하는 경우가 오히려 흔한 것이다.

이런 이유로 학생들을 잘 지도하기 위해 다른 연구 활동을 아예 포기하다 시피 하는 사람들도
많다. 일찍이 이런 미묘한 함수 관계를 잘 파악한 구미의 대학에서는 아예 학자들에게 강의나
연구 둘 중 하나에 관심을 쏟도록 유도하고 있다. 두 마리의 토끼를 쫓다가 둘 다 놓치는
어리석음을 범하기보다는 한 명이라도 더 훌륭한 연구가 또는 강의자를 확보하려는 노력이라고
볼 수 있다.

이런 상황에서 하물며 학자가 일반 대중을 위한 훌륭한 책을 쓰기란 하늘의 별을
따기보다 더 힘들다. 재능의 한계도 그러려니와, 대중이란 학생이라기보단 극장의 관객에 더
가까운 존재이기 때문이다. 관객은 흥미가 없으면 극장에 들어가지 않고, 재미가 없으면 극장을
떠난다. 이런 두려움 때문인지 우리 학계에서는 현대 과학이라는 거대한 명작을 관심 있는
사람들에게 공연해 보이려는 노력이 미흡했던 것 같다. 물론 매우 짧은 우리의 근대 학문사나,
절대수가 모자라는 엷은 학자층을 생각하면 이를 무조건 탓할 수만은 없다. 그러나 요즈음
해외에서 공부한 젊은 학자들을 중심으로 활발한 연구 분위기가 조성되고 있다.

이러한 신선한 학계의 공기를 일반 대중과 같이 호흡해 보려는 의도에서 현대물리학자의
최대 발견이었던 상대성이론과 이를 바탕으로 한 현대 우주론을 되도록이면 쉽게 써 보았다.

이 책을 읽는 사람 중 아인슈타인의 이름을 처음 듣는 사람은 거의 없을 것이다. 그것은
물론 그가
독자적으로 제창한 두개의 상대성 이론, 즉 특수상대성과 일반상대성 이론 때문이다. 이 두
이론은 인류가 수레를 발명한 이래 최대의 발견으로 평가된다. 이론의 논리적 수학적 아름다움은
차지하더라도 그 심오한 의미는 우리 인류의 우주적 존재 문제에 대한 궁극적 해답을 내포하고
있다. 이런 의미에서 상대성이론을 제창한 아인슈타인은 맹자, 다윈, 피카소, 쇤베르크, 그리고
제임스 조이스를 능가하는 세기적 선각자의 위치를 차지한다.

상대성이론은 근본적으로 시간과
공간에 대한 이론이다. 그러므로 이 이론은 우리가 살고 있는 주변 공간과 우리의 손목시계로
재는 시간의 우주적 연관 관계를 파헤치는 이론이다. 특히 이 이론은 라이프니츠와 데타르트
이후 많은 사람들이 질문을 던져 왔던 공간의 존재성에 관한 명확한 해답을 제공한다.
사람들은 지난 2000년간 과연 물질이 전혀 존재하지 않는 공간이란 과연 자연에 실제로
존재하는 것이냐고 질문해 왔다. 물론 어느 누구도 이러한 질문에 대해 명확한 대답을
제시하지 못했다. 상대성이론에 따르면 물질이 차지하는 영역만이 바로 공간이라는 것을
보여준다. 아무것도 없는 '텅빈 공간' 이란 우리가 사는 우주 어디에도 존재하지 않는 다는
결론인 것이다.

이러한 상대론이 갖는 자연 철학적 함축성 때문인지 상대성 이론의 발전에는 비단 자연 과학자
뿐만 아니라 철학자, 사상가, 신학자들도 큰 역할을 담당해 왔다. 예를 들면 상대론적 우주론의
탄생엔 벨기에의 신부가 중요한 역할을 했고, 상대론적 우주론의 핵심에는 '부분은 전체를
포괄한다.'는 불경의 구절이 이론화되어 있다.

상대성이론이 가장 빛을 발하는 영역은 공간과 그 안에 있는 물질의 시간적 진화 과정을
다루는 학문, 즉 우주론이다. 상대성 이론의 등장으로 우주론은 20세기에 들어와 가장 급격히
발전한 학문이라고 해도 과언은 아닐 것이다. 실제로 현재 세계 유수의 대학에서 상대론적
우주론을 강의하고 있지 않은 학교란 거의 없는 실정이다.

옛날부터 우주론의 탐구 영역이
넓어짐에 따라 과학 종교, 또는 과학과 종교철학과의 분란이 있어 왔다. 우주론의 탐구 영역이
넓어지면 그만큼 종교의 영역을 축소시킨다는 위기감이 존재한다는 것이다. 바로 이러한 이유
때문에 코페르니쿠스의 친구였던 파문을 당했다.

1981년은 아주 의미 있는 해였다. 왜냐하면
혁명적인 인플레이션 이론이 나옴으로써 현행 우주론이 급격한 진보를 했기 때문이다. 이 해
로마의 바티칸 천주교회는 이러한 상대론적 우주론의 급격한 발전에 따라 만에 하나라도 발생할
수 있는 종교와의 상등을 해소하려는 목적으로 세계의 석학들을 초청, 현대  우주론에 대한
바티칸 학회를 열었다.
이 학회는 연사로 나온 교황을 요한 바오로 2세는 종교와 현대 과학의 관계를 다음과 같이
설명했다.

\beginemph
... 자연과학의 탐구 대상이 되는 어떠한 문제라도, 그것이 원자든 또는 우리가 사는 우주
전체든, 항상 자연과학자들은 그들이 속해 있는 우주의 근원에 관한 궁극적 미해결 점을 안고
있다. 현대 우주론에 관한 어떤 이론, 또는 어떤 석학들이라도 이 점에 대해서는 절대로 명확한
해답을 줄 수 없으리라 믿는다. 우리 종교가의 입장에서 보면 이러한 질문이 요구하는 것은
물리학이나 천문학적 지식이 아니라 이를 초월하는 어떠한 형이상학적인 초월한 궁극적 신학의
영역을 느끼는 것이다...
\endemph

여기에서 우리는 상대론적 우주론과 종교간에는 아주 자연스런 공존 관계가 성립될 수 있음을
본다.

그렇다면 일반 독자들은 다음과 같은 질문을 던질 수 있을 것이다. 결국 이러한 과제를 다루는
상대성이론이란 과학자, 철학자, 또는 그 외 사상가들의 전유물일 수밖에 없는 것일까. 수락이나
물리학의 배경이 없는 일반 독자들은 이 이론을 전혀 이해할 수 없는 것일까. 물론 그것은
절대로 아니다. 이 책을 읽는 독자들도 차츰 느끼리라 생각되지만, 상대성이론은 복잡한 수학을
사용한 계산의 결과라기보다 오히려 깊은 사색과 명상의 결과로 얻어진 것이다. 그러므로
논리의 흐름을 면밀히 좇는 독자라면 어떠한 깊은 수학적 물리학적 배경 없이도 상대성이론의
정수를 고스란히 이해할 수 있다. 물론 수학이 자연현상을 탐구하는 데에 있어서는 안될 강력한
도구인 것만을 틀림이 없다. 그러나 수학은 항상 대다수 사람들로 하여금 자연법칙의 진정한
면모를 맛볼 수 없게 하는 가장 큰 방해꾼의 입장에 서 왔다. 과학과 일반 대중의 괴리를 깊게
하는 주범이었던 것이다. 이러한 관계를, 26세 수락의 천재 맥스웰에게 보낸 편지를 동해 소개를
해보자.

\beginemph
...수학을 써서 결론에 도달했을 때 비단 수학적 표현을 쓰지 않더라도 명확한 의미는 전달할
수 있다고 믿고 있네. 수학의 문외한인 나 같은 사람도 이를 이해하여 다른 실험을 계속할 수
있을 테니까... 그러니 새 결과가 나오면 반드시 그것들을 이집트 상형문자 같은 난해한 수학적
표현으로부터 문외한이라도 이해하기 쉬운 명확한 설명이나 말로 바꿔 주길 바라네... 훌륭한
학자라면 그러한 능력을  지니고 있다고 믿네.
\endemph

패러데이는 정규 교육을 전혀 받지 못해 거의 이해하지 못했던 사람이다. 그래도 그는
무전 독학으로 전자기유도 현상을 발견해 근대 전자기학의 대부가 된 사람이다.

이러한 상황을 감안하여 이 책에서는 되도록이면 불필요한 수락 배경, 즉 어떠한 모순 점들이
아인슈타인을 비롯한 많은 사람들을 괴롭혔던가, 그리고 그러한 문제점들이 어떻게 상대성
이론을 통해 해결되었는가를 설명하였다. 이 과정에서 독자들이 상대성이론이 나오게 된 전체적
상황을 파악할 수 있게만 된다면 이미 상대성 이론을 반 이상 이해한 것이라 할 수 있다. 물론
수학이 전혀 없으면 특수 상대성 이론의 심오한 맛이 제거되는  안타까움 때문에 로렌츠변완식에
관한 수학적 계산  일부 채용한 것을 양해해 주기 바란다.

끝으로 아인슈타인의 다음과 같은 말을 소개하면서, 독자들도 과학과 철학, 형이상학과 신학을
넘나드는 기나긴 사고 여행을 떠나길 바라 마지 않는다.

\beginemph
끝이 없는 어둠을 속을 걷는 듯한 끊임없는 사고의 혼란, 여기에서 오는 견딜 수 없는 심적
고통, 그러다가 갑자기 모든 것이 이해되는 순간적 기쁨. 오직 이런 경험을 맛본 사람만이 이를
이해할 수 있다.
\endemph
\vfill

\rightline{\sl 1993년 6월, 대덕연구단지 화암 동산에서}
\smallskip
\rightline{\sl 나대일}
\eject

\null\vfill\eject

\beginchapter 특수상대성이론

\section 장자의 상대론적 나비꿈

``장자'' 제물론에는 나비가 되어 날아다니는 꿈을 꾼 장자의 이야기가 있다. 장자가 나비를
꿈꾼 것인가, 아니면 나비가 장자를 꿈꾸고 있는 것인가 모르겠다는 이야기. 이러한 장자의
나비꿈이 어떤 식으로 상대성이론과 연관이 있을까 의아해 하는 독자들이 많을 것이다. 실상
장자의 나비꿈은 상대성이론을 이해하는 데 가장 중요한 관성계, 그리고 상대성원칙(The
Princople of Relativity)이라는 상대성이론의 기본 정수를 모두 포괄하고 있다. 그렇다면
관성계란 무엇일까.

\beginbf
관성계란 일정한 상대속도로 움직이고 있는 계(좌표계)를 말한다. 가장 대표적인 관성계는
일정한 속도로 달리고 있는 승용차 안에 있는 사람이 느끼는 계다. 즉 언제라도 승용차 운전사가
속도계를 보았을 때 속도계 검침이 한 위치에 고정되어 있는 상태이다. 물론 이때 검침이
가리키는 값, 즉 시속 0km이거나 60km이거나 하는 값과는 전혀 상관이 없다.
\endbf

어떤 사람이 일정한 속도로 달리는 기차의 침대칸에서 한밤중에 깨어났다. 마침 기차의 창문은
모두 닫혀 있어서 그는 창밖을 전혀 볼 수 없었다. 그저 칠흑 같은 어둠 속에서 깨어난 것이다.
여기서 우리는 그 기차가 아주 완벽하게 설계된 직선선로상을 달리고 있기 때문에 전혀
덜컹거리거나 흔들리지 않는다고 가정한다. 그렇다면 이런 캄캄한 어둠 속에서 그는 기차가
움직이고 있는지, 아닌지를 알아낼 방법이 있을까.

이 경우 잠에서 깨어난 승객이 약간 멍청한 물리학자라면 그는 어떤 방법을 고안해 내어 그가
타고 있는 기차가 움직이고 있는지, 아닌지를 실험해 보려 할 것이다. 
그러나 그러한 실험은 항상
실패할 것이다. 그는 영원히 그가 탄 기차가 움직이고 있는지, 또는 아닌지를 알 수 없는
것이다.

이와 같이 일정한 상대속도로 일정한 방향으로 움직이는 계는 매우 독특한 계이다. 우리는
이러한 계를 통틀어 관성계(inertial frame)라 이름지어 부른다. 관성계에 있는 사람은 어떤
실험을 통해서도 자신이 속한 계가 움직이는지, 아닌지를 가려내지 못한다. 예를 들면 지표에
대해 정지한 기차 안의 관측자가 잰 수소원자의 질량은 달리는 기차 안에서 잰 수소원자의
질량과 똑같다. 이런 관성계의 독특한 성질을 우리는 상대성원칙이라고 부른다.

\beginbf
  상대성원칙은 바로 모든 관성계에서 성립되는 물리법칙은 항상 같다는 것이다.
\endbf

이 원칙에 따라 지표면에 대해 시속 50km로 움직이는 기차 안에서 물이 $100^\circ C$에서 
끓는다면
지구에 대해 시속 3만km로 움직이는 아폴로  우주선 안에서도, 그리고 태양에 대해 시속
18만km로 움직이는 지구상에서 모두 같은 온도인 $100^\circ C$에서 물이 끓어야 한다.
만약 이러한 상대성원칙이 틀린다면 우리가 생각할 수 있는 모든 관성계들은 전부 물리실험으로
구분 가능한 계가 된다는 결론이 나온다.

구체적인 예를 들어보자. 거의 광속에 가까운 일정한 속도로 달리는 은하철도 999호를
생각해 보자. 여기에서 인용된 은하철도는 재미있게 보았던 어느 만화영화에서 나오는 기차다. 이
기차를 인용하는 이유는 이 기차가 우주 공간을 매우 빠른 속도로 여행할 수 있게 묘사되어 있기
때문이다. 거의 광속으로 달리는 이 기차의 정거장은 군데군데 떨어진 행성들이다.

일단 이 은하철도 999호가 일정한 속도로 달리고 있으므로 은하철도 999호는 관성계에 속한다.
그런데 마침 999호를 스쳐 지나가는 은하철도 998호가 있다고 하자. 이 경우 은하철도 998호는
999호보다 좀더 빠른 속도로 움직이고 있다고 가정한다. 물론 998호 역시 일정한 속도로 달리고
있었기 때문에 또 하나의 관성계가 된다. 999호와 998호는 서로 움직이는 속도가 달라도 단지
일정한 속도로 움직인다는 사실에서 둘 다 관성계에 속한다. 이때 은하철도 999호의 주방에서
100d에 주전자의 물이 끓고 있다고 하자. 상대성원칙이 맞지 않다면 어떤 현상이 벌어질까. 가장
쉬운 가정은 만약 999호에서 물이 100d에 끓는다면 998호에서는 100d가 아닌 다른 
온도에서 물이
끓을 것이다. 그러므로 998호에서는 잠을 자다 관측자는 물이 끓는점을 측정함으로써 998호의
속도를 추정할 수 있게 된다.

  그렇다면 왜 우리가 사는 우주에서는 상대성원칙이 성립되고 있을까. 대답은 의외로 간단하다.
우주 공간 어디에 미아가 되어 버린 우주선이 있다고 가상해 보자. 그 안에는 마침 동면에서
깨어난 승무원이 있다. 그가 깨어난 순간 그는 우연히 차창을 스쳐 지나가는 혜성을 목격한다.
이 경우 그가 관측할 수 있는 물리량이란 우주선과 혜성간의 '상대속도' 뿐이다. 그런데 승무원의
입장에서 보면 자신은 항상 정지해 있는 것처럼 보인다. 따라서 어떤 승무원도 자신이 속한 계는
정지해 있는 계라고 생각할 것이다. 마치 움직이는 기차에 탄 승객이 자신은 정지해 있고 창밖의
가로수가 뒤로 물러가고 있다고 생각하는 것과 마찬가지다. 따라서 어떤 관성계이든 그것은 항상
정지계가 된다. 정지계에서 성립되는 물리법칙은 항상 같으므로 모든 관성계에서 성립되는
물리법칙은 항상 같다.

이야기를 바꿔서 관성계와 상대성원칙을 나비꿈 이야기와 비교해 보자. 장자는 '장자 자신이
장자의 꿈을 꾸는 나비'인지, 아니면 '나비의 꿈을 꾸는 장자'인지 전혀 알 수 없었다. 이것은
달리는 열차에서 막 깨어난 승객이 자신이 움직이는 기차에 있는지, 아니면 정지해 있는 기차에
있는지 알 수 없는 경우에 해당한다. 누구라도 두 개의 다른 세계를 구분해 주는 어떤 실험
방법을 생각할 수 없는 경우 '장자의 나비꿈 세계'와 '나비의 장자꿈 세계' 사이에도
상대성원칙이 충실히 성립된다고 우길 수 있는 것이다.

이 장을 통해서 우리는 상대성원칙이 우리가 사는 우주의 가장 신성한 원칙 중의 하나임을
알게 되었다. 바로 이 원칙이 성립됨으로써 우리는 아무런 불편 없이 열차 안에서 책을 읽고
음료수를 마시는 것이다.

우리는 태양 주위를 1년 주기로 공전하는 지구상에 살고 있다는 것을 생각해 보자. 지구의
태양에 대한 상대속도는 1년을 주기로 변하고 있다. 만약 상대성원칙이 틀린다면 지구상의 온갖
실험 결과 역시 1년의 주기성을 가지고 변할 것이다. 매년 5월 5일이 되면 물이 어떤 온도에서
끓고, 또한 그 다음날에는 이와는 다른 온도에서 물이 끓게 된다고 상상해 보라. 이렇게 되면
장안의 모든 커피집에서는 1년 365일, 매일 물의 끓는점을 명시한 달력이 걸려 있을 것이다.
물론 이러한 일은 절대 벌어지지 않는다. 이렇게 장안의 주방장들이 매일 확인해 주는
상대성원칙은 '우리 우주의 가장 신성한 법칙'인 것이다(상대성원칙이 틀릴 경우 나타날 수 있는
지구의 태양에 대한 상대속도에 따라 변하는 물의 끓는점 차이는 약간 과장되어 있다. 좀더
정확한 계산에 의하면 물의 연중 비등점 변화는 주어진 지구의 공전속도와 빛의 전파속도에 대한
상대비율로 정해진다. 지구의 공전속도를 초속 30km라고 하고, 빛의 진공 전파속도를 초속
30만km라고 하면 물의 연간 비등점 변화는
대략 $\hbox{(섭씨 100도)}\times (30/300{,}000)^2 \sim 
\hbox{(섭씨 100만분의 1도)}$에
불과하다).

\section 상대성원칙 위반하는 골칫덩이 빛

앞장을 통해 독자들은 관성계와 상대성원칙이 무엇을 뜻하는 말인지 알게 되었을 것이다.
그리고 그 중에서 상대성원칙은 우리 우주에 존재하는 모든 법칙 중 가장 신성한 법칙의
하나라는 것도 실감하게 되었을 것이다. 상대성원칙은 갈릴레이 이후 현재에 이르기까지
발전해 온 고전역학의 핵심으로 가장 신성한 원칙 중의 하나이다. 그럼 여기서, 아인슈타인의
상대성이론이란 고전역학과 전자기학의 충돌로 인한 문제를 해결하려다 탄생되었다고 강조해 온
것을 상기하자. '고전역학과 전자기학의 충돌'이란 바로 이렇게 신성한 상대성원칙을 전자기학의
총아로 등장한 빛이 노골적으로 위반하고 있다는 사실을 말한다. 이 문제는 금세기 초 모든
과학자들을 괴롭혔던 장본인으로, 이 때문에 '모든 길은 로마로 통한다'는 말처럼 금세기초
'모든 문제는 빛으로 통한다는 농담'이 학계에 떠돌았을 정도였다.

그러면 과연 빛이 어떻게 상대성원칙을 위반하고 있는지 좀더 자세히 보여주는 다음과 같은
아인슈타인의 사고실험을 소개한다. 

밀폐된 방에 어떤 아가씨가 손거울을 들고 앉아 있다고 하자.
방은 엄청나게 크고 그 한 구석에는 엄청나게 밝은 촉광의 전구가 빛을 발하고 있다. 따라서
그녀는 방 어디에서도 그녀의 모습을 손거울을 통해 비추어 볼 수 있다. 이 경우 상대성원칙에
따른 그녀는 그녀가 있는 방 전체가 정지해 있는지, 아니면 일정한 상대속도로 움직이고
있는지를 알아낼 방법이 없다.

그런데 어느 한순간 그녀가 손거울을 들고 전등빛에서 멀어져 가는 방향으로 달리기
시작했다고 하자. 그녀의 달리는 속도가 광속에 이르면 전등에서 나오는 빛은 그녀의
손거울에 다다르지 못하게 된다. 그 순간 손거울에 비친 그녀의 모습은 사라진다. 이때
그녀는 '아하! 나는 빛의 속도로 달리고 있구나!'하고 그녀가 달리는 속도를 알아차릴 수
있다. 그런데 그녀가 그녀의 달리는 속도를 알아낼 수 있다는 사실 자체가 이미 상대성원칙을
위반한 것이다.

이처럼 아인슈타인은 평소에 명쾌한 사고실험을 좋아했다. 혼자 사색에 잠길 때마다 그는
이러한 실험방법을 궁리해 냄으로써 수십 명의 우수한 학자들이 오랜 시간을 허비해야 하는
수학 계산이나 논리 전개를 대신했다.

우연의 일치였을까. 앞서 아인슈타인이 16세 때 던졌던 질문 "사람이 빛의 속도로 달리면 빛은
어떻게 보일까?"를 상기해 보자. 이 역시 관측자가 광속에 가까운 속도로 움직일 때 '빛의
파동성이 사라진다'는 심각한 문제가 발생했던 것이다. 다시 말하면, 지금까지 소개된
아인슈타인의 두 개 사고실험 모두 관측자의 속도가 빛의 속도에 다다르기만 하면 맥스웰
방정식이 예언하는 빛의 파동성이 깨어지고 고전역학에 있어서의 신성한 상대성원칙이 깨어지는
기현상이 벌어지는 것이다. 왜 하필 물체의 운동속도가 빛의 속도에 가까워지면 이런 이상한
현상들이 빚어지는 것일까. 그렇다면 상대성원칙은 빛이 수반되는 전자기 현상에서만은 성립되지
않는다는 예외 규정이 존재하는 것일까. 아니면 관측자가 빛의 속도로 달릴 수 있다는 사실이
틀린 것일까. 이런 문제들이 바로 금세기 초 많은 석학들을 괴롭혔다. 이들을 해결하려는
노력에서 20세기 최대의 이론으로 알려진 특수상대성이론이 발견되었던 것이다.

빛은 신성시되었던 상대성원칙을 범하는 데 만족하지 않고 이번엔 다른 방법으로 고전역학을
괴롭힌다. 빛의 재범 현장을 보여주기 위해 독자들을 침대칸 있는 열차로 초대한다.

침대칸에 있던 승객이 기차가 달리는 방향으로 동전을 던졌다고 가정하자. 편의상 동전이 그의
손을 떠나는 순간, 동전의 그의 손에 대한 속도를 A라 정한다. 이 순간 그가 탄 기차가 지면에
대해 움직이는 속도를 V. 그러면 지상에 있는 관측자가 측정한 동전의 속도는 기차의 지면에
대한 속도 V와 동전의 기차에 대한 속도 A의 단순한 합산이 된다. 이를 W라 표기하면
$$  W = V + A $$
이 식의 의미는 명백하다. 동전의 지면에 대한 속도는 동전을 기차가 달리는 반대 방향으로
던질 경우보다 훨씬 크다. 시속 10km로 달리는 기차에서 승객이 앞방향으로 시속 10km의 속도로
동전을 던진다면 동전의 지표에 대한 속도는 시속 20km$(10\hbox{km}+10\hbox{km})$가 되는 것을 말한다.

마찬가지로, 승객이 동전을 기차가 달리는 반대 방향으로 던졌다면 동전의 지면에 대한 속도는
기차의 지면에 대한 속도에서 동전의 기차에 대한 속도를 뺀 값이 된다.
$$ W = V - A$$
이 정도 수준의 더하기 빼기는 어느 누구도 이해할 수 있을 것이다.

다음에는 이러한 속도합산법칙을 빛이 경우에 적용해 보자. 즉 앞의 동전을 회중전등의
불빛으로 바꿔 보자는 것이다. 승객이 지면에 대해 V라는 속도로 달리는 기차에서 회중전등을
앞방향으로 비추었을 때 지면에 정지해 있는 사람이 본 불빛의 속도는 단순히 빛의 속도에
기차의 속도를 더한 값이 될 것이다.
$$  W = C + V $$
마찬가지로 회중전등을 기차 뒷쪽으로 비춘 경우에는 지면에 대한 불빛의 속도는
$W = C - A$일 것이다.

과연 빛이 이러한 고전역학적 속도합산법칙을 그대로 따르고 있는가를 알아보려던 실험이 앞서
소개한 유명한 마이켈슨-몰리의 실험이다. 그들은 지구의 공전운동이 우주 공간에 선택된 임의의
전지점을 기준으로, 지구의 공전속도가 6개월을 간격으로 속도의 방향이 바뀌는 것에 착안, 이
범에 대한 빛의 연간 속도 변화를 측정하려 했다. 이를 기차의 경우에 대비해 말하면 지구는
기차가 되고  우주 공간에서 임의로 선택된 점은 지표면이다. 마이켈슨과 몰리는 끈질기게
실험의 정밀도를 올려 가며 지구의 공전운동에 따른 빛의 속도 변화, $C+V$, $C-V$
값을 측정하려 애썼다. 그러나   실망스럽게도 실험결과는 항상 $W=C$라는 일정한
값으로 나타난 것이다. 빛의 전파 속도는 빛이 발해진 계의 속도나 방향에 상관없이 항상
일정하다는 결과였다.

여기에서 우리는 고전역학의 단순한 속도가감법칙이 빛에 관해서만은 성립되지 않는다는
사실을 알 수 있다.

\section 알고 보니 빛은 무죄

그러면 독자들은 과연 어떠한 문제들이 금세기 초에 존재했던가를 알게 되었을 것이다.
그렇다면 우리는 상대성원칙을 포기해야 하는가, 아니면 우리가 빛의 전파에 관한 속성을 무언가
잘못 이해하고 있다고 생각해야 하는가.

  이런 상황에서 아인슈타인의 특수상대성이론이 등장한다. 특수상대론적 처방을 보자.
특수상대성이론은 다음과 같은 두 가지 가설(postulate)을 기본으로 한다.
\item{(1)} 상대성원칙은 항상 맞다. 그러므로 관성계에서의 모든 물리법칙은 항상 같다.
\item{(2)} 광속은 항상 C이고 그 값은 어떤 관성계에서도 항상 같다.

  여기서 독자들은 펄쩍 뛸 것이다. 아니, 이 두개 가설은 앞서 증명된 것처럼 상호 모순관계가
아니었던가. 그러나 잠시 진정하기 바란다. 아인슈타인의 주장에 따르면 상대성원칙과 빛의
전파속성 모두 틀림없는 자연의 진리이다.

\beginbf
  고전역학과 전자기학이 서로 모순관계에 있는 것처럼 보이는 이유는 우리가 평소에 시간과
거리라는 개념을 잘못 이해하고 있기 때문이다.
\endbf

아인슈타인은 만약 우리가 시간과 거리의 개념을 제대로 이해하기만 한다면 관측자는 절대로
광속 C로 달릴 수 없다는 결론을 얻게 될 것이라고 말했다. 아주 흥미 있는 주장이다. 앞서
소개되었던 두개의 사고실험 모두 관측자가 광속으로 달리는 바람에 문제가 파생되었던 것을
기억하자. 만약 관측자의 속도가 광속에 이르지 못한다면 앞서 사고실험에서 발생된 모든 문제는
자연스레 사라지는 것이다.

문제의 해결점이 정말로 엉뚱한 곳에서 나왔다고 독자들은 갸우뚱할 것이다. 바로 이 점에서
아인슈타인의 천재성이 더욱 빛을 발한다. 만약 이런 엉뚱한 처방이 맞다면 이 처방의 결과로
탄생된 특수상대성이론은 고전적인 상대성원칙을 부정하기는커녕 오히려 발전시킨 모습이 된다.
바로 이런 이유로 아인슈타인은 그의 새 이론을 '특수상대성이론(The Special Theory of
Relativity)'이라고 명명하였다.

\section 진범은 시간과 공간

이 장에서 우리가 거리(또는 공간)라는 개념들을 어떻게 잘못 이해하고 있는지를 살펴보기로
한다. 이를 위해 먼저 우리가 학창시절 지겹게(?) 배웠던 물리 지식을 되새겨 보자. 우리가
역학적으로 주어진 물체의 운동을 기술한다 함은 그 물체의 시간적 위치를 정확히 안다는
것이다. 위치란 바로 주어진 기준점에서 잰 거리를 말한다. 그런데 여기서 말하는 물체의 위치란
대체 무엇일까. 독자들은 정말로 쓸데없는 질문을 한다고 귀찮게 여길지 모른다. 아니, 그렇다면
우리는 여태까지 위치가 무엇인지 모르고 살아왔다는 소리인가. 물체의 위치라는 개념을 정확히
이해하려면 일단 뉴턴역학이 잘 정리해 놓은, 뉴턴 자신이 평생 목놓아 '존재한다'고 외쳤던
절대공간이라는 개념을 자세히 알아보아야 한다.

뉴턴이 발한 절대공간은 뉴턴의 역학법칙이 성립되는 관성계의 기준점을 말한다. 즉
뉴턴역학의 경우 물체의 '위치'는 절대적으로 정지해 있다고 믿어지는 물체 주변의 공간을
기준으로 잰 거리를 말한다. 뉴턴의 관성의 법칙을 되새겨 보자.

주어진 물체는 다른 물체에서 충분히 멀리 떨어져 있을 때 원래 정지해 있었다면 영원히
그대로 정지한 상태로 남아 있으려 한다. 그리고 만약 원래 일정한 속도로 움직이고 있었다면
영원히 같은 속도, 그리고 같은 방향으로 움직이려는 속성이 있다.

여기에서 물체의 영원히 같은 방향으로 움직이려는 속성은 바로 갈릴레이가 발견한 관성을
말하는 것이다. 바로 이 관성이 존재하기 때문에 자동차가 커브길에 들어설 경우 운전자는 몸이
어느 한쪽으로 쏠리는 느낌을 받는다. 원래 직선 방향으로 움직이려던 관성이 커브를 트는 순간
나타나기 때문이다.

그렇다면 자동차가 '움직인다'라고 말할 때 그러한 움직임은 어디를 기준으로 해서 움직인다고
하는 것일까. 직선 고속도로를 달리고 있는 자동차의 경우를 다시 생각해 보자. 이 경우 차의
운동의 기준은 간단히 뒷차창으로 보이는 먼 산이라고 할 수 있다. 그러므로 차는 그 산에 대해
일정한 속도로 '움직이고' 있다. 그런데 만약 운전자가 커브를 트는 순간 기적(?)이 일어나서 먼
산이 차가 커브를 트는 반대 방향으로 움직였다고 하자. 이왕 내친 김에 거짓말을 더 보태서, 그
때 산이 움직이는 속도는 운전자에게 보이는 뒷차창 속에서 움직이는 산의 움직이는  속도를
상쇄해서, 커브를 트는 순간 뒷차창의 산은 정지된 영화의 화면처럼 전혀 움직임이 없었다고
하자. 이 경우 커브를 트는 순간에도 운전자의 입장에서 보면 자신은 여전히 직선운동을 하고
있다. 따라서 그가 먼 산에 대해 느끼는 관성은 0이고 그는 몸의 쏠림을 전혀 느끼지 못한다는
엉뚱한 결론이 나온다.

멀리 떨어진 기준점의 움직임이 바로 이곳에서 벌어지는 물리현상에 영향을 주게 된다는
것이다. 물론 우리가 아는 자연법칙은 이것을 용납하지 않는다. 따라서 오직 이러한 혼란을
방지하는 유일한 방법은 울리가 '이곳에서 어떻게 움직이든' 저쪽의 '항상 잘 고정되어 있는'
그러한 기준점이 존재해야 한다. 바로 이런 이유로 뉴턴의 관성계가 갖는 기준점을 절대적으로
정지해 있는 '물체 주변의 공건'으로 정하고 이를 '절대공간'이라고 이름지었다. 이런 이유로
우리가 고속도로에서 커브를 틀면 반드시 관성을  느끼게 되는 것이다. 이것이 바로 관성운동에
대한 뉴턴의 설명이다.

아인슈타인은 그러한 절대 정지공간이나 이에 대해 정의된 물체의 위치라는 개념은 전혀
미혹적인 허상에 불과하다고 주장했다. 그는 이러한 개념은 순전히 인위적인 것으로 절대
정지공간은  우리 우주 어느 곳에도 존재하고 있지 않다고 말했다. 이를 증명하기 위해서 다음과
같은 아인슈타인의 사고실험을 보자.

먼 훗날 달 표면을 달리는 은하철도 999호가 있다고 하자. 달 표면이기 때문에 공기의 저항은
전혀 없다. 은하철도 999호는 시종  일정한 속도로 움직이고  있다. 그런데 손님 중의 하나가
동전을 창 밖으로 던졌다.

물론 그에게는 동전이 그냥 수직 방향으로 땅에 떨어지는 것처럼 보인다. 그런데 그 순간
떨어지는 동전을 기차 밖에서 목격한 다른 관측자가 있었다고 하자. 이 경우 그에게는 기차의
속도 때문에 동전은 포물선을 그으며 땅에 떨어지는 것처럼 보인다.

여기까지는 아무것도 이상할 게 없다. 그런데 떨어지는 동전의 진짜 '위치'는 어디였느냐고
물어 보자. 떨어지는 동전은 낙하하는 수직선상에 있었던가, 아니면 포물선 위에 있었던가.
여기서  우리는 이상한 모순점에 봉착하게 되는 것이다. 한 개의 동전이 동시에 두 곳에 존재할
수 없는 것은 확실하다. 그렇다고 해서 떨어지는 동전의 궤도가 직선상에 있다고 단언할 수도,
포물선 위에 있다고 단언할 수도 없는 것이다. 또한 동전이 직선과 포물선으로 동시에 보이는
장소에 있다고 말할 수도 없는 것이다. 실상 동전이 동시에 직선과 포물선으로 떨어지는 점은
우리 우주 어디에도 존재하지 않는다.

결론적으로, 우리는 관측자의 위치에 상관없이 같게 보이는, 어떠한 '절대적 의'를 갖는
물체의 궤도란 존재하지 않는다는 것을 알게 되었다. 다시 말해서 운동하는 물체의 위치란 특정
관측자가 속한 관성계를 기준으로 해서만 정의되는 양이다.

\beginbf
  관측자에 상관없이 어디에서나 존재하는 절대 궤도 또는 이것이 정의되는 절대기준점 또는
절대공간은 어디에도 존재할  수 없다고 결론지을 수 있다.
\endbf

삼라만상 물체의 역학적 상태란 항상 주어진 기준계에 따라 정의되는 양인 것이다. 기차에
고정된 관성계에서 기술된 동전의 운동은 직선이요, 달 표면에 고정된  관성계에서 본 동전의
운동은 포물선인 것이다.

물체의 궤도가 관측자의 운동 상태에 상관없는 절대적인 개념이 아니고 관측자에 따른
상대적인 개념이라는 것이 밝혀진  이상, 물체의 위치를 나타내는 거리라는 개념도 관측자에
따라 정해지는 상대적인 양에 불과하다는, 아주 중요한 결론에 다다른다. 즉 지구와 태양까지의
거리가 1억5000만km라는 말은 지구에 있는 지구 표면에 정지된 관측자가 잰 거리가 그렇다는
말이다. 관측자의 운동 상태에 따라 물체의 궤적이 직선으로, 또는 포물선으로 되는 것처럼 이
거리는 관측자의 운동 상태에 따라 달라질 수 있다. 아인슈타인의 주장에 따르면 관측자의
운동 상태에 관계없이 항상 1억5000만km로 재어지는 거리란 우리 우주의 어디에도 존재하지
않는다.

모든 물체의 운동 기준점으로 삼아 왔던 절대공간이라는 개념이 우리 우주에 존재하지
않음으로써 우리가 흔히 일상적으로 알아 왔던 거리(또는 공간)란 순전히 상대적으로 정의되는
상대적인 개념임을 알게 되었다. 어느 사람이 그가 들고 있는 줄자의 길이가 1m라고 우기면 그
1m라는 길이는 줄자를 들고 있는 그 사람이 속한 관성계에서만 인정되는 길이인 것이다.

마찬가지로 아인슈타인은 시간 역시 비슷한 성질을 가지고 있다고 주장했다. 시간 역시 특정
관성계에 속하는 관측자가 가진 손목시계에 따라 정의되는 양으로 관측자가 속한  관성계의
운동 상태에 상관없이 항상 우주적으로 동시에 존재하는 시간이란 어디에도 존재하지 않는다는
것이다.

예를 들어 어느 테러리스트가 시한폭탄을 지구상 어디에 장치했다고 하자. 그는 폭탄이
장치 후 1시간 후에 터지게 해놨다. 그런데 이런 끔찍한 범행을 지구에서 200만년 떨어진
안드로메다 은하의 어느 외계인이 목격했다면 그는 폭탄이 터지는 시간이 1시간이 아닌 다른
값이라고 믿게 되는 것이다. 달리는 기차에서 떨어지는 동전의 궤도가 관측자가 속한
관성계의 운동 상태에 따라 직선도 되고 포물선도 되는 것을 상기하자. 안드로메다 은하 역시
지구라는 관성계에 대해 약 초속 30km의 상대속도로 우주 공간상을 여행하고 있는 또 다른
관성계이기 때문이다. 다시 말해서 지구와 안드로메다를 포괄하며 우주적으로 '동시'에
째깍거리는 우주시계, 절대시간이라는  개념은 존재하지 않는다.

이러한 주장을 뒷받침하기 위해 아인슈타인은 우선 일상생활에서 흔히 쓰는
동시성(simultaneity)이라는 개념을 비판했다.

보이저 인공위성이 목성을 향해 날아가던 도중 무슨 이유에서인가 위성 내 기계를 움직이는
모터에 이상이 생겼다는 것을 발견한 사건이 있었다. 이를 발견한 지상국에서는 이를 교정하는
컴퓨터 프로그램을 보이저 위성에 송신했는데 무려 두 시간이 넘어서 위성에 도달했다. 빛의
전파속도가 아무리 빨라 보여도 광활한 우주에서는 별로 빠른 느낌이 들지 않는다.

여기서 주목할 점은 지상국에서 새 컴퓨터 자료를 송신한 순간과 보이저 위성이 그 자료를
수신한 순간 사이에 존재하는 두 시간이라는 시간 간격이 우리가 수천 년간 의심없이 믿어 왔던
동시성이라는 개념을 완전히 파괴하고 있다는 점이다. 지상 교신국과 보이저 위성 간에는 광활한
거리가 존재하는 이유 때문에 지상국과 보이저 위성 사이에는 절대로 동시라는 개념이 존재할 수
없는 것이다.

좀더 다른 예를 들어보자. 독자들을 다시 우리의 친밀한 기차로 초대한다. 기차의 객실
한가운데에 승객이 앉아 있었다. 그런데 갑자기 기차의 진행 방향 앞쪽과 뒷쪽에서 두 개의
섬광이 발생했다. 여기서 우리는 그 섬광들이 발생한 위치가 기차에 앉아 있는 사람으로부터
같은 거리에 있다고 가정하자. 그러므로 기차 밖에서, 두 섬광에서 같은 거리에 위치한
사람에게는 그 두 섬광이 동시에 일어난 것처럼 보일 것이다. 그러나 기차 안에 앉아 있는
승객의 경우에는 상황이
다르다. 왜냐하면 그가 타고 있는 기차는 섬광이 발생한 순간에 시시각각 앞쪽의 섬광을 향해
다가가고 있었고, 또한 뒷쪽 섬광에 대해서는 멀어지고 있었기 때문이다. 당연히 그는 기차
앞쪽에서 발생한 섬광을 먼저 보게 되고, 그 다음에 뒷쪽에서 발생한 섬광을 보게 될 것이다.
물론 빛의 속도가 무한하다면 이러한 조그마한 시간의 차이는 별로 의미가 없을 것이다.

그러나 맥스웰 방정식으로 빛의 속도가 유한한 것으로 밝혀진 이상 마이켈슨과 몰리 같은
사람들이라면 충분히 이러한 섬광의 발생 시간 차이를 측정해 낼 것이다. 따라서 기차 안의
승객은 앞쪽 섬광이 먼저 발생했고, 뒷쪽 섬광이 늦게 발생했다고 결론을 내릴 것이다. 결국
기차 밖에 있는 사람에게 '동시'에 일어난 사건이라도 움직이는 기차 안에 있는 사람에게는
절대로 '동시'가 아니다. 즉 동시라는 개념은 순전히 관측자의 운동 상태에 따르는 상대적인
개념으로 나타나는 것이다.

\beginbf
  우주의 전역에 걸쳐 동시에 째깍거리는 어떤 절대시간이라는 개념은 존재하지 않는다.
시간이라는 개념은 관측자가 속해 있는 관성계에 따라 정의되는 상대적인 양에 불과하다.
\endbf

그런데 여기서 이렇게 반론하는 독자도 있을 수 있다. 기차 밖에 앉아 있는 관측자에게는
동시라는 개념이 정의되고 있지 않은가. 기차 앞뒤에서 일어난 두개의 섬광까지의 거리가 같다면
두개의 섬광은 앉아 있는 사람에게 동시에 전파될 것이 아닌가. 그러므로 기차 밖에 앉아 있는
관측자에게는 동시라는 개념이 아주 잘 정의되고 있는 개념이 아닌가. 물론 얼핏 듣기엔 맞는
말 같다.

이러한 반론에 아인슈타인은 다음과 같이 대답했다. 그러한 동시성이 정말로 성립되려면
앉아 있는 사람이 과연 두 개의 섬광이 정말로 같은 속도로 그에게 도달했는가를 확인할 수
있어야 한다. 이를 위해서는 먼저 전파되는 섬과의 속도를 재기 위한 두개의 시계가 필요하다.
그리고 두 개의 시계가 섬광들이 앉아 있는 사람에게 도달했을 때 정확하게 같은 시간을 읽고
있었다는 것을 증명하기 위해서는 초기 조건으로 섬광이 발생한 두 지역에 미리 놓아둔 두 개의
시계를 '동시에' '같은 시간'으로 맞출 수 있어야 한다. 그런데 문제는 같은 시간이라는 것을
증명하기 위해서는 또 다른 동시에라는 초기 조건이 필요하게 되는 점이다. 따라서 이것은
그냥 맴도는 논리만 낳는다.

이러한 반론을 통해 아인슈타인은 공간상 떨어진 두 점 사이에는 우리가 수천 년간 의심없이
사용해 왔던 동시라는 개념이 절대로 존재할 수 없기 때문에, 전우주적으로 동시에 째깍거리는
절대시간이라는 개념은 허구에 불과하다는 것을 증명해 보였다. 오직 존재하는 시간이란 각
관성계에 있는 관측자가 갖고 있는 시계가 읽는 상대시간뿐이다. 그러므로 우주에 존재하는
관성계의 숫자만큼 많은 수의 시계가 우주에 존재한다고 말할 수 있는 것이다. 바로 이런
맥락에서 "나는 우주 공간을 시계로 가득 채웠다. 그런데 정작 내 방에는 쓸 만한 시계가 하나도
없다"라는 아인슈타인의 농담을 이해할 수 있다.

\section 축지법은 가능하다

여기까지 논리의 전개를 조심스레 따라온 독자는 비로소 뉴턴이 아무런 비판 없이 존재한다고
인정했던 절대공간이나 절대시간이라는 개념이 절대허구임을 알게 되었을 것이다. 그렇다면
이러한 절대공간과 절대시간 개념이 파괴될 때 어떠한 결과가 야기될까. 첫번째 결과는 시간과
공간의 탄력성이다. 즉 시간과 공간은 마치 고무줄처럼 늘어나기도 하고 줄어들기도 한다는
것이다.

우선 아인슈타인은, 관측자가 속한 관성계에 대한 시간의 종속성이 앞에서 나타났던
상대성원칙과 빛의 전파속성 간에 나오는 갈등을 풀어 주는 실마리가 됨을 간과했다. 앞서
승객이 지면에 대해 V라는 속도로 달리는 기차에서 회중전등을 앞방향으로 비추었을 때
지면에 정지해 있는 사람이 본 불빛의 속도는 단순히 빛의 속도에 기차의 속도를 합한 값
$W = C + V$, 또는 $W = C - A$가 되지 않고 $W=C$였던
것을 상기해 보자. 빛의 경우 단순한
속도의 합산법칙이 전혀 맞지 않는 이러한 기묘한 현상은 마이켈슨과 몰리의 실험에 의해
충분히 증명되었음을 우리는 안다.

이에 대한 아인슈타인의 설명은 다음과 같다. 우리는 동시성의 개념이 파괴됨에 따라 각
관측자마다 그가 속한 관성계 고유의 시계가 정의되고 있음을 알게 되었다. 그러므로 기차에
앉아 있는 사람이 느끼는 시간과, 기차 밖에 서 있는 사람이 느끼는 시간은 전혀 다른
시간들이다. 즉 달리는 기차에서 앞으로 던져진 동전의 속도란 순전히 이를 던진 사람, 즉
기차에 탄 사람의 손목시계에 의해서만 정해지는 양이다. 그러므로 달리는 기차에서 앞쪽으로
비춘 빛이 퍼져 가는 속도 역시 기차에서 회중전등을 들고 있는 사람의 손목시계에 의해서
정의되는 양인 것이다. 이러한 시간과 속도의 양은 절대로 기차 밖에 서 있는 사람의
손목시계로 정의되는 양이 아니다. 그러므로 달리는 기차에서 퍼져 가는 빛의 속도를 기차
밖에 서 있는 사람의 입장에서 그냥 $C+V$,  또는 $C-V$라고 우긴다면 그는 두
개의 전혀 다른 시간을 통해 정의된 물리량을 그냥 막무가내로 합산해 버린 오류를 범한 셈이
된다.

물론 이러한 시간개념의 상대성은 기차나 동전의 속도가 빛의 속도보다 매우 느린 경우에는
거의 무시할 만하다. 관성계간의 상대속도가 빛의 속도보다 훨씬 적은 우리 일상생활의 경우
이러한 시간에 대한 구분은 거의 무시해도 된다. 아폴로 우주선이 초속 10km로 움직일 때
우주선 안에 있는 조종사가 찬 시계의 째깍거리는 시간 간격은 지구상에 있는 어떤 시계의
시간 간격과 거의 같다. 그러나 아폴로 우주선의 속도가 빛의 속도에 다가가는 경우 구개의
시계가 째깍거리는 간격의 차이는 점점 뚜렷해진다.

홍길동은 축지법의 명인이다. 옛날 이야기에 많이 나오는 축지법은 문자 그대로 두 점 사이의
거리를 고무줄처럼 당겨 놓고 건너뛴 다음 이를 다시 원래대로 펴, 손쉽게 먼 거리를 여행하는
방법이다. 이와 같은 공간 축지법(space warp)은 "스타트렉(star trek)"이라는 인기영화에서도
나오는데 이를 보면 수시로 우주선 선장이 공간을 구부리는(warp) 속도로 비행선을 달리라는
명령을 내린다.

이는 전혀 비과학적인 환상이 아니다. 특수상대성이론에 따르면 공간을 실제로 구부릴 수
잇고, 이를 잘 이용한다면 수천억 광년의 거리를 순식간에, 마치 옆동네 다녀오듯 여행할 수
있는 것이다. 다음과 같은 예를 통해 이를 좀더 명확히 설명해 보자.

어느 순간, 은하철도 999호에 앉아 있던 승객이 차창의 가로 크기를 재고 싶어했다. 그래서
그는 갖고 있는 자를 가지고 창의 가로 거리를 쟀다. 창의 길이는 60cm로 나왔다고 가정하자.
그런데 마침 달리는 기차밖에 서 있던 어떤 사람이 기차 안에 있던 사람과 똑같은 자를 달리는
기차의 창에 스칠 듯 말듯 가깝게 대주었다. 그래서 기차 안에 있던 승객은 창밖을 스쳐
지나가는 자의 눈금을 통해 창의 크기를 또 한번 읽었다. 이 경우 창밖의 사람이 들어준 자를
통해 잰 기차 창문의 크기는 자신이 잰 60cm와 다르게 나타난다. 다시 말해서 기차 안에,
그리고 기차 밖에라는 두 개의 다른 관성계에 있는 같은 길이의 자라도 두 개 관성계간의
상대속도에 따라 다른 길이로 읽혀지는 것이다.

이렇게 두 개의 관성계간 상대속도에 따라 주어진 자의 길이가 고무줄처럼 늘어나고 수축하는
현상을 처음 발견한 사람의 이름을 따 로렌츠-피츠제럴드 길이단축현상(Lorentz-FitzGerald
Length Contraction Phenomenon)이라 부른다. 이러한 현상 역시 기차의 속도가 빛의 속도에
비해 매우 작은 일상생활에서는 거의 무시할 만하다. 그러나 기차의 속도가 차츰 빛의 속도에
가까이 가면 갈수록 그러한 자의 늘어남이나 수축현상은 점점 더 두드러지게 된다.

이쯤 되면 독자들도 '아하! 그래서 축지법이 가능하겠구나!'하고 이해가 될 것이다. 1000억
광년 떨어진 거리를 빛의 속도에 가까운 로켓으로 여행하자면 거의 1000억년의 세월이 걸린다.
그러나 우주선의 속도가 빛의 속도에 접근해 가면 접근할수록 우주선 안의 시계는 점점 더
천천히 간다. 예를 들면 우주선 안에 있는 승무원에게는 1000억년의 시간이 그보다 훨씬 짧은,
단지 몇 년의 세월로 느껴지는 것이다. 그러므로 지구에서 무려 1000억 광년으로 측정된
거리를 그 승무원은 겨우 몇 광년의 거리로 축지해서 여행하는 것이다.

\section 빛의 속도는 일정하다

앞의 장들을 통해, 우리는 빛의 전파속성과 상대성과의 갈등 원인은 우리가 무비판적으로
다음과 같은 사실을 받아들인 데 있었다는 것을 알았다.
\item{(1)} 주어진 두 사건간에 경과한 시간은 그 사건이 일어난 관성계(즉 열차계 또는 선로계)에
상관없이 항상 같다는 절대시간이 존재한다.
\item{(2)} 주어진 두 점 사이의 거리는 두 점이 속한 관성계에 관계없이 항상 같게 측정된다.

  그러므로 만약에 우리가 이러한 절대시간과 절대공간이라는 개념만 포기하면 앞서 말한
모순성은 사라진다. 이 경우 두개의 관성계간에 존재하는 속도의 단순한 산술적 합산은 본래의
물리적 의미를 상실하기 때문이다.

  이러한 절대공간과 절대시간의 존재가 부정되면, 필연적으로 빛의 전파속도는 모든 관성계에서
같다는 빛의 이상한 성질, 관성계의 속도에 따라 나타나는 시간지체현상, 거리단축현상, 이 모든
현상들이 한꺼번에 설명된다. 다시 말해서 이들은 특수상대성이론의 필연적인 결과인 것이다.
그러므로 남은 과제는, 이러한 현상들을 수학적으로 증명해 주는 관성계간의 좌표변환식을
찾아내는 작업이다.

  실상 이 변환식은 특수상대성이론이 나오기 이전에 이미 독일의 로렌츠라는 사람에 의해
발견돼 존재하고 있었다. 그래서 아인슈타인은 특수상대성이론을 제창할 당시 이러한 변환식을
유도하는 복잡한 과정을 겪을 필요가 전혀 없었다. 그로서는 매우 운이 좋았던 셈이다. 그래서
비록 이 변환식이 특수상대성이론의 핵심 구조로 모든 교재에서 소개되고 있으나 그 이름만은
최초 발견자의 이름을 따서 로렌츠변환식(Lorentz Transformation)이라 부른다. 물론 이 장의
주목적은 이러한 모든 사실들을 수학적으로 증명해 주는 로렌츠변환식을 소개하는 데 있다.
따라서 수학에 자신 없는 독자들은 이 장과 다음 장을 빠뜨린다 해도 이 책을 이해하는 데
아무런 무리가 없을 것이다.

  로렌츠변환식을 가장 쉽게 이해하는 방법은 다음과 같다. 예를 들어 서울 남산 꼭대기에 어떤
사람이 앉아 있다 하자. 그의 위치와 시간을 공간좌표 X, Y, Z와 시간 T로 표시한다. 그러면
그의 좌표는
$$\hbox{서울 남산 꼭대기 기준계}:X, Y, Z, \hbox{시간은} T$$
  이다. 마찬가지로 이 순간 어느 방향으로 움직이는 특정 기차에 앉아 있는 승객의 위치와
시간은
$$\hbox{승객 기준계}:X', Y', Z', \hbox{시간은} T'$$
  이다. 여기에서 좀더 구체적으로 기차는 X축 방향으로 V라는 속도로 움직인다고 가정해 보자.
그러면 그러한 일정한 속도 V로 움직이는 기차의 관성계와 그에 대해 정지해 있는 관측자의
관성계를 다음가 같은 도식으로 나타낼 수 있다.
  (그래프 생략)

  이 경우 부록에 유도된 로렌츠의 변환공식에 따르면 다음과 같은 변환관계가 위의 두 개
관성계 사이에 존재한다.
$$
\eqalign{
   X' &={X-VT\over\sqrt{1-\Big({V\over C}\Big)^2}}\cr
   Y' &= Y\cr
   Z' &= Z\cr
   T' &={T-\Big({VX\over C^2}\Big)\over\sqrt{1-\Big({V\over C}\Big)^2}}}
$$
  이것이 바로 유명한 로렌츠변환식이다. 이 변환식의 의미는 다음과 같다. 
여기서 만약에 $X$축을
기차의 남북간 위치라고 하면 $Y$와 $Y'$는 각기 그에 수직한, 기차의 동서 거리를 지정하는
좌표이다(물론 우리는 식을 간단히 하기 위하여 기차가 남북방향으로만 움직인다고 가정하였다).
역시 동쪽 좌표나 서쪽 좌표의 변화는 0이므로 그냥 $Y=Y'$로 놓았다. $Z$는 기차의 해발고도
를
가리키는데 역시 기차의 해발고도는 변하지 않았다고 가정하고 $Z=Z'$로 놓았다. 마지막으로
$T$, $T'$는 각각 기차 안의 승객과 서울의 정지한 관측자의 손목시계가 읽는 시간들이다.

  이 공식을 통해 알 수 있는 것은 로렌츠변환식이 앞서 말했던 고전역학적 속도합산법칙을
확장시킨 형태라는 것이다. 다시 말해서 일단 위의 공식에서 나타나는 $V/C$ 항을 무시해 보라.
그러면 위의 로렌츠변환은
$$
\eqalign{
   X' &= X - VT\cr
   Y' &= Y\cr
   Z' &= Z\cr
   T' &= T}
$$
로 바뀐다. 물론 이것은 앞에서 나왔던 단순한 속도합산법칙을 수학적으로 표시한 것에
불과하다. 이렇게 변화된 형태의 로렌츠변환공식을 우리는 갈릴레이변환공식(Galilean
Transformation Law)이라 부른다.

이 시점에서 독자들은 그렇다면 과연 로렌츠변환식이 빛의 전파 속도가 관성계에 상관없이
항상 같은 속도로 나타난다는 것을 보여주는가 하고 묻게 될 것이다. 물론 로렌츠변환식은 이를
보장해 준다. 주어진  시간 $T=T'=0$에 서울에 앉아 있는 관측자가  카메라의
플래시를 터뜨렸다 하자. 그러면 그의 손목시계로 10초가 경과한 후에 플래시로부터 퍼져 간
빛은 $X=CT=C\times10\hbox{초}=300{,}000{\rm km}/\hbox{초}\times
10\hbox{초}=3{,}000{,}000{\rm km}$ 떨어진 곳을 여행하고 있다. 이를
위의 로렌츠변환식에 대입하면 움직이는 기차에 앉아 있는 승객에게는 다음과 같이 표현된다.
$$
\eqalign{
  X'={(C-V)T\over\sqrt{1-\Big({V\over C}\Big)^2}}\cr
  T'={\Big(1-{V\over C}\Big)T\over\sqrt{1-\Big({V\over C}\Big)^2}}}
$$
  따라서 둘째식 $T'$에 빛의 속도 $C$를 곱하면 $X'=CT'=3{,}000{,}000$km가 된다. 
즉 빛이 같은 시간에 여행한 전파 거리는 관성계에 상관없이 항상 같게 나타난다.

\section 진시황의 불로초

앞서 말한 로렌츠의 변환식을 이용하여 정지한 사람에게 움직이는 사람이 가진 자의 길이가
짧아지는 현상을 증명해 보겠다. 앞의 로렌츠변환식에서
$Xm=VT+X'\sqrt{1-(V/C)^2}$
이다. 이 결과를 이해하려면 먼저 기차 안의 승객이 1m짜리 자를 가지고 있다고 하자. 그는
자를 기차가 가는 방향(즉 $X'$방향)으로 평행하게 놓고, 자의 원점을 $X'$축의 원점에 맞추었다.
그러면 주어진 시간 $T'$(승객의 시계가 정한 시간)에 측정한 자의 길이는
$X'= 1{\rm m} - 0{\rm m}= 1{\rm m}$ 이다. 그런데 로렌츠변환식에 따르면
$X=(1{\rm m})\sqrt{1-(V/C)^2}$이다.
놀랍게도 기차 밖에 정지해 있는 목격자에게는 원래의 $X=1{\rm m}$라는
길이가 $\sqrt{1-(V/C)^2}$
만큼 줄어 보이는 것이다. 이것이 바로 유명한 로렌츠-피츠제럴드 길이단축현상이다.

  이러한 단축현상은 이 세상에 존재하는 모든 만물에 적용된다. 아무리 단단한 무쇠로 만들어진
자라도 정지해 있는 사람의 눈에는 원래 1m 길이에서 $\sqrt{1-(V/C)^2}$을 곱한 만큼
줄어 보이는 것이다. 물론 이런 현상은 기차의 속도가 증가할수록 점점 뚜렷해지는데 기차의
속도가 광속에 이르는 경우 자의 길이는 자주 없어져 버린다. 이러한 길이단축현상을 그림을
통해 보면 다음과 같다.

  (그림 생략)

  여기에서 이렇게 묻는 독자들도 있을 것이다. 그렇다면 만약 기차의 속도, 즉 자의 속도가
빛보다 빠른 경우엔 어떻게 될까? 이  경우 $\sqrt{1-(V/C)^2}$의 값은 허수가 된다.
이는 아무런 물리적인 의미가 없는 수로 어떠한 물체도 빛의 속도보다 빠를 수 없다는 중요한
결론을 제공한다. 마찬가지로 기차 밖에 정지해 있는 m의 자의 길이는 기차 안의 승객에게는
역시 $\sqrt{1-(V/C)^2}$를 곱한 만큼 줄어 보인다.  왜냐하면 기차의 승객 입장에서
보면 기차는 정지해 있고 창밖의 풍경이 $V$의 속도로 뒤로 밀려가는 것처럼 보이기 때문이다.
이러한 현상은 상대성 원칙의 정수라고 할 수 있다.

이번에는 두 관성계간에서 시간이 고무줄처럼 늘어나는 현상을 로렌츠변환식을 통해
증명해 보자. 이는 일반 사람들에게는 위의 길이단축현상보다 오히려 더 이상하게 보인다. 바로
이것이 많은 사람들로 하여금 특수상대성이론을 못 미더워 하게 하는 주범이었다.

아까와 마찬가지로, $V$라는 속도로 달리는 기차 안에 어떤 승객이 앉아 있다고 하자. 물론 그의
관성계는 앞서 말한 좌표계$(X', Y', Z', T')$가 된다. 이 경우 승객은 그가 가진 자의 한쪽 끝,
즉 원점에 주저앉아 있다. 그러면  $X'=0$으로 놓을 수 있다. 이 경우 그가 그 자리에 1초간
(즉 $[(T'=1\hbox{초})-(T'=0\hbox{초})]= 1\hbox{초}$) 앉아 있었다고
할 때 기차  밖에 서 있는 사람에게는 원래 1초라는
시간이 $\sqrt{1-(V/C)^2}$만큼 늘어난다.  다시 말하면, 움직이는 물체 안에 있는
시계는 정지해 있는 세계의 시계보다 좀더 늦게 가는 것이다. 이 현상을 도표를 통해 보면
다음과 같다.

(표)

여기서도 우리는 기차의 속도가 광속에 다가감에 따라 시간이 천천히 흐르는 현상이 점점 더
뚜렷해짐을 알 수 있다. 진시황이 끝내 못 찾았던 불사의 진리란 바로 특수상대성이론이었던
것이다. 물론 위의 도표 수치가 보여주듯 이러한 시간지체현상은 우리의 일상생활에서는 거이
느끼지 못할 정도이니 구태여 장수할 요량으로 비행기나 우주선의 승무원직을 택할 필요는 없다.

\section 물질은 에너지 덩어리

$E=mC^2$이라는 식은 특수상대성이론의 여러 공식 중 가장 유명한 것이라 생각된다.
이것은
가공할 원자폭탄과 수소폭탄의 파괴력을 설명해 주고 또한 태양이나 별들이 어떻게 십억년 동안
빛을 발하는가를 설명해 주는 공식이다. 또한 평화적으로 쓰이는 원자에너지, 즉 원자력
발전소에서 나오는 에너지를 설명해 주는 공식이기도 하다.

  아인슈타인 자신은 처음 질량에너지 공식을 발견했을 때 그의 생전에 이런 질량에너지가 우리
일상생활에 실용화될 수 있으리라 전혀 예상치 못했다. 만년에 아인슈타인은 회고하기를 그의
생전에 이런 질량에너지 공식이 실용화되는 것을 볼 가능성이란 ``마치 캄캄한 시골 밤에 총을
쏘아 우연히 지나가던 새가 잡히기를 기원하는$\cdots\cdots$'' 정도로 희박하다고 했다.

  처음 이 공식이 학술지를 통해 발표될 당시인 1910년대는 원자핵 붕괴 현상이 이해되지 않던
시절이었기 때문이다. 그러나 후에 가벼운 알파입자를 무게가 무거운 우라늄 같은 중원소에
쏘이면 우라늄이 붕괴되고 그 과정에서 질량에너지 공식이 예견한 만큼의 에너지가 방출된다는
사실이 확인되면서 이 공식은 갑자기 특수상대성이론의 총아로 부각되었다.

  그러면 이 질량에너지 법칙은 어떻게 유도되었을까. 불행하게도 이것은 앞에서 자주
소개되었던 사고실험과 같은 방법으로 유도되지 않는다. 이 공식은 로렌츠변환식을 고전역학에
직접 응용한 결과로 나타난다. 즉 로렌츠변환식을 고전역학에 응용하면, 고전역학의 물리량들이
여러 가지 다른 형태로 변화되어 나타나는데, 그 대표적인 예가 바로 질량과 에너지다.

  고전역학의 가장 중요한 물리량을 꼽아 보면, 첫째 물체의 질량과 속도를 곱한 양으로 이를
흔히 물체의 운동량이라 부른다. 즉 물체의 질량을 $m$이라 하고 그의 속도가 $V$라 하면 그 물체의
운동량은 $mV$이다. 둘째로 물체의 에너지다. 물체의 에너지는 흔히 $E$로 표기되는데, 이는 물체의
질량과 속도를 알면 금방 정해지는 양이다.

  그런데 이 두 물리량이 올바른 로렌츠변환을 하려면 물체의 총에너지는
$$ E= mC^2$$
  의 형태를 취해야만 한다. 따라서 특수상대성이론을 따르면 질량을 가진 모든 물체는 $mC^2$
이라는 크기의 에너지를 갖는다는 결론이 나온다. 특수상대성이론에 따르면 어떤 물체라도 그의
질량이 0이 아니면 항상 엄청난 크기를 갖는 에너지 덩어리가 되는 것이다.

일반적으로 물체의 속도가 광속에 훨씬 못 미칠 경우, 이와 같은 질량에너지는 절대로 다른
형태 (예를 들면, 감마선이나 다른 소립자)로 변하지 않는다. 그러므로 우리 일상세계에서는
그러한 에너지가 그냥 물체 속에 갇혀 있다고 할 수 있다.

  그러나 이 질량결손 에너지는 인류 역사 이래 지속되었던 의문점, 즉 태양과 별이 어떻게
수십억년간 빛을 발하고 있는가에 대한 해답을 제공한다. 한때 사람들은 태양을 불타고 있는
석탄 덩어리로 생각했고, 또는 칸트 같은 사람은 태양이 자체의 만유인력에 의해 스스로 영원히
수축하고 있는 물체라고 했다.

  그런데 이런 과정에서 태양의 에너지가 얼마나 오랫동안 방출되는가를 계산해 보면 항상
지질학상에 나타난 태양의 연령, 즉 45억년에 훨씬 못 미치는 숫자로 나왔다. 매우 실망스런
결과였다. 그러다가 1930년대에 이르러 핵반응이 이해되고 이를 이용해 사람들은 태양 내부에
핵융합반응이 일어나고 있다고 믿게 되었다. 그래서 별과 태양에너지의 근원이 바로
수소핵융합반응이라고 가정하고 결과적으로 파생되는 질량 결손 에너지를 계산해 본 결과,
태양과 별들이 현재의 밝기로 타는 시간대는 수십억년보다 크게 나왔다. 태양과 별의 에너지
근원은 수소핵융합반응으로 인한 핵연료의 질량결손 에너지라는 것이 자연스레 밝혀진 것이다.
  참고로 이러한 질량결손 에너지가 얼마나 큰지 다음과 같은 예를 통해 알아보자. 간단히
100kg인 역기가 모두 '질량결손 에너지'로 전환된다고 가정해 보자. 그러면
$$
\eqalign{
E &= mC^2\cr
&= (1000{\rm kg})\times(3\times10^8{\rm m}/{\rm sec})^2\cr
&= 9\times10^{19}({\rm joules})}
$$  
이다. 여기에서 줄(joule)은  에너지의 단위인데, $9\times10^{19}$
줄이라는  에너지는, 100와트 전구를 대략 30억년간 켤 수 있는 에너지에 해당한다. 바로 이런
엄청난 양의 에너지가 바로 원자폭탄이나 수소폭탄이 갖는 가공할 파괴력의 근원이다.

\section 달리면 무거워진다

마지막으로 특수상대성이론에 반드시 나타나는 현상으로 별로 잘 알려지지 않은 현상을
소개해 보자.

  아인슈타인은 로렌츠변환식을 통해 시간지체현상이나 길이단축현상에 버금가는 질량증가현상이
있음을 알아내었다. 즉 주어진 관성계에서 시간이 천천히 흘러가듯, 일정한 상대속도로 운동하는
관성계 내 물체의 질량은 원래의 크기에서 $1/\sqrt{1-(V/C)^2}$을  곱한 만큼 증가하는
것이다.
이 현상 역시 물체의 속도가 광속에 다가가면 갈수록 더욱 뚜렷해지는데, 물체 속도가 광속 $C$에 이르면 물체의 질량은 아예 무한대가 된다.

  참고로, 무한대의 질량을 갖는 물체를 광속에 이르게 추진하는 데에는 무한대의 에너지가
필요하게 됨을 유의하자. 그런데 무한대의 추진 에너지는 자연에 존재하지 않으므로, 바로 이
질량증가현상은 물체의 속도가 절대로 광속에 이를 수 없다는 사실을 자연스레 설명하는 이유가
된다.

  뉴턴식 고전역학에서는 관측자가 어떤 임의의 속도로 움직일 수 있었음을 상기하자.
특수상대성이론에 따르면 관측자의 속도가 광속에 다가가면 점점 그의 질량이 증가되어
가속하기가 점점 힘들어지는 것이다.
  이러한 관측자의 속도를 광속에 이르지 못하게끔 작용하는 질량을 그래프를 통해 보면 다음과
같다.

  (그래프 생략)

  그럼 지금까지 나타난 특수상대성이론의 대표적인 결과들을 다음과 같이 정리해 본다. 정지해
있는 관측자에 대해 일정한 상대속도로 움직이는 기차에 탄 승객의 경우, 그의 진행 방향 길이는
줄어들고, 그의 손목시계는 느리게 가며, 그의 질량은 늘어난다.

\section 시간은 감각이다

  지금까지 우리는 수많은 가상 열차기행을 해왔다. 저자는 교통 체증이 없는 열차나 지하철을
애용하는 고객이다. 그 이유로 열차나 지하철은 책을 읽기에 안성맞춤이기 때문이다. 그래서
도쿄와 서울 지하철의 경우 많은 사람들이 책을 읽고 있는 것을 본다. 심지어 미국사람들도
보스턴이나 샌프란시스코의 지하철 구내에선 책을 읽고 있다. 따라서 열차를 이용하는
독자들께서는 열차가 움직이고, 정거할 때마다 지금까지 가상 열차기행을 통해 알아본
상대성이론의 결론들을 하나하나 되새겨 볼 것을 권한다.

  상대론원칙은 항상 맞다. 즉 서로 일정한 속도로 운동하는 관성계 내에서 행해진 모든
물리 실험의 결과는 같다. 따라서 일정한 상대속도로 움직이는 열차 안의 승객들에게는 그
열차가 일정한 속도로 움직이는지, 아니면 정지해 있는지를 밝혀 줄 실험 방법은 절대로
존재하지 않는다. 별도로, 지난 200여년간 알아 온 우리의 전자기학적, 광학적 지시에 따르면
빛의 진공 전파속도는 유한하고, 그 값은 어떠한 관성계에서도 같다.

  아인슈타인은 이 두 자연의 법칙을 위반하는 현상은 우리가 사는 우주 어디에도 존재하지
않는다고 했다. 그런데 막상 이러한 두개 기본 원칙이 상호모순 구조를 보이자, 그는 문제의 근본
원인이 잘못 이해되어 왔던 시간과 거리의 개념에 있음을 발견했다.

  특수상대성이론의 가장 중요한 결론 중의 하나는 뉴턴이 도입한 절대공간에 대한 철저한
부정이다. 특수상대성이론은 독일의 유명한 수학자이며, 철학자인 라이프니츠(Leibniz)의 다음과
같은 말을 증명했다.

\beginemph
존재하는 것은 이미 존재하는 물질간의 상호작용일 뿐, 존재하는 물질이 아무것도 없는
절대공간이란 그 자체로 아무런 의미가 없다.
\endemph

  여기서 한발 더 나아간 아인슈타인은 기존의 절대시간 개념도 파괴시켰다. 우주 전역에 걸쳐
흐르는 그러한 절대적 시간개념은 결코 존재하지 않는다. 그는 이를 동시성이라는 개념이
존재하지 않음을 보임으로써 증명했다. 그러므로 한 좌표계에서 잰 거리와 시간은, 다른
좌표계에서 볼 때에는 전혀 다른 값으로 나타날 수 있다. 현재까지 지구상에서 행해진 모든 실험
중 이렇게 시간과 거리가 고무줄처럼 늘어났다 줄었다 하는 관계, 좀더 자세히 말해서 이러한
증명해 주는 로렌츠의 변환식을 위반하는 사례는 한 건도 없었다. 물론 만에 하나 이러한 사례가
발견된다면 이는 언제라도 물리학에 새로운 풍파를 일으킬 수 있다.

  마지막으로 필자는 가끔 이런 질문을 받는다.

\beginemph
그렇다면 시간이란 무엇인가?
\endemph

  결론부터 말하면, 시간은 우리가 느끼는 감각에 불과하다. 마치 우리가 무지개의 색을 눈을
망막을 통해 느끼듯, 시간 역시 우리의 인식을 통해 느끼는 아주 인위적인 감각에 불과한
것이다. 예를 들면 시간이 과거에서 미래로 흐른다는 인식은 우리가 시계나 달력을 통한 숫자의
변화를 통해 읽은 인식의 결과에 불과한 것이다. 꼬리를 물고 일어나는 사건들의 경과를
시계라는 기계를 통해 표식을 하다 보니 우리는 자신도 모르게 일어나는 모든 사건들이 어떠한
절대적인 시간이 흐르고 있음으로써 진행되고 있다고 믿게 된 것이다.

  이를 좀더 쉽게 이해하는 방법으로 다음과 같은 예를 생각해 보자. 어느 순간, 갑자기 우주
전체에서 시간이 사라졌다고 하자. 즉 우주의 운행을 감지하는 모든 관측자들이 없어지고,
그들이 만든 시계나 달력들이 없어지고, 심지어 신의 손목시계까지 압수 당했다(?) 하자. 이
경우에도 지구는 태양 주위를 계속 도는 것이다. 우주의 전역에서 째깍거리며 사건의 전개를
통괄해 주는 어떠한 절대적 시간이 있든 없든 간에, 그래도 지구는 돈다.

  바로 이런 이유로 시간은 우리가 '느끼는' 감각에 불과한 것이다. 아테나여신의 유방이
아름답다고 느끼는 그러한 감각에 속하는 것이다.

\section 침묵 깬 플랑크의 편지

  세계의 어느 대학, 어느 도서관을 가 보아도 1905년판 독일의 학술지 "물리학 연감 (Annalen
Der Physik)" 중 특수상대성이론이 실린 달의 잡지는 눈에 보이지 않는다. 왜냐하면
특수상대성이론이 실린 달의 잡지는 이미 도난을 당했거나, 아니면 이를 염려하여 엄중하게
경호하는 특수한 장소에 보관돼 있기 때문이다. 아인슈타인의 특수상대성이론이 유명해지자
그것이 처음 실린 학술지 자체가 수집가들의 귀중한 수집대상으로 변해 버린 탓이다. 저자 역시
미국에서 손꼽힐 정도로 많은 장서를 자랑하는 펜실바니아대학의 도서관에서 이 잡지를 찾아보려
했으나 결국 마이크로 필름에 수록된 복사본으로 만족해야만 했다.

  아인슈타인은 특수상대성이론을 그가 26세 때인 1905년 $\langle$움직이는 물체의 전자기학(Zur
Elektrodynamik Bewegter Korper)$\rangle$이라는 제목으로 발표하였다.

  그의 누이동생인 마야 아인슈타인(Maja Einstein)의 회고에 따르면 그는 논문이 가장 권위
있는 학술지에 발표되었으므로 즉각적인 학회의 반응이 있으리라 기대했다고 한다. 또한 논문의
내용이 상식을 뛰어넘는 면이 많았으므로 이에 대한 즉각적인 반발이나 심한 비평이 따를 것이라
예상했다는 것이다.

  그러나 논문이 출간된지 한달, 두 달이 지나도 학계로부터의 반향은 전무했다. 바로 무덤 속의
침묵이었다. 이러한 차가운 침묵에 젊은 아인슈타인은 매우 실망했다고 마야 아인슈타인은
전한다.

  그러던 어느 날 그는 베를린에서 온 편지 한 통을 받았다. 발신인은 현대 양자론을 탄생시킨
장본인인 유명한 막스 플랑크(Max Planck)교수였다. 편지 내용은 출간된 상대론 논문을
이해하기가 힘드니 좀더 자세히 설명해 달라는 것이었다. 이는 정말로 오랜 침묵을 깬
학계로부터의 첫 반응이었다. 마야 아인슈타인의 회상을 빌리면 아인슈타인은 이 편지를 받고
뛸 듯이 기뻐했다고 한다. 그도 그럴 것이, 이는 최소한 그의 논문을 다른 사람이 읽었다는
의미였고, 당시 최고의 학자로 손꼽히던 플랑크 교수로부터 긍정적 반응을 받았다는 신호였기
때문이다.

  플랑크 교수가 그에게 편지를 보냈다는 소문이 퍼지자 갑자기 상대성이론에 대해 묻는 많은
편지들이 보내지기 시작했다. 지난 몇 달 동안 모든 사람들이 서로의 눈치만 보며 침묵을 지키고
있었다는 의미였다.

  현재까지 알려진 특수상대성이론의 첫 신봉자는 플랑크 교수다. 그는 특수상대성이론을 최초로
이해한 사람으로 알려져 있고, 또한 특수상대성이론에 관한 첫 물리학 강좌를 대학교에서 개설한
사람이다. 그리고 특수상대성이론에 관한 첫 박사학위논문 역시 플랑크 교수의 지도 아래
나왔다.

\bye
