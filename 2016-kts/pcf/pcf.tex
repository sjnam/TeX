%
% http://mathworld.wolfram.com/PeriodicContinuedFraction.html
%

\documentclass{article}

\usepackage{luacode}

\begin{luacode*}
  local ffi = require "ffi"
  local ffi_new = ffi.new
  local sqrt = math.sqrt
  local tonumber = tonumber
  local concat = table.concat
  local sformat = string.format
  local uint8_type = ffi.typeof("uint8_t")
  
  function not_square(n)
     local s = tonumber(ffi.new(uint8_type, sqrt(n)))
     return s*s ~= n
  end

  function periodCF(n)
     local period = {}
     local a0 = tonumber(ffi.new(uint8_type, sqrt(n)))
     local m, d = a0, n-a0*a0
     if d == 0 then return a0 end
  
     local m0, d0 = m, d
     repeat
        local a = ffi.new(uint8_type, (a0+m)/d)
        period[#period+1] = tonumber(a)
        m = d*a - m
        d = (n-m*m)/d
     until m==m0 and d==d0

     return sformat("[%d;\\overline{%s}]",a0,concat(period,','))
  end
\end{luacode*}

\newcommand*{\periodCF}[2]{%
  \begin{tabular}{rl}
    \luaexec{
      local sprint = tex.sprint
      for j=#1,#2 do
         if not_square(j) then 
            sprint("$\\sqrt{"..j.."}$&$"..periodCF(j).."$\\\\")
         end
      end
    }%
  \end{tabular}
}  

\begin{document}

\section*{Periodic continued fraction}
A periodic continued fraction $[a_0;a_1,a_2,\ldots]$ is periodic
if the sequence eventually repeats, 
i.e there exists some $m$, $n$ with $a_{m+i}=a_{n+i}$ for all $i\ge0.$
A continued fraction,
$[1;2,2,...],$ is periodic and turned out to be $\sqrt2.$

If $r > 1$ is a rational number that is not a perfect square, then
$$\sqrt r = [a_0;\overline{a_1,a_2,\ldots,a_2,a_1,2a_0}].$$

In particular, if $r$ is any non-square positive integer, 
the regular continued fraction expansion of $\sqrt r$ contains 
a repeating block of length $m$, in which the first 
$m-1$ partial denominators form a palindromic string.

\bigskip

\hskip-3cm
\hbox{\hss
  \periodCF{1}{35}~\periodCF{37}{68}~\periodCF{69}{99}
  \hss}

\end{document}

