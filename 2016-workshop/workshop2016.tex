% 2016 공주대 워크숍
% 텍 매크로 작성법

\documentclass{beamer}

\usetheme{metropolis}
\metroset{outer/progressbar=head}

\usepackage{fancyvrb}
\usepackage{kotex}
\hypersetup{pdfencoding=auto}


% title
\title{\TeX\ 매크로}
\subtitle{매크로 작성의 기초와 예제}
\date{2016년 11월 5일 토요일}
\author{남수진}
\institute{
  2016 공주대학교 문서작성 워크숍 2016\\
  공주대학교 인문사회과학관 1층 컴퓨터실 107호}
\titlegraphic{\hfill\includegraphics[height=3cm]{meta.pdf}}


%%
\begin{document}

\maketitle


%
\begin{frame}{매크로 관련 명령어}
  \vspace{4mm}
  \hbox to\hsize{\hss\includegraphics[height=7.5cm]{cs.png}\hss}
\end{frame}


%
\begin{frame}[fragile]{매크로 정의}
  \medskip
  \hbox to\hsize{\hss
    \verb+\def+<control sequence>\alert{<parameter text>}%
    \verb+{+\alert{<replacement text>}\verb+}+\hss}
  \smallskip
  \begin{itemize}
  \item 문서에서 여러 번 반복되어 사용되는 문구나 일련의 명령어의 나열을
    하나의 명령어(control sequence)로 만든 것
\begin{Verbatim*}[fontsize=\small, formatcom=\color{blue}]
\def\xvec{(x_1,\ldots,x_n)}
\def\row#1{(#1_1,\ldots,#1_n)}
\def\cs #1. #2\par{...}
\cs You owe \$5.00. Pay it.\par
\end{Verbatim*}
  \item \verb+\def+, \verb+\gdef+, \verb+\edef+, \verb+\xdef+
  \item 텍의 처리 과정 중에서, \alert{전개 과정}에서 매크로는 치환 텍스트로 전개된다.
  \end{itemize}
\end{frame}


%
\begin{frame}{텍의 처리 과정}  
  \begin{itemize}
  \item \alert{\bf 입력(Input)} 파일을 줄(line) 단위로 읽어서
    \alert{토큰 리스트}를 만든다.
  \item \alert{\bf 전개(Expansion)} 위의 토큰 리스트를 입력으로
    받어서 전개할 수 있는
    모든 토큰을 전개해서 더이상 전개 할 수 없는 토큰들로 구성된
    새로운 \alert{토큰 리스트}를 만든다.
  \item \alert{\bf 실행(Execution)}
  \item \alert{\bf 출력(Visual)}
  \end{itemize}
\end{frame}


%
\begin{frame}[fragile]{토큰 리스트}
  \alert{입력 과정}

  \begin{itemize}
  \item 연속된 여러 개의 공백 문자는 하나의 공백 문자로 처리
  \item 입력 줄의 맨 마지막에 공백 문자(\verb+^^M+) 추가한다.
    (\href{http://www.asciitable.com}{ASCII})
  \item 빈 줄 다음에 \verb+\par+ 토큰 추가
  \item 명령어 다음의 공백은 제거된다.
  \end{itemize}
  
  %\verb*+{\hskip  36 pt}+\\
  \verb*+{\hskip 36 pt}+\\
  \bigskip
  \verb|{|$_1$\quad\fbox{hskip}\quad\verb|3|$_{12}$
  \quad\verb|6|$_{12}$\quad
  \verb*| |$_{10}$\quad\verb|p|$_{11}$\quad\verb|t|$_{11}$\quad\verb|}|$_{2}$
  %\verb|^^M|$_5$
\end{frame}


%
\begin{frame}[fragile]{토큰 리스트}
\begin{verbatim*}
\def\tokentwo{\iftrue this \else that \fi}
\tokentwo
\end{verbatim*}
    \bigskip
    \alert{입력 과정}
    
    \fbox{tokentwo}
    
    \bigskip
    \alert{전개 과정}
    
    \fbox{iftrue}\quad
    \verb|t|$_{12}$\quad
    \verb|h|$_{12}$\quad
    \verb|i|$_{12}$\quad
    \verb|s|$_{12}$\quad
    \verb*| |$_{10}$\quad
    \fbox{else}\quad
    \verb|t|$_{12}$\quad
    \verb|h|$_{12}$\quad
    \verb|a|$_{12}$\quad
    \verb|t|$_{12}$\quad
    \verb*| |$_{10}$\quad
    \fbox{fi}

    \bigskip
    \verb|t|$_{12}$\quad
    \verb|h|$_{12}$\quad
    \verb|i|$_{12}$\quad
    \verb|s|$_{12}$\quad
    \verb*| |$_{10}$\quad
\end{frame}


%
\begin{frame}[fragile]{매크로 vs 원시명령어}
  Plain 포멧에는 900여개의 명령어가 있고, 그 중 300여개가 원시명령어(primitive)
  \begin{itemize}
  \item 매크로는 \alert{전개}의 대상
  \item 원시명령어는 \alert{실행}의 대상
  \item 매크로와 원시명령어의 구분
    \begin{itemize}
    \item 텍북의 찾아보기에 해당 항목에 `\verb+*+'가 붙어있으면 원시명령어
    \item \verb+\show+ 명령어로 확인
    \end{itemize}
  \end{itemize}
\end{frame}


%
\begin{frame}[fragile]{매크로 전개}
  다음의 명령어들은 전개 과정에서 전개된다.
  \begin{itemize}
  \item 매크로
  \item 조건문 (\verb+\if+, \verb+\ifx+, \verb+\ifnum+, \verb+\ifcat+, $\ldots$)
  \item \verb+\number, \romannumeral+
  \item \verb+\string+, \verb+\fontname+, \verb+\jobname+,
    \verb+\meaning+, \verb+\the+
  \item \verb+\csname ... \endcsname+
  \item \verb+\expandafter, \noexpand+
  \item \verb+ ... +
  \end{itemize}
\end{frame}


%
\begin{frame}[fragile]{매크로 전개}
  매크로가 전개되지 않는 경우
  \begin{itemize}
  \item 매크로 정의 시점, \verb+\def+, \verb+\gdef+, \verb+\edef+, \verb+\xdef+
\begin{Verbatim}[fontsize=\small, formatcom=\color{blue}]
\def\sayhello{Hello, world}
\end{Verbatim}
  \item let 할당문, \verb+\let+ 또는 \verb+\futurelet+,
  {\color{blue} \verb+\let\a=\b+}
  \item 파라미터 텍스트 또는 매크로 인자를 읽어 들일 때
\begin{Verbatim}[fontsize=\small, formatcom=\color{blue}]
\def\foo#1\bar{...}
\foo abc \abc xyz \xyz \bar
\end{Verbatim}
  \item \verb+\def+, \verb+\gdef+로 정의된 치환 텍스트을 읽어 들일 때
\begin{Verbatim}[fontsize=\small, formatcom=\color{blue}] 
\def\hello{Hello, \world}
\edef\hello{Hello, \world}} % Error
\end{Verbatim}
  \item \verb+\uppercase+, \verb+\lowercase+ 를 읽어 들일 때
\begin{Verbatim}[fontsize=\small, formatcom=\color{blue}]
\def\say{hello}
\uppercase{\say, abc} => hello, ABC
\uppercase\expandafter{\say, abc} => HELLO, ABC  
\end{Verbatim}
  \end{itemize}
\end{frame}


%
\begin{frame}[fragile]{토큰 리스트}
\begin{verbatim*}
\def\tokentwo{\iftrue this \else that \fi}
\def\tokenone#1{...}
\expandafter\tokenone\tokentwo
\end{verbatim*}
    \bigskip

    \alert{입력 과정}
    
    \fbox{expandafter}\quad
    \fbox{tokenone}\quad
    \fbox{tokentwo}
    
    \bigskip
    \alert{전개 과정}

    \fbox{tokenone}\quad
    \verb|t|$_{12}$\quad
    \verb|h|$_{12}$\quad
    \verb|i|$_{12}$\quad
    \verb|s|$_{12}$\quad
    \verb*| |$_{10}$\quad {\color{red} (X)}

    \bigskip
    \fbox{tokenone}\quad
    \fbox{iftrue}\quad
    \verb|t|$_{12}$\quad
    \verb|h|$_{12}$\quad
    \verb|i|$_{12}$\quad
    \verb|s|$_{12}$\quad
    \verb*| |$_{10}$\quad
    \fbox{else}\quad
    \verb|t|$_{12}$\quad
    \verb|h|$_{12}$\quad
    \verb|a|$_{12}$\quad
    \verb|t|$_{12}$\quad
    \verb*| |$_{10}$\quad
    \fbox{fi}
\end{frame}


%
\begin{frame}[fragile]{그룹 만들기}
  \textbf{\alert{그룹}}
  \begin{itemize}
  \item \verb+{+, \verb+}+
  \item \verb+\bgroup+, \verb+\egroup+
    {\color{blue}\small \verb+\let\bgroup={ \let\egroup=}+}
  \item \verb+\begingroup+, \verb+\endgroup+ (원시명령어, primitive)
  \end{itemize}
  \alert{사용례}
  \begin{itemize}
  \item \verb+\bgroup \bf Hello}+
  \item \verb+{\bf Hello \egroup+
  \item \verb+\begingroup \bf Hello}+ {\color{red} (X)}
  \end{itemize}
\begin{Verbatim}[fontsize=\small, formatcom=\color{blue}]
    \def\hello{Hello}
    { \baselineskip=14pt \def\hello{Hola} \hello }
    \hello
    Hola Hello
\end{Verbatim}
\end{frame}


%
\begin{frame}[standout]
  예제: \texttt{\string\bold} 매크로
\end{frame}


%
\begin{frame}[fragile]{\texttt{\string\bold} 매크로}
\begin{Verbatim}[fontsize=\small, formatcom=\color{blue}]
{\bf Hello world}
    
\bold{Hello world}
\end{Verbatim}
  \begin{alertblock}{Programmer}
    \verb+\def\bold#1{{\bf #1}}+
  \end{alertblock}
\end{frame}


%
\begin{frame}[fragile]{\texttt{\string\bold} 매크로}
\begin{Verbatim}[fontsize=\small, formatcom=\color{blue}]
\bold{
  Hello

  world
}

Runaway argument?
{ Hello
! Paragraph ended before \bold was complete.
\end{Verbatim}
  \begin{alertblock}{Programmer first class}
    \verb+\long\def\bold#1{{\bf #1}}+
  \end{alertblock}
\end{frame}


%
\begin{frame}[fragile]{\texttt{\string\bold} 매크로}
  \begin{alertblock}{Hacker}
    \verb+\def\beginbold{\bgroup\bf}+
    
    \verb+\def\endbold{\egroup}+
  \end{alertblock}

\begin{Verbatim}[fontsize=\small, formatcom=\color{blue}]
\beginbold
Hello

world
\endbold
\end{Verbatim}
\end{frame}


%
\begin{frame}[fragile]{\texttt{\string\bold} 매크로}
  \begin{alertblock}{Wizard}
    \verb+\def\bold{\bgroup\bf\let\next=}+
  \end{alertblock}

\begin{Verbatim}[fontsize=\small, formatcom=\color{blue}]
\bold{text}
\bgroup\bf\let\next={text}
\end{Verbatim}

%  \hbox to\hsize{\hss
%    \verb+\let+<control sequence><equals>%
%    <one optional space><token>\hss}
%  <equals>$\rightarrow$<optional spaces> | <optional spaces>=

\bigskip

  \fbox{bgroup}\quad\fbox{bf}\quad
  \verb|t|$_{11}$\quad\verb|e|$_{11}$\quad
  \verb|x|$_{11}$\quad\verb|t|$_{11}$\quad
  \verb|}|$_{2}$

  \verb|{|$_1$\quad\fbox{bf}\quad
  \verb|t|$_{11}$\quad\verb|e|$_{11}$\quad
  \verb|x|$_{11}$\quad\verb|t|$_{11}$\quad
  \verb|}|$_{2}$
\end{frame}


%
\begin{frame}[standout]
  예제: 스크립트 매크로
\end{frame}


%
\begin{frame}[fragile]{스크립트 매크로}
\begin{Verbatim}[fontsize=\small]
\beginscript
Now, at last, you can easily typeset
conversations you eavesdrop on in
restaurants and on planes.
  
Really? That's just what I've been waiting
for! How do I do it?
  
Exactly the way this script was done.
  
Is it easy?
  
Extremely.
\endscript
\end{Verbatim}
\end{frame}


%
\newcount\spk
\def\beginscript{\bgroup \parindent=0pt \color{red} \spk=1 \rightskip.4in
  \def\par{\ifnum\spk=1 \endgraf \color{blue} \spk=2 \leftskip.4in
    \rightskip0in
    \else \endgraf \color{red} \spk=1 \leftskip0in \rightskip.4in \fi}}
\def\endscript {\egroup}

\begin{frame}[fragile]{스크립트 매크로}
  \hsize 3in
  \beginscript
  Now, at last, you can easily typeset
  conversations you eavesdrop on in
  restaurants and on planes.
  
  Really? That's just what I've been waiting
  for! How do I do it?
  
  Exactly the way this script was done.
  
  Is it easy?
  
  Extremely.
  \endscript
\end{frame}


%
\begin{frame}[fragile]{스크립트 매크로}
\begin{Verbatim}[fontsize=\small]
\let\endgraf=\par %plain.tex 
\newcount\spk
\def\beginscript{\bgroup \parindent=0pt \color{red}
  \spk=1 \rightskip.4in
  \def\par{\ifnum\spk=1 \endgraf \color{blue} \spk=2
             \leftskip.4in \rightskip0in
           \else \endgraf \color{red} \spk=1
             \leftskip0in \rightskip.4in \fi}}
\def\endscript{\egroup}
\end{Verbatim}
\end{frame}


%
\begin{frame}[standout]
  예제: FIFO 매크로
\end{frame}


%
\begin{frame}[fragile]{조건문}
  \begin{itemize}
  \item \verb+\if <token1> <token2>+
    \begin{itemize}
    \item 두 토큰의 문자 코드(character code)가 같은지 검사한다.
    \item 검사 전에, \verb+\if+ 뒤에 나오는 토큰을 두 개의 토큰이 나올 때까지 전개한다.
\begin{Verbatim}[fontsize=\small, formatcom=\color{blue}]
\def\foo{abc} \def\bar{abc}
\if\foo\bar true \else false \fi  % => false
\end{Verbatim}
    \end{itemize}
  \item \verb+\ifx <token1> <token2>+
    \begin{itemize}
    \item 두 개의 토큰이 서로 일치하는지 검사한다.
    \item \verb+\if+와 달리 \verb+\ifx+ 뒤에 나오는 토큰을 전개하지 않는다.
\begin{Verbatim}[fontsize=\small, formatcom=\color{blue}]
\def\foo{abc} \def\bar{abc}
\ifx\foo\bar true \else false \fi  % => true
\end{Verbatim}
    \end{itemize}
  \end{itemize}
\end{frame}


%
\begin{frame}[fragile]{FIFO 매크로}
\begin{Verbatim}[formatcom=\color{blue}]
\def\fifo#1{\ifx\ofif#1\ofif\fi
  \process#1\fifo}
\def\ofif#1\fifo{\fi}
\end{Verbatim}
\begin{Verbatim}
\fifo abc\ofif
=> \process a \process b \process c
\end{Verbatim}
\end{frame}


%
\def\fifo#1{\ifx\ofif#1\ofif\fi
  \process#1\fifo}
\def\ofif#1\fifo{\fi}
\def\process#1{\fbox{#1}\,}

\begin{frame}[fragile]{FIFO 매크로}
  \begin{Verbatim}[fontsize=\small, formatcom=\color{blue}]
\def\fifo#1{\ifx\ofif#1\ofif\fi
  \process#1\fifo}
\def\ofif#1\fifo{\fi}
  \end{Verbatim}
  \begin{Verbatim}[fontsize=\small]
\def\getlength#1{
  \count255=0
  \def\process##1{\advance\count255 by 1 }
  \fifo #1\ofif \number\count255}
\getlength{argument} % => 8 

\zipcode#1{%
  \def\process##1{\fbox{##1}\,}
  \fifo #1\ofif}
\end{Verbatim}

\verb+\zipcode{17077}  % =>+ \quad \fifo 17077\ofif
\end{frame}


%
\begin{frame}{참고 문서}
  \begin{itemize}
  \item \href{http://ftp.ktug.org/tex-archive/systems/knuth/dist/tex/}
    {The \TeX book}
  \item \href{http://ftp.ktug.org/tex-archive/info/impatient/book.pdf}
    {\TeX\ for the Impatient}
  \item \href{http://ftp.ktug.org/tex-archive/info/texbytopic/TeXbyTopic.pdf}
    {\TeX\ By Topic}
  \item \href{http://pgfplots.sourceforge.net/TeX-programming-notes.pdf}
    {Notes On Programming in TeX}
  \item \href{https://www.tug.org/TUGboat/tb08-3/tb19knut.pdf}
    {Macros for Jill}
  \item \href{https://www.tug.org/TUGboat/tb14-1/tb38laan.pdf}
    {FIFO and LIFO sing the BLUes}
  \item \href{https://www.tug.org/TUGboat/tb15-1/tb42arseneau.pdf}
    {The TeX\ Hierarchy}
  \end{itemize}
\end{frame}


%
\begin{frame}[standout]
  ¿Tienes alguna pregunta?
\end{frame}

\end{document}


%
\begin{frame}[fragile]{매크로 사용자 단계}
  The \TeX\ Hierarchy, TUGboat, Volume 15 (1994), No.1, 7---9.
  \begin{description}
  \item [Novice] has heard of macros, but has never seen one.
  \item [User] writes macros that are used once, and that are
    longer than the code they replace.
  \item [Programmer], having been bitten by unwanted spaces,
    writes macros that don't contain spaces, and every line ends with
    a `{\small\verb+%+}'.
  \item [Hacker] has written self-modifying macros, writes
    {\small\verb+\endlinechar=-1+} or {\small\verb+\catcode'\^^M=9+}
    to prevent having to put {\small\verb+%+}'s at the end of lines in macros.
  \item [Guru] has written macros containing {\small\verb+####+}, more than 3
    {\small\verb+\expandafter+}'s in a row, and the sequence
    {\small\verb+\expandafter\endcsname+}.
  \end{description}
\end{frame}


