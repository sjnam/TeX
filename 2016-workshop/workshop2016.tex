% 2016 공주대 워크숍
% 텍 매크로 작성법

\documentclass{beamer}

\usetheme{metropolis}
%\metroset{inner/sectionpage=none}
\metroset{outer/progressbar=head}
\metroset{color/block=fill}

\usepackage{kotex}

\hypersetup{pdfencoding=auto}


% title
\title{텍 매크로 작성법}
\subtitle{텍 매크로 작성의 기초와 고급}
\author{남수진}
\date{2016년 11월 5일(토)}
\institute{
  2016 공주대학교 문서작성 워크숍 2016\\
  공주대학교 인문사회과학관 1층 컴퓨터실 107호}


%%
\begin{document}

\maketitle


%
\begin{frame}{매크로 관련 명령어}
  \vspace{4mm}
  \hbox to\hsize{\hss\includegraphics[height=7.5cm]{cs.png}\hss}
\end{frame}


%
\begin{frame}[fragile]{텍의 처리 과정}
  \begin{description}
  \item [입력(Input)] 파일을 줄(line) 단위로 읽어서 \alert{토큰 리스트}를 만든다.
  \item [전개(Expansion)] 위의 토큰 리스트를 입력으로 받어서 전개할 수 있는
    모든 토큰을 전개해서 더이상 전개 할 수 없는 토큰들로 구성된 새로운 토큰 리스트를 만든다.
  \item [실행(Execution)]
  \item [출력(Visual)]
  \end{description}

  \begin{exampleblock}{토큰 리스트}
  \verb*+{\hskip 36 pt}+\par
  \smallskip
  \includegraphics[width=10cm]{tokens.jpg}
  \end{exampleblock}
\end{frame}


%
\begin{frame}[fragile]{매크로 정의}
  \medskip
  \hbox to\hsize{\hss
    \verb+\def+<control sequence>\alert{<parameter text>}%
    \verb+{+\alert{<replacement text>}\verb+}+\hss}
  \smallskip
  \begin{itemize}
  \item 문서에서 여러번 반복적으로 사용되는 문구나 명령어의 나열을 하나의
    명령어(control sequence)로 만든 것
  \item \verb+\def+, \verb+\gdef+, \verb+\edef+, \verb+\xdef+
  \item 텍의 전개 과정에서 매크로는 치환문으로 교체된다.
  \item 정의 시점에는 전개하지 않는다.
  \item 파라미터 텍스트와 치환 텍스트에는 정의되지 않은 매크로 사용이 가능하다.
    
    \verb+\def\control#1\sequence{...}+
  \item 매크로를 전개하기 위해서 인자를 읽어 들일때는 전개하지 않는다.
  \end{itemize}
\end{frame}


%
\begin{frame}[fragile]{그룹 만들기}
  \begin{itemize}
  \item \verb+{+, \verb+}+
  \item \verb+\bgroup+, \verb+\egroup+
    \begin{exampleblock}{사용예}
      \begin{itemize}
      \item \verb+\let\bgroup={ \let\egroup=}+
      \item \verb+\bgroup \bf Hello}+
      \item \verb+{\bf Hello \egroup+
      \end{itemize}
    \end{exampleblock}
  \item \verb+\begingroup+, \verb+\endgroup+ (원시명령어, primitive)
  \end{itemize}

  {\small
  \begin{verbatim}
    \def\hello{Hello}
    { \baselineskip=14pt \def\hello{Hola} \hello }
    \hello
  \end{verbatim}}
\end{frame}


%
%\begin{frame}{골치 아픈 공백 문자(space)}
%  \begin{itemize}
%  \item <return>은 공백 문자이다.
%  \item 연속된 여러 개의 공백 문자는 하나의 공백 문자로 바뀐다.
%  \item 입력 과정(input process)에서 명령어 다음의 공백은 제거된다.
%  \item 매크로를 정의할 때, <parameter 텍스트>와 <replacement 텍스트>에서의
%    공백은 중요하다.
%  \end{itemize}
%\end{frame}


%
\plain{\huge 예제}


%
\begin{frame}[fragile]{\texttt{\string\bold} 매크로}
  \begin{verbatim}
    {\bf Hello world}
    
    \bold{Hello world}
  \end{verbatim}
  \begin{exampleblock}{Programmer}
    \verb+\def\bold#1{{\bf #1}}+
  \end{exampleblock}
\end{frame}


%
\begin{frame}[fragile]{\texttt{\string\bold} 매크로}
  \begin{verbatim}
    \bold{
      Hello

      world
    }

    Runaway argument?
    { Hello
    ! Paragraph ended before \bold was complete.
  \end{verbatim}
  \begin{exampleblock}{Programmer first class}
    \verb+\long\def\bold#1{{\bf #1}}+
  \end{exampleblock}
\end{frame}


%
\begin{frame}[fragile]{\texttt{\string\bold} 매크로}
  \begin{exampleblock}{Hacker}
    \verb+\def\beginbold{\bgroup\bf}+
    
    \verb+\def\endbold{\egroup}+
  \end{exampleblock}

  \begin{verbatim}
    \beginbold
    Hello

    world
    \endbold
  \end{verbatim}
\end{frame}


%
\begin{frame}[fragile]{\texttt{\string\bold} 매크로}
  \begin{exampleblock}{Wizard}
    \verb+\def\bold{\bgroup\bf\let\next=}+
  \end{exampleblock}
  \medskip
  <equals>$\rightarrow$<optional spaces> | <optional spaces>=
  
  \hbox to\hsize{\hss
    \verb+\let+<control sequence><equals>%
    <one optional space><token>\hss}
  \medskip

  \begin{exampleblock}{사용예}
    \begin{itemize}
    \item \verb*+\let\a= \b+
    \item \verb*+\let\a\b+
    \item \verb*+\let\a= Hello+
    \end{itemize}
  \end{exampleblock}
\end{frame}


%
%\begin{frame}[fragile]{\texttt{\string\bold} 매크로}
%  \begin{alertblock}{Guru}
%    \verb+\def\bold#{\bgroup\bf\let\next= }+
%  \end{alertblock}
%\end{frame}


%
\begin{frame}[fragile]{조건문 검사}
  \begin{exampleblock}{\string\if\ <token1> <token2>}
    \begin{itemize}
    \item <token1>과 <token2>의 문자 코드가 같은지를 검사한다. 명령어의 문자 코드는 256.
    \item <token1>과 <token2>의 문자 코드가 같은지를 검사한다. 명령어의 문자 코드는 256.
    \end{itemize}
  \end{exampleblock}
  \begin{exampleblock}{\string\ifx\ <token1> <token2>}
    \begin{itemize}
    \item <token1>과 <token2>의 문자 코드가 같은지를 검사한다. 명령어의 문자 코드는 256.
    \item <token1>과 <token2>의 문자 코드가 같은지를 검사한다. 명령어의 문자 코드는 256.
    \end{itemize}
  \end{exampleblock}
\end{frame}


%
\newcount\spk
\def\beginscript{\bgroup \parindent=0pt \rm \spk=1 \rightskip.4in
  \def\par{\ifnum\spk=1 \endgraf \sl \spk=2 \leftskip.4in \rightskip0in
    \else \endgraf \rm \spk=1 \leftskip0in \rightskip.4in \fi}}
\def\endscript {\egroup}

\begin{frame}[fragile]{스크립트 매크로}
  \hsize 3in
  \beginscript
  Now, at last, you can easily typeset
  conversations you eavesdrop on in
  restaurants and on planes.
  
  Really? That's just what I've been waiting
  for! How do I do it?
  
  Exactly the way this script was done.
  
  Is it easy?
  
  Extremely.
  \endscript
\end{frame}


%
\begin{frame}[fragile]{스크립트 매크로}
  \begin{verbatim}
  \beginscript
  Now, at last, you can easily typeset
  conversations you eavesdrop on in
  restaurants and on planes.
  
  Really? That's just what I've been waiting
  for! How do I do it?
  
  Exactly the way this script was done.
  
  Is it easy?
  
  Extremely.
  \endscript
  \end{verbatim}
\end{frame}


%
\begin{frame}[fragile]{스크립트 매크로}
  \begin{verbatim}
  \let\endgraf=\par
  \newcount\spk
  \def\beginscript{\bgroup \parindent=0pt \rm
    \spk=1 \rightskip.4in
    \def\par{\ifnum\spk=1 \endgraf \sl \spk=2
               \leftskip.4in \rightskip0in
             \else \endgraf \rm \spk=1
               \leftskip0in \rightskip.4in \fi}}
  \def\endscript{\egroup}
  \end{verbatim}
\end{frame}


%
\begin{frame}[fragile]{FIFO 매크로}
  \begin{verbatim}
    \def\fifo#1{\ifx\ofif#1\ofif\fi
      \process#1\fifo}
    \def\ofif#1\fifo{\fi}

    \fifo abc\ofif
    => \process a \process b \process c
  \end{verbatim}
\end{frame}


%
\begin{frame}[fragile]{FIFO 매크로}
  \begin{verbatim}
    \def\fifo#1{\ifx\ofif#1\ofif\fi
      \process#1\fifo}
    \def\ofif#1\fifo{\fi}

    \newcount\length
    \def\process#1{\advance\length 1}

    \fifo aapnoon\ofif \number\length % => 7
  \end{verbatim}
\end{frame}


%
\begin{frame}{참고문서}
  \begin{itemize}
  %\item \url{https://github.com/sjnam/TeX/2016-workshop}
  \item \href{http://ftp.ktug.org/tex-archive/systems/knuth/dist/tex/}
    {The \TeX book}
  \item \href{http://ftp.ktug.org/tex-archive/info/impatient/book.pdf}
    {\TeX\ for the Impatient}
  \item \href{http://ftp.ktug.org/tex-archive/info/texbytopic/TeXbyTopic.pdf}
    {\TeX\ By Topic}
  \item \href{http://pgfplots.sourceforge.net/TeX-programming-notes.pdf}
    {Notes On Programming in TeX}
  \item \href{https://www.tug.org/TUGboat/tb14-1/tb38laan.pdf}
    {FIFO and LIFO sing the BLUes}
  \item \href{https://www.tug.org/TUGboat/tb08-3/tb19knut.pdf}
    {Macros for Jill}
  \item \href{https://www.tug.org/TUGboat/tb15-1/tb42arseneau.pdf}
    {The TeX\ Hierarchy}
  \end{itemize}
\end{frame}


%
\plain{\huge ¿Tienes alguna pregunta?}


\end{document}


%
\begin{frame}[fragile]{매크로 사용자 단계}
  The \TeX\ Hierarchy, TUGboat, Volume 15 (1994), No.1, 7---9.
  \begin{description}
  \item [Novice] has heard of macros, but has never seen one.
  \item [User] writes macros that are used once, and that are
    longer than the code they replace.
  \item [Programmer], having been bitten by unwanted spaces,
    writes macros that don't contain spaces, and every line ends with
    a `{\small\verb+%+}'.
  \item [Hacker] has written self-modifying macros, writes
    {\small\verb+\endlinechar=-1+} or {\small\verb+\catcode'\^^M=9+}
    to prevent having to put {\small\verb+%+}'s at the end of lines in macros.
  \item [Guru] has written macros containing {\small\verb+####+}, more than 3
    {\small\verb+\expandafter+}'s in a row, and the sequence
    {\small\verb+\expandafter\endcsname+}.
  \end{description}
\end{frame}


