% 2017 KTS
% 텍 매크로 파일 만들기

\documentclass{beamer}

\usefonttheme[onlymath]{serif}
\usetheme{metropolis}
\metroset{outer/progressbar=head}

\usepackage{fancyvrb}
\usepackage{kotex}
\hypersetup{pdfencoding=auto}

\def\TEX/{$\textrm{\TeX}$}
\def\LATEX/{$\textrm{\LaTeX}$}

% title
\title{텍 매크로 파일 만들기}
\subtitle{플레인텍으로 만드는 레시피 카드}
\date{2017년 2월 11일 토요일}
\author{남수진}
\institute{
  2017 한국텍학회 학술대회 및 정기총회 \\
  동국대학교 법학관 B253호}
\titlegraphic{\hfill\includegraphics[height=3cm]{meta.pdf}}


%%
\begin{document}

\maketitle

%
\begin{frame}[standout]
  왜 플레인 텍인가?
\end{frame}


%
\begin{frame}[fragile]{플레인 텍 (Plain \TEX/)}
  \begin{itemize}
  \item 도널드 크누스가 만든 텍, \alert{순수 텍(virgin \TEX/)}
    \begin{itemize}
    \item 매크로가 하나도 없는 갓난 아이 같은 텍. 순수 그 자체
    \item 문서 작성에 사용하기에는 무리가 있다.
    \end{itemize}
  \item 순수 텍을 실제로 사용할 수 있도록 기본적인 세팅과 편리한 매크로들을 제공한다.
    하지만 이것도 사용하다 보면 매크로가 많이 부족하다고 느낌.
  \item ``포멧은 이렇게 만드는 것이다.''라는 표본 제시
  \end{itemize}
\end{frame}


%
\begin{frame}[fragile]{라텍 (\LATEX/)}
  \begin{itemize}
  \item 우리가 흔히 텍이라고 부르는 것.
  \item 고품질의 조판을 요하는 문서 작성 시스템
  \item 어느 정도 분량이 있는 구조를 갖춘 문서 작성에 제격이다.
  \item 클래스 파일, 다수의 패키지 파일과 복잡한(?) 글꼴 세팅이 필요하다.
  \end{itemize}
  
  \begin{Verbatim}[fontsize=\small]
    \documentclass[a4paper,tocentry,microtype]{oblivoir}
    \usepackage[dbl4x6]{fapapersize}
    \usepackage[dvipdfmx]{graphicx}
    \usepackage{hyperref}
    \usepackage{wrapfig}
    \usepackage{caption}
    ...
  \end{Verbatim}
\end{frame}


%
\begin{frame}{1995년 TUG 미팅에서}
  Questions and Answers with Prof. Donald E. Knuth, 
  TUGboat \textbf{17} (1996), 7--22
  \begin{description}
  \item[Silvio Levy:] How come you don't use \LATEX/? [\textsl{laughter}]
  \item[Don:] How come I don't use \LATEX/? [\textsl{laughter}]
    I'm scared of \alert{large systems!} [\textsl{louder laughter\/}]
    Bart?
  \end{description}
\end{frame}


%
\begin{frame}{플레인 텍을 사용하면}
  \begin{itemize}
  \item 매년 새롭게 나오는 텍라이브를 설치할 필요가 없다.
  \item 심심할 때마다 {\small\alert{\texttt{tlmgr update --all --self}}}를
    칠 필요가 없다.
  \item 20년 전에 만들었던 텍 문서가 여전히 잘 컴파일 된다.
  \item 프리엠블로 고민할 필요가 없다. 폰트에 대한 욕심을 줄이면, 할 일이 아무것도 없다.
  \item 한글 문서라면, {\small\alert{\texttt{\string\input\ kotex.sty}}}
    한줄로 끝.
  \item {\scriptsize 자신만의 간단한 매크로를 만들어야 할 때가 있다.}
  \end{itemize}
  
  구조적이지 않고 일정한 틀이 없는 간단한 문서 작성에는 플레인 텍을 사용해 보자.
\end{frame}


%
\begin{frame}[standout]
  레시피 카드 매크로
\end{frame}


%
\begin{frame}[fragile]{Active 문자}
  \begin{itemize}
  \item 백슬래시 없이도 명령어 역할을 하는 문자
  \item 플레인 텍에서는 \alert{\string~}
  \item 주로 \texttt{\string\let}을 이용하여 명령어로 동작한다.
  \end{itemize}
  \begin{Verbatim}
    \catcode`\*=13 \let*=\medskip
  \end{Verbatim}
\end{frame}


%
\begin{frame}[fragile]{\texttt{\string\obeylines}}
  \begin{itemize}
  \item 텍은 엔터(CR)도 간격을 나타내는 문자로 인식한다. (\verb+^^M+)% ascii 참고
  \item 엔터를 우리가 기대하는 기능 그대로! 
  \end{itemize}

  \begin{table}[ht]
  \begin{minipage}[t]{.4\textwidth}
  \begin{Verbatim}[fontsize=\small]
{\obeylines
Roses are red,
\quad Violets are blue;
Rhymes can be typeset
\quad With boxes and glue.
\smallskip}
  \end{Verbatim}
  \end{minipage}%
  \qquad\qquad
  \begin{minipage}[t]{.4\linewidth}
    {\obeylines
      Roses are red,
      \quad Violets are blue;
      Rhymes can be typeset
      \quad With boxes and glue.
      \smallskip}
  \end{minipage}
  \end{table}
  
  \begin{Verbatim}
   \def\obeylines{\catcode`\^^M=13 \let^^M=\par}
  \end{Verbatim}
\end{frame}


%
\begin{frame}{참고 문서}
  \begin{itemize}
  \item \href{http://ftp.ktug.org/tex-archive/systems/knuth/dist/tex/}
    {The \TeX book}
  \item \href{https://tug.org/TUGboat/tb17-1/tb50knut.pdf}
    {TUG95, Questions and Answers with Prof. Donald E. Knuth}
  \item \href{https://www.tug.org/TUGboat/tb08-3/tb19knut.pdf}
    {Macros for Jill}
  \end{itemize}
\end{frame}


%
\begin{frame}[standout]
  감사합니다
\end{frame}

\end{document}

