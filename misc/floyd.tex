\magnification\magstep1
\hsize=.8\hsize

The first problem is one that Bob Floyd posed to the first-year students in
our Ph.D. program at Stanford during the fall of 1972:
\smallskip
{\narrower\noindent\llap{``}%
The numbers $\sqrt1$, $\sqrt2$, $\ldots$, $\sqrt{50}$ are to be partitioned
into two parts whose sum is nearly equal; find the best such partition you
can, using less than 10 seconds of computer time.''\bigskip}

\noindent{\sl[Readers should now work on Floyd's problem, and wait one month,
before reading on.]\bigskip}

\leftline{\bf An Approach to Floyd's Problem}
\bigskip
\noindent Last month I discussed some of the virtues of Floyd's problem, which
is to divide the numbers
$$\hbox{$\sqrt1$,\ $\sqrt2$,\ $\ldots,$\ $\sqrt{50}$}$$
into two parts whose
sums are as equal as you can make them after ten seconds of computer time. Here
is the way I decided to approach the subject; perhaps some reader will have a
much better idea.

Since $(\sqrt1+\sqrt2+\cdots+\sqrt{50})/2$ is equal to
$$119.51790\,03017\,60392\,24702\,02231\ldots,$$ we seek the subset of
$\{\sqrt1,\sqrt2,\ldots,\sqrt{50}\}$ whose sum least exceeds this value.

In the first place it is helpful to estimate the kind of results that might be
expected. Most subsets of the given set have about 25 elements, and in fact the
number of subsets with exactly 25 elements is $50!/25!\,25!$; this is about
$$2^{50}\over\sqrt{25\pi},$$ according to Stirling's approximation. [The exact
number is 126,410,\allowbreak606,437,752, compared to
$2^{50}=1{,}125{,}899{,}906{,}842{,}624.$]~ All of these subsets have
a sum that lies between $\sqrt1+\sqrt2+\cdots+\sqrt{25}\approx85.6$ and
$\sqrt{26}+\sqrt{27}+\cdots+\sqrt{50}\approx153.4,$ and they tend to cluster
about the average value 119.5. So we have more than $10^{14}$ numbers packed
into an interval of length less than 70; therefore $70/10^{14}$ appears to be
a conservative estimate for the amount by which the best partitioning exceeds
$(\sqrt1+\sqrt2+\cdots+\sqrt{50})/2.$

But not all of these subsets will give different sums. In the first place, the
seven numbers $$\hbox{$\sqrt1$, $\sqrt4$, $\ldots,$ $\sqrt{49}$}\quad=\quad
\hbox{1, 2, $\ldots,$ 7}$$ will
yield only integer values (and it is easy to verify that each integer from 0 to
28 can be represented.) Our problem therefore reduces to finding a subset of
the 43 numbers $$\hbox{$\{\sqrt2$, $\sqrt3$, $\sqrt5$, $\sqrt6$, $\ldots,$
$\sqrt{48}$, $\sqrt{50}\,\}$}$$
whose sum have a {\it fraction part\/} least exceeding
$$0.51790\,03017\,60392\,24702.$$
Unless the integer part of the sum is unusually large or small, we will be able
to adjust it by adding a suitable subset of $\{1,2,\ldots,7\},$ bringing the
integer part up to 119.

We now have $2^{43}$ possibilities to try; but {\it they\/} aren't all distinct,
either. The values $$\{\hbox{$\sqrt2$, $\sqrt8$, $\sqrt{18}$, $\sqrt{32}$,
$\sqrt{50}$}\,\}\ =\ \{\hbox{$\sqrt2$, $2\sqrt2$, $3\sqrt2$, $4\sqrt2$,
$5\sqrt2$}\,\}$$ leads to subset whose sum
is restricted to sixteen values $$\hbox{0, $\sqrt2$, $2\sqrt2$, $\ldots,$
$15\sqrt2$};$$ so only 32
subsets are different. Similiarly, there are only eleven essentially different
subset of $$\{\hbox{$\sqrt3$, $\sqrt{12}$, $\sqrt{27}$, $\sqrt{48}$}\,\}$$
to try. By means of these observation out program can do in 10 seconds what it
would take $10\cdot32\cdot16/(16\cdot11)\approx29$ seconds to do otherwise.

But there are still more than $2^{41}>10^{12}$ possibilities remaining, and this
is far too large; if it takes 100 microseconds for me to test one subset, I
have time to test only $10^5$ subsets.

One way to gain speed is to divide
$$\{\hbox{$\sqrt2$, $\sqrt3$, $\sqrt5$, $\sqrt6$, $\ldots,$ $\sqrt{48}$,
$\sqrt{50}$}\,\}$$
into two parts $A$ and $B,$ and to form a table containing all sums of the
subsets of $A.$ Then for each subset of $B$ we can look in the table to see if
there is an entry with the appropriate leading bits.

That is what I eventually did. Let
$$
\displaylines{\quad
  A=\{\hbox{$\sqrt2$, $\sqrt8$, $\sqrt{18}$, $\sqrt{32}$, $\sqrt{50}$, $\sqrt3$,
  $\sqrt{12}$, $\sqrt{27}$, $\sqrt{48}$,}\hfill\cr
  \hfill{}\hbox{$\sqrt5$, $\sqrt6$, $\sqrt7$, $\sqrt{10}$, $\sqrt{11}$,
  $\sqrt{13}$, $\sqrt{14}$, $\sqrt{15}$}\,\},\quad\cr}
$$ %$
so that the subsets of $A$ take on ${11\over16}\cdot2^{16}$ different values. I
stored the 32 leading bits of these fraction parts into a hash table of size
$2^{16},$ and in another table of size $2^{16},$ I stored a bit pattern
identifying the corresponding subset. (The hash code was the low order 16 bits
of each 32-bit key.) Let $B=\{\sqrt{17},\sqrt{19},\sqrt{20},\ldots,\sqrt{47}\}$
be the remaining set of 26 elements; in the machine I used only the 32 leading
bits of the fraction parts of each number $\sqrt k.$ Since my hash table
contained ${11\over16}\cdot2^{16}$ more-or-less random 32-bits numbers, I
expected to get a match once out of every $2^{32}/({11\over16}\cdot2^{16})=
{11\over16}\cdot2^{16}$ times I looked up another 32-bit value. Therefore I
wanted to have a program that generate $2^{17}$ or so subsets $S$ of $B,$ looking
up the 32 bits corresponding to $(.51700\cdots-\sum S)\bmod1$ in the hash table.
I figured that such a sampling would probably give partition. But I would have
only about 75 to 80 microseconds to spend per subset, so I needed a fast way to
probe the hash table and to generate the subset sums.

The solution was to use {\it ordered\/} hash tables with linear probing; this is
a variant of hashing Ole Amble and I had discovered a few years ago. Such hash
tables are especially well suited to cases when searches are unsuccessful,
requiring only 2.1 probes per search in this case. For the subset generating,
I used Gray code, since this meant that only one subset elements changed state
each time and the subset sum was therefore easy to update. It took about 1.5
seconds to build the hash table. I started the subset generation of $B$ in a
random part of its cycle, since I knew that I would be able to look at only a
small fraction of it subsets.

The best of the three results I got before 10 seconds expired was actually
typed out after only 6 seconds, namely,
$$
\displaylines{\quad
\sqrt2+\sqrt3+\sqrt4+\sqrt5+\sqrt8+\sqrt{12}+\sqrt{17}+\sqrt{18}\hfill\cr
\qquad+\sqrt{19}+\sqrt{22}+\sqrt{27}+\sqrt{28}
+\sqrt{29}+\sqrt{33}+\sqrt{34}+\sqrt{35}\hfill\cr
\qquad+\sqrt{37}+\sqrt{38}+\sqrt{40}+\sqrt{42}
+\sqrt{45}+\sqrt{46}+\sqrt{49}+\sqrt{50}\hfill\cr
\hfill{}\approx119.51790\,03021\,65123\,39726\,54768.\quad\cr}
$$
This sum differs from its complementary sum by approximately $8\cdot10^{-10}.$
(This optimum partition might well be a thousand times better, as mentioned
above; my program could have found it if 10 thousand seconds were available
instead of ten!)
\bigskip
\bigskip
\leftline{\bf Addendum}
\bigskip
\noindent While preparing this book for publication, I decided to try Floyd's
problem on the computer I now have at home, a vintage 1992 workstation
(SPARC~2), in spite of the fact that I claim to be uninterested in the answer.
Using a slight improvement of the method sketched avobe, I found the exact
solution after 26 seconds of computation:
$$
\displaylines{\quad
\sqrt2+\sqrt3+\sqrt6+\sqrt{10}+\sqrt{11}+\sqrt{14}+\sqrt{17}
+\sqrt{18}+\sqrt{19}\hfill\cr
\qquad+\sqrt{21}+\sqrt{22}+\sqrt{23}+\sqrt{24}
+\sqrt{26}+\sqrt{28}+\sqrt{29}+\sqrt{32}\hfill\cr
\qquad+\sqrt{34}+\sqrt{35}+\sqrt{36}+\sqrt{39}
+\sqrt{40}+\sqrt{42}+\sqrt{45}+\sqrt{50}\hfill\cr
\hfill{}\approx119.51790\,03017\,60463\,73981.\quad\cr}
$$
This sum differs from its complement by approximately $1.4\times10^{-13},$ so it
is more than 5000 times better than 10-second solution that I found in 1976.
The improved procedure avoids storing the two files of $\approx2^{21}$ numbers
in memory; instead, one can use $x_i$ and $y_i$ ``on the fly'' as they are
generated, since the top-level algorithm merely needs to increase $i$ by 1 or
decrease $j$ by 1 as it examines each $x_i$ and $y_i$ only once, and since the
files can be generated in sorted order by a multiway merge of shifts of shorter
files using the ``tree of losers'' technique.

\bye
