% More Programming Pearls
% Confessions of a Coder
% Jon Bentley
%
% Part I: Algorithms
% Column 3: Confessions of a Coder

\input mppearls

\pageno=27
\columntitle 3:confessions of a coder

\noindent
Most programmers spend a lot of time testing and debugging, but those activities
don't often get much attention in writing. This column describes how I tested
and debugged a few hard subroutines, with an emphasis of the {\it scaffolding\/}
I used in the process. The scaffolding around a building provides access to
components that workers couldn't otherwise reach. Software scaffolding consiste
of temporary programs and data that give programmers access to system
components. The scaffolding isn't delivered to the customer, but it is indispensible during testing and debugging.

Enough background, and on to two painful stories.

{\medskip\more
{\it Confession 1\/}. Several years ago I needed a selection routine in the
middle of a 500-line program. Because I knew the program was hard, I copied
a 20-line subroutine from an excellent algorithm text. The program usally
ran correctly, but failed every now and then. After two days I debugging, I
tracked the bug down to the selection routine. During most of the debugging,
that routine was above supicion: I was convinced by the book's informal proof
of the routine's correctness, and I had checked my code several times to make
sure it matched the book. Unfortunately, a ``$<$'' in the book should have been
a ``$\le$''. I was a little upset with the authors, but a lot madder at myself:
fifteen minutes worth of scaffolding around the selection routine would have
displayed the bug, yet I wasted two days on it.
\smallskip\more
{\it Confession 2\/}. Several weeks before I firs wrote this column I was
working on a book of my own, which included a selection routine. I used
techniques of a program verification to derive the code, so I was sure it
was correct. After I placed the routine in the text, I wondered it was even
worth my time to test it. I hemmed and hawed, trying to decide......
\medskip}
  
\noindent The conclusion and another confession come later in this column.

This column is about the testing and debugging tools I use on subtle
algorithms. We'll start by scrutinizing two subroutines, complete with several
common bugs to make our study more interesting. As a reward for plowing through
all the code, this column concludes by describing a little subroutine library
and some tests of its correctness; I hope that the library will make it easy
for you to use correct version of these routine in your programs.
$$
\frame{.5}{4}{56mm}{\hfil\bf Warning --- Buggy Code Ahead\hfil}
$$

\vskip-\medskipamount

\section Binary Search

The ``black box'' approach is at one extreme of testing: without knowing the
structure of the program, hence viewing it as a black box, the tester presents
a series of inputs and analyzes the out for correctness. This section is about
a testing approach at the opposite extreme: the code is in a white box,%
{\eightpoint\parindent=8pt\footnote\dag{Logic dictates that the boxes should be
``opaque'' and ``transparent'' (``painted'' and ``glass''?), but I'll stick
with the traditional and illogical black and white.}} and we watch it as it
works.

The code we'll study is a binary search. Here is the routine, together with a
simple testbed:
\begindisplay
\vbox{
\+\bf function ${\it bsearch}(t,l,u,m)\ \{$\cr
\+\quad$l\leftarrow1;\ u\leftarrow n$\cr
\+\quad\bf while $(l\le u)\ \{$\cr
\+\qquad{\it print} |" "|$,l,m,u$\cr
\+\qquad\bf if \phantom{else} &$(x[m]<t)\ l=m$\cr
\+\qquad\bf else if &$(x[m]>t)\ u=m$\cr
\+\qquad\bf else return $m$\cr
\+\quad$\}$\cr
\+\quad\bf return $0$\cr
\+$\}$\cr}
\enddisplay
\vbox{
|$1=="fill"   { n = $2; for (i = 1; i <= n; i++) x[i] = 10*i }|\par
|$1=="n"      { n = $2 }|\par
|$1=="x"      { x[$2] = $3 }|\par
|$1=="print"  { for (i = 1; i <= n; i++) print i ":\t" x[i] }|\par
|$1=="search" { t = bsearch($2); print "result:", t }|}
\smallskip\noindent
The Awk binary search function has the single argument $t$; later elements in
the parameter list are the local variables. It should return an index of $t$ in
$x[1\ltdot n]$ if $t$ is present in the array, and zero otherwise. The {\it
print\/} statement traces the values of $l$, $m$ and $u$ throughout the search
(the lower end, middle, and upper end of the current range.) I indicate that it
is scaffolding by placing it in the left margin. Can you spot any problems with
the code?

The bottom five lines of the program are Awk ``pattern-action'' pairs. If the
input typed by the user matches the pattern on the left, then the code within
brackets on the right is executed. The pattern in the first pair is ture if the
first field of the input line typed by the user ({\tt\$1}) is |fill|. On such
lines, the number in the second field ({\tt\$2}) is assigned to the variable
$n,$ and the {\bf for} loop fills $n$ values of the array $x.$

Here's a transcript of a run of the program. I first typed {\tt fill 5}, and the
program created a sorted array of five elements. When I typed {\tt print}, the
program printed the contents of the array.

\bye
