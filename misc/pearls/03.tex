% More Programming Pearls
% Confessions of a Coder
% Jon Bentley
%
% Part I: Programming Technigues
% Column 3: Confessions of a Coder

\input pearls

\pageno=27
\column 3:Confessions of a Coder

\noindent
Most programmers spend a lot of time testing and debugging, but those activities
don't often get much attention in writing. This column describes how I tested
and debugged a few hard subroutines, with an emphasis of the {\it scaffolding\/}
I used in the process. The scaffolding around a building provides access to
components that workers couldn't otherwise reach. Software scaffolding consists
of temporary programs and data that give programmers access to system
components. The scaffolding isn't delivered to the customer, but it is
indispensible during testing and debugging.

Enough background, and on to two painful stories.

\medskip
\more
{\it Confession 1\/}.\enspace Several years ago I needed a selection routine in
the middle of a 500-line program. Because I knew the program was hard, I copied
a 20-line subroutine from an excellent algorithm text. The program usually
ran correctly, but failed every now and then. After two days of debugging, I
tracked the bug down to the selection routine. During most of the debugging,
that routine was above suspicion: I was convinced by the book's informal proof
of the routine's correctness, and I had checked my code several times to make
sure it matched the book. Unfortunately, a ``$<$'' in the book should have been
a ``$\le$''. I was a little upset with the authors, but a lot madder at myself:
fifteen minutes worth of scaffolding around the selection routine would have
displayed the bug, yet I wasted two days on it.
\smallskip\more
{\it Confession 2\/}.\enspace Several weeks before I firs wrote this column I
was working on a book of my own, which included a selection routine. I used
techniques of program verification to derive the code, so I was sure it
was correct. After I placed the routine in the text, I wondered whether it was
even worth my time to test it. I hemmed and hawed, trying to decide......
\medskip
  
\noindent The conclusion and another confession come later in this column.

This column is about the testing and debugging tools I use on subtle
algorithms. We'll start by scrutinizing two subroutines, complete with several
common bugs to make our study more interesting. As a reward for plowing through
all the code, this column concludes by describing a little subroutine library
and some tests of its correctness; I hope that the library will make it easy
for you to use correct version of these routine in your programs.
$$
\frame{.5}{4}{\bf Warning --- Buggy Code Ahead}
$$

\vskip-\medskipamount

\section Binary Search

The ``black box'' approach is at one extreme of testing: without knowing the
structure of the program, hence viewing it as a black box, the tester presents
a series of inputs and analyzes the out for correctness. This section is about
a testing approach at the opposite extreme: the code is in a white box,%
{\footnote\dag{Logic dictates that the boxes should be
``opaque'' and ``transparent'' (``painted'' and ``glass''?), but I'll stick
with the traditional and illogical black and white.}} and we watch it as it
works.

The code we'll study is a binary search. Here is the routine, together with a
simple testbed:
\begindisplay
\vbox{
\+\bf function ${\it bsearch}(t,l,u,m)\ \{$\cr
\+\quad$l\leftarrow1;\ u\leftarrow n$\cr
\+\quad\bf while $(l\le u)\ \{$\cr
\+\.{print " ", l, m, u}\cr
\+\qquad\bf if \phantom{else} &$(x[m]<t)\ l=m$\cr
\+\qquad\bf else if &$(x[m]>t)\ u=m$\cr
\+\qquad\bf else return $m$\cr
\+\quad$\}$\cr
\+\quad\bf return $0$\cr
\+$\}$\cr}
\enddisplay
\begindisplay
\vbox{\begingroup
\verbatim
$1=="fill"   { n = $2; for (i = 1; i <= n; i++) x[i] = 10*i }
$1=="n"      { n = $2 }
$1=="x"      { x[$2] = $3 }
$1=="print"  { for (i = 1; i <= n; i++) print i ":\t" x[i] }
$1=="search" { t = bsearch($2); print "result:", t }
!endgroup
\endgroup}
\enddisplay
The Awk binary search function has the single argument $t$; later elements in
the parameter list are the local variables. It should return an index of $t$ in
$x[1\ltdot n]$ if $t$ is present in the array, and zero otherwise. The
\.{print} statement traces the values of $l$, $m$ and $u$ throughout the search
(the lower end, middle, and upper end of the current range.) I indicate that it
is scaffolding by placing it in the left margin. Can you spot any problems with
the code?

The bottom five lines of the program are Awk ``pattern-action'' pairs. If the
input typed by the user matches the pattern on the left, then the code within
brackets on the right is executed. The pattern in the first pair is ture if the
first field of the input line typed by the user (\.{\$1}) is \.{fill}. On such
lines, the number in the second field (\.{\$2}) is assigned to the variable
$n,$ and the {\bf for} loop fills $n$ values of the array $x.$

Here's a transcript of a run of the program. I first typed \.{fill 5}, and the
program created a sorted array of five elements. When I typed \.{print}, the
program printed the contents of the array.
\begindisplay
\vbox{\begingroup
\verbatim
fill 5
print
1:      10
2:      20
3:      30
4:      40
5:      50
!endgroup
\endgroup}
\enddisplay
 Now come a few searches in that array. I typed \.{search 10}, and
the next three lines show the range narrowing to find that 10 is in position 1
in $x.$ The searches for 40 and 30 are also correct.
\begindisplay
\vbox{\begingroup
\verbatim
search  10
    1 3 5
    1 2 3
    1 1 2
result   1
search  40
    1 3 5
    3 4 5
result   4
search  30
    1 3 5
result   3
!endgroup
\endgroup}
\enddisplay
Unfortunately, the next search runs into trouble.
\begindisplay
\vbox{\begingroup
\verbatim
search  10
    1 3 5
    3 4 5
    4 4 5
    4 4 5
    4 4 5
    4 4 5
    4 4 5
    4 4 5
     ...
!endgroup
\endgroup}
\enddisplay
With this clue, can you find  the bug in the program?

Binary search is supposed to narrow the range $l\ltdot u$ until termination. The
assignment $l\leftarrow m$ usually does so, but when $l$ and $m$ are equal it
loop endlessly. The assignment should instead be $l\leftarrow m+1,$ which
excludes $m$ from the range. (The techniques of program verification help one
derive this code systematically; excluding $m$ is the key point in the proof of
termination.) The assignment $u\leftarrow m$ should similarly be changed to
$u\leftarrow m-1.$ The resulting correct binary search is in Appendix 2.

The \.{n} and \.{x} commands allow us to alter the arrays produced by
\.{fill}. To find how the correct code behaves on a two-element array with
equal elements, the command \.{x 2 10} sets $x[2]$ to 10, and the next
command set $n$ to 2.
\begindisplay
\vbox{\begingroup
\verbatim
fill 5
x 2 10
n 2
print
1:      10
2:      10
search 20
   1 1 2
   2 2 2
result: 0
!endgroup
\endgroup}
\enddisplay
The search for 20 then correctly reports that it is not in the array.

I'd be {\it really\/} surprised if someone shows me a bug in the final binary
search program. I used program verification techniques to prove the code
correct, and then I beat on it with the black-box test reproduced in Appendix 2.
Simple observation like those described in this section reassure me that the
code indeed behaves as I thought. That reassurance cost just six lines of Awk
scaffolding beyond the binary search code.

The techniques of this section are simple and well known. Unfortunately,
scaffolding is too often neglected by programmers. A few minutes spent testing
a prototype of a subtle algorithm like binary search can save hours of debugging
after it is incorporated in a large system. If a hard routine fails in a big
program, construct scaffolding so you can access it directly, or better yet,
build a small version in a supportive language like Awk.

\section Selection

The next program uses Hoare's algorithm to {\it select\/} the $k^{\rm th}$-%
smallest element of the array $x[1\ltdot n].$ Its job is to permute the elements
of $x$ so that $x[1\ltdot k-1]\le x[k]\le x[k+1\ltdot n].$ We will study this
routine in detail in Column 15:
\begindisplay
\vbox{
\+\bf function ${\it select}(l,u,k,i,m)\ \{$\cr
\+\quad\bf if $(l<u)\ \{$\cr
\+\qquad${\it swap}(l,{\it rand\_int}(l,u))$\cr
\+\qquad$m\leftarrow l$\cr
\+\qquad\bf for $(i\leftarrow l+1;\ i\le u;\ i\PP)$\cr
\+\qquad\quad\bf if $(x[i]<x[l])$\cr
\+\qquad\qquad${\it swap}(\PP m,i)$\cr
\+\qquad${\it swap}(l,m)$\cr
\+\qquad\bf if \phantom{else} &$(m<k)$ ${\it select}(m+1,u,k)$\cr
\+\qquad\bf else if &$(m>k)$ ${\it select}(l, m-1, k)$\cr
\+\quad$\}$\cr
\+$\}$\cr}
\enddisplay
The code was easy to prove correct; it passed all tests on the first try.

That program uses ``tail recursion'': the recursive call is the last statement
in the procedure. Tail recursion can always be transformed to iteration by
replacing subroutine call by assignment and loop statements. The next version
replaces the recursive routine with a {\bf while} loop, and this returns us to
my next confession. My first mistake, of course, was in debugging whether to
test the code. Any author who errs as often as I do should either test the
program or label it with ``WARNING --- UNTESTED CODE''. The second mistake is in
the selection routine itself; any ideas?
\begindisplay
\vbox{
\+\bf function ${\it swap}(i,j,t)\ \{\ t\leftarrow x[i];\ x[i]\leftarrow x[j];
  \ x[j]\leftarrow t\ \}$\cr
\smallskip
\+\bf function ${\it select}(k,l,u,i,m)\ \{$\cr
\+\quad$l\leftarrow1;\ u\leftarrow n$\cr
\+\quad\bf while $(l\le u)\ \{$\cr
\+\.{print l, u}\cr
\+\qquad${\it swap}(l,{\it rand\_int}(l,u))$\cr
\+\qquad$m\leftarrow l$\cr
\+\.{comps = comps + u-1}\cr
\+\qquad\bf for $(i\leftarrow l+1;\ i\le u;\ i\PP)$\cr
\+\qquad\quad\bf if $(x[i]<x[l])$\cr
\+\qquad\qquad${\it swap}(\PP m,i)$\cr
\+\qquad${\it swap}(l,m)$\cr
\+\qquad\bf if \phantom{else} &$(m<k)$ $l=m+1$\cr
\+\qquad\bf else if &$(m>k)$ $u=m-1$\cr
\+\quad$\}$\cr
\+$\}$\cr}
\enddisplay
\begindisplay
\vbox{\begingroup
\verbatim
$1=="fill"  { n = $2; for (i = 1; i <= n; i++) x[i] = rand() }
$1=="n"     { n = $2 }
$1=="x"     { x[$2] = $3 }
$1=="print" { for (i = 1; i <= n; i++) print "   ", x[i] }
$1=="sel"   { comps = 0; select($2); print "  compares:", comps
              print "  compares/n:", comps/n
              for (i=1;   i < k;  i++) if (x[i] > x[k]) print i
              for (i=k+1; i <= n; i++) if (x[i] < x[k]) print i
            }
!endgroup
\endgroup}
\enddisplay
We'll first watch the program at work. The \.{fill} command sprinkles random
numbers in the range $[0,1]$ into the array, and \.{print} is like that in the
previous program.
\begindisplay
\vbox{\begingroup
\verbatim
fill 5
print
    0.93941
    0.532356
    0.392797
    0.446203
    0.535331
!endgroup
\endgroup}
\enddisplay
The command \.{sel 3} partitions the array so that the third-smallest element
is in $x[3]$. It displays the computation as it progresses, and also checks the
correctness of the final answer. The subsequent \.{print} command then displays
the partitioned array.
\begindisplay
\vbox{\begingroup
\verbatim
sel 3
1 5
3 5
3 5
3 5
3 4
  compares: 11
  compares/n: 2.2
print
    0.446203
    0.392797
    0.532356
    0.535331
    0.93941
!endgroup
\endgroup}
\enddisplay
Although the code produces the correct answer, the trace is suspicious.
Can you find the hint of the bug in that history?

We'll corner the bug in a more perverse array, built and displayed as follows.
\begindisplay
\vbox{\begingroup
\verbatim
fill 2
x 1 5
x 2 5
print
    5
    5
!endgroup
\endgroup}
\enddisplay
Selecting the second-smallest element works just fine, but there are problems in
finding the smallest.
\begindisplay
\vbox{\begingroup
\verbatim
sel 2
1 2
  compares: 1
  compares/n: 0.5
sel 1
1 2
1 2
1 2
1 2
 ...
!endgroup
\endgroup}
\enddisplay
With this information, it was easy for me to spot the bug and to handle the tail
recursion more carefully; the final code is in Section 15.2 and Appendix 2. (The
program computed many correct answers only because the bug was often hidden by
the randomizing {\it swap\/} statement. Randomization, for better or worse,
often compensates for bugs.)

Apart from its correctness problem, the origianl code has a ``performance bug'':
even when it gives the right answer, it takes too logn. We'll see in Column 15
that a correct selection program requires roughly $3.4n$ comparisions to find
the median of an $n$-element array. These test (and a dozen more like them) show
that the performance of the correct selection routine from Appendix 2 is in the
right ballpark:
\begindisplay
\vbox{\begingroup
\verbatim
fill 50
sel 25
  compares: 134
  compares/n: 2.68
fill 100
sel 50
  compares: 363
  compares/n: 3.63
!endgroup
\endgroup}
\enddisplay
I have removed the output of the \.{print} statements that traced the value of
$l$ and $u$ to save space in this column; it was a pleasure to watch them
behave properly as I conducted the real tests.

\section A Subroutine Library

Before this column was originally published in {\sl Communications of the
ACM\/}, many programmers had mentioned that they used pseudocode published in
previous columns as a basic for implementing an algorithm in their favorite
language. For some time I had wanted to collect the algorithms into a little
library, but the code was always too long. When the Awk language acquired
functions in early 1985, I realized that it was the ideal vehicle for
communicating a set of useful subroutines in clean, succinct, and tested code.

The designer of a industrial-strength subroutine library must face the difficult
problems of portability, efficiency, and general interfaces. The designer must
also choose an implementation language, which gives programmers in that language
easy access to the routine. Unfortuantely, that choice useally denies the
routine to users of other language.

Appendix 2 is a set of ``language-independent'' subroutines, suitable for
copying into various implementation laguages. Since no sane programmer would
code a serious application of this nature in Awk,{\eightpoint\parindent=8pt
\footnote\dag{Apart from sequential search and insertion sort, all subroutines
in the library use designed for efficient asymptotic running times --- usually
$O(n\log n).$ For problems on arrays, the overhead of Awk's interpretation and
associative arrays renders it orders of magnitude slower than conventional
compiled languages.}} the code is equally useful to a programmer using an
Algol-like language. The routines are short. Tradeoffs were made for brevity and
against twenty- and thirty-percent improvements in efficiency. There are no
interfaces; all routines operate on the array $x[1\ltdot n].$ These short,
clean, correct routines provide a useful starting point for programmers without
a better library.

The routines themselves are less than half the program text; the remainder is a
black-box correctness test. (Scaffolding is often this big. In Chapter 13 of
{\sl The Mythical Man Month}, Fred Brooks states that there might be ``half as
much code in scaffolding as there is in product''; in Section 1.4.1 of {\sl
Fundamental Algorithms\/} Knuth raised that to as much scaffolding as delivered
code.) The tests all have the same structure: an input is constructed, the
routine is called, and the answer is then checked for correctness. The progress
of the tests is reported as they are run. This is helpful for locating any
error, and encouraging for the runs that report no errors --- at least you know
they did something. Most tests are run for all values of $n$ in $0\ltdot{\it
bign}$, where ${\it bign}=12;$ at most $O(n^3)$ work is performed at each value
of $n$. The sorting test examines $n!$ random permutations for $n$ in $0\ltdot
{\it smalln},$ where ${\it smalln}=5.$ That gives a high probability of
uncovering any permutation on which the algorithm fails. (Most random tests
aren't so through.) The complete program required seven minutes of CPU time on
a VAX-11/750.

With the exception of the selection routine discussed earlier (and described in
detail in Column 15), I wrote the Awk routines by transliterating psedocode
published in previous columns. Those columns give informal correctness
arguements using techniques of program verification. I had tested all the
routines before publication, using a combination of watching, measuring and
black-box testing; some columns report bugs I found during that process. I
therefore wasn't suprised when testing uncovered no logical errors in the
routines; I fixed a few syntax errors in less than a minute each.

Testing did, however, uncover two interesting bugs in Awk. The first manifested
itself as an infinite loop in the binary search routine {\it bsearch}. When I
extracted from Appendix 2 a tiny scaffolding program like the one in Section
3.1, the infinite loop was obvious. I presented the resulting fifteen lines to
Brian Kernighan, who was adding several new features to Awk at that time. I was
unsure of whether the bug was in my program or his, but hopeful enough that it
might be Kernighan's fault to risk certain ridicule if the fault were mine.
Changing the line
\begindisplay
\.{else return m}
\enddisplay
to
\begindisplay
\.{else \{ print "returning"; return m \}}
\enddisplay
showed that th Awk interpreter's new functions had the common bug of not
properly executing a \.{return} from within a loop. After the bug was
identified, Kernighan fixed Awk within ten minutes.

I then ran back to my terminal to watch with glee as the test of binary search
ran successfully for all $n$ in the range $1\ltdot9.$ I was heartbroken to see
the test fail for $n=10.$ At that time, ${\it bign}=10.$ Because I couldn't
think of any good reason why should fail at $n=10,$ I re-ran the test ${\it
bign}=9$ and ${\it bign}=11,$ hoping that the problem was in the last test.
Unfortunately, the code consistently worked properly up through 9 and failed at
10 and 11. What changes between 9 and 10?

Awk variables can be either numbers or strings. The Awk manual states that if
the two operands in a comparison are both numeric then they are compared as
numbers, otherwise they are compared as strings. Becasuse of unusal
circumstances in this program involving functins calls, the interpreter
inappropriately observed that the string ``\.{10}'' precedes the string
``\.{5}''. I created a six-line program that tickled this bug, and Kernighan
had the problem fixed the next day.

\section Principles

This column has touched on several tasks common in the programmer's day-to-day
life. None is particularly glamorous, but all are important.

{\it Scaffolding.}\quad This column illustrates prototype routines, print
routines to observe program behavior, measurement code, and component tests.
Other scaffolding includes test data (dummy files and data structures) and
program ``stubs'' that facilitate top-down testing by simulating unfinished
routines.

{\it Special-Purpose Language.}\quad The right language can make a program an
order of magnitude shorter and cleaner. Exploit the strengths of the languages
available to you. Awk is a fine language for prototyping algorithms; its
associative arrays let you simulate many common data structure; its fields,
implicit loop, and pattern-action pairs simplify input/output; implicit
declaration and initialization of vaiables lead to succint programs. Chapter 7
of {\sl The AWK Programming Language\/} (cited in Section 2.6) present more
information on using Awk to experiment with algorithms. Section 13.2 and
Solution 14.6 give Awk scaffolding for two subtle algorithms.

{\it Testing and Debugging.}\quad This column concentrated on testing and
debugging small components. White-box views of the computation initially show
that the code behaves as we expected. Black-box tests are later used to
increase our confidence in the correctness of the routine.

{\it Bug Reports.}\quad The component test of the subroutine library
inadvertently turned into a system test for Awk's recently added functions.
Kernighan call this the ``new user phenomenon'': each new user of a fresh
system uncovers a new class of bugs. I pushed harder on functions that previous
users. On the two occasions when the 300-line program tickled an Awk bug, I
reproduced the bizarre behavior in a short program before reporting the problem
(fifteen lines in one case, sis in the other). Stu Feldman of Bell
Communications Research speaks from years of experience maintaining a Fortran
compiler:

{\medskip\more
The program author, support organization, and your friends will all ignore you
if you send a bug report and a 25{,}000 line program listing. It took me several
years to teach [name changed to protect the guilty] this fact and get him to
work on it. Techniques involve staring at code, intuition, bisection (try
throwing out the last half of the subroutine), etc.
\medskip}

\noindent If you find a bug, report it with the smallest possible test case.

{\it The Role of Program Verification.}\quad I need all the help I can get in
making a correct program. Informal verification techniques help me write the
code and check it before I ever implement it, and testing is crucial after I
have the code in hand. Because I'm getting better at verification, I'm no longer
astounded when a small but complex routine works the first time. If it doesn't
work, I use testing and debugging to help me locate the invalid assertions and
fix them along with the code (I'm usually able to resist those urges to ``just
change it until it works'' --- I try to write only programs that I understand).
Appendix 2 illustrates two uses of assertions: the pre- and postconditions of a
routine provide a precise and concise specification of it behavior, and
assertion comments in the code (especially loop invariants) explain the
algorithms. For a more direct application of verification ideas to testing, see
Problem 3.

\section Problems

\vskip-\medskipamount

\prob
Build scaffolding that allows you to observe the behavior of routines in
Appendix 2. Heaps are particularly intersting to watch.

\prob
Improve the {\it assert\/} routine of Appendix 2 so that it tell more about the
location of error.

\prob
The {\it assert\/} routine can also be used in white-box testing: change the
assertion that are presently comments into calls to the assert routine. Rewrite
assertions in that form for one of the routines in Appendix 2. Does that
strengthen the tests in the sense of Problem 4?

\prob
Evaluate the quality of the black-box test in Appendix 2 by introducing bugs
into the various routines. Which bugs are caught by which tests?

\prob
Rewrite the programs in this column in another language. How long are they
compared to the Awk code?

\prob
Write scaffolding that allows you to time the performance of various algorithms
in Appendix 2. How can you present the results graphically?

\prob
Build a subroutine library like that in Appendix 2 for a different problem
domain, such as graph algorithms. Strive for short, correct algorithms that are
also reasonabley efficient.

\prob
By the literal specifications in Appendix 2, thsi is a correct sorting
algorithm:
\begindisplay
\vbox{
\+\bf for $(i=1;\ i\le n;\ i\PP)$\cr
\+\quad$x[i]=i;$\cr}
\enddisplay
A sorting algorithm must, of course, also guarantee that the output array is a
permutation of the input. The sorting, heap, and selection algorithms in
Appendix 2 guarantee this property by altering the array only by using the
{\it swap\/} routine. How would you test a less structured program for the
permutation property?

\bye
