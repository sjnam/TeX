% Programming Pearls
% Jon Bentley
%
% Part II : Performance
% Column 7: Algorithm Design Techniques

\input luamplib.sty
\input pearls

\def\lhead{\eightrm PROGRAMMING PEARLS\hfill COLUMN \colno}
\pageno=69
\column 7:Algorithm Design Techniques

\noindent
Column 2 describes the ``everyday'' impact that algorithm design can have on
programmers: an algorithmic view of a problem gives insights that can make a
program simpler to understand and to write. In this column we'll study a
contribution of the field that is less frequent but more impressive:
sophisticated algorithmic methods sometimes lead to dramatic performance
improvements.

This column is built around one small problem, with an emphasis on the algorithms
that sovle it and the techniques used to design the algorithms. Some of the
algorithms are a little complicated, but with justification. While the first
program we'll study takes thirty-nine days to solve a problem of size ten
thousand, the final one solves the same problem in less than a second.

\section The Problem and a Simple Algorithm

The problem arose in one-dimensional pattern recognition; I'll describe its
history later. The input is a vector $x$ of $n$ real numbers; the output is the
maximum sum found in any {\sl contiguous\/} subvector of the input. For
instance, if the input vector is
$$
\mplibcode
input boxes
beginfig(0);
boxjoin(a.se=b.sw; a.ne=b.nw);
boxit.a(btex \strut$31$ etex);
boxit.b(btex $-41$ etex);
boxit.c(btex 59 etex);
boxit.d(btex 26 etex);
boxit.e(btex $-53$ etex);
boxit.f(btex 58 etex);
boxit.g(btex 97 etex);
boxit.h(btex $-93$ etex);
boxit.i(btex $-23$ etex);
boxit.j(btex 84 etex);
drawboxed(a,b,c,d,e,f,g,h,i,j);
pair A,B;
A=c.c-(0,6mm);
label.bot(btex $3$ etex, A);
drawarrow A -- c.s;
B=g.c-(0,6mm);
label.bot(btex $7$ etex, B);
drawarrow B -- g.s;
endfig;
\endmplibcode
$$
then the program returns the sum of $x[3\ltdot7],$ or 187. The problem is easy
when all the numbers are positive -- the maximum subvector is the entire input
vector. The rub comes when some of the numbers are negative: should we include a
negative number in hopes that the positive numbers to its sides will compensate
for its negavie contribution? To complete the definition of the probelm, we'll
say that when all input are negative the maximum sum subvector is the empty
vector, which has sum zero.

The obvious program for this task iterates over all pairs of integers $l$ and $u$
satisfying $1\le l\le u\le n;$ for each pair it computes the sum of
$x[l\ltdot u]$ and checks whether that sum is than the maximum sum so far. The
pseudocode for Algorithm 1 is
\begindisplay
\vbox{
\+${\it MaxSoFar}\leftarrow 0.0$\cr
\+\bf for $l\leftarrow1$ to $n$ do\cr
\+\quad\bf for $u\leftarrow1$ to $n$ do\cr
\+\qquad${\it Sum}\leftarrow 0.0$\cr
\+\qquad\bf for $i\leftarrow l$ to $u$ do\cr
\+\qquad\quad${\it Sum}\leftarrow {\it Sum}+x[i]$\cr
\+\qquad\tt /* {\it Sum} now contains the sum of $x[l\ltdot u]$ */\cr
\+\qquad${\it MaxSoFar}\leftarrow {\it max}({\it MaxSoFar}, {\it Sum})$\cr}
\enddisplay
This code is short, straightforward, and easy to understand. Unfortunately, it
has the several disadvantage of beging slow. On the computer I typically use, for
instance, the code takes about an hour if $n$ is 1000 and thirty-nine days if $n$
is 10{,}000; we'll get to timing details in Section 7.5.

Those times are anecdotal; we get a different kind of feeling for the algorithm's
efficiency using the ``big-oh'' notation described in Section 5.1. The statements
in the outermost loop are executed exactly $n$ times, and those in the middle
loop are executed at most $n$ times in each execution of the outer loop.
Multiflying those two factors of $n$ shows that the four lines contains in the
middle loop are executed $O(n^2)$ times. The loop in those four lines is never
executed more than $n$ times, so its cost is $O(n).$ Multiplying the cost per
inner loop times its number of executions shows that the cost of the entire
program is proportional to $n$ cubed, so we'll refer to this as a cubic
algorithm.

This example illustrates the technique of big-oh analysis of run time and many
of its strengths and weaknesses. Its primary weakness is that we still don't
really know the amount of time the program will take for any particular input;
we just know that the number of steps it executes is $O(n^3).$ Thas weakness is
often compensated for by two strong points of the method. Big-oh analyses are
usually easy to perform (as above), and the asymptotic run time is often
sufficient for a back-of-the-envelope calculation to decide whether or not a
program is efficient enough for a given application.

The next several sections use asymptotic run time as the only measure of program
efficiency. If that makes you uncomfortable, peek ahead to Section 7.5, which
shows that for this program such analyses are extremely informative. Before you
read further, though, take a minute to try to find a faster algorithm.

\section Two Quadratic Algorithms

Most programmers have same response to Algorithm 1: ``There's an obvious way to
make it a lot faster.'' There are two obvious ways, however, and if one is
obvious to a given programmer tehn the other often isn't. Both algorithms are
quadratic --- they take $O(n^2)$ steps on an input size of $n$ --- and both
archieve their run time by computing the sum of $s[l\ltdot u]$ in a constant
number of steps rather than in the $u-l+1$ steps of Algorithm 1. But the two
quadratic algorithms use very different methods to compute the sum is constant
time.

The first quadratic algorithm computes the sum quickly by noticing that the sum
of $x[l\ltdot u]$ has an intimate relationship to the sum previously computed,
that of $x[l\ltdot u-1].$ Exploiting that relationship leads to Algorithm 2.
\begindisplay
\vbox{
\+${\it MaxSoFar}\leftarrow 0.0$\cr
\+\bf for $l\leftarrow1$ to $n$ do\cr
\+\qquad${\it Sum}\leftarrow 0.0$\cr
\+\qquad\bf for $u\leftarrow l$ to $n$ do\cr
\+\qquad\quad${\it Sum}\leftarrow {\it Sum}+x[u]$\cr
\+\qquad\quad\tt /* {\it Sum} now contains the sum of $x[l\ltdot u]$ */\cr
\+\qquad\quad${\it MaxSoFar}\leftarrow {\it max}({\it MaxSoFar}, {\it Sum})$\cr}
\enddisplay
The statements inside the first loop are executed $n$ times, and those inside
the second loop are executed at most $n$ times on each execution of the outer
loop, so the total run time is $O(n^2).$

An alternative quadratic algorithm computes the sum in the inner loop by
accessing a data structure built before the outer loop is ever executed. The $I^
{\rm th}$ element of {\it CumArray\/} contains the cumulative sum of the values
in $x[1\ltdot l],$ so the sum of the values in $x[\l\ltdot u]$ can be found by
computing ${\it CumArray}[u]-{\it CumArray}[l-1].$ This results in the following
code for Algorithm 2b.
\begindisplay
\vbox{
\+${\it CumArray}[0]\leftarrow 0.0$\cr
\+\bf for $i\leftarrow1$ to $n$ do\cr
\+\qquad${\it CumArray}[i]-{\it CumArray}[i-1]+x[i]$\cr
\+${\it MaxSoFar}\leftarrow 0.0$\cr
\+\bf for $l\leftarrow1$ to $n$ do\cr
\+\quad\bf for $u\leftarrow l$ to $n$ do\cr
\+\qquad\quad${\it Sum}\leftarrow{\it CumArray}[u]-{\it CumArray}[l-1]$\cr
\+\qquad\quad\tt /* {\it Sum} now contains the sum of $x[l\ltdot u]$ */\cr
\+\qquad\quad${\it MaxSoFar}\leftarrow {\it max}({\it MaxSoFar}, {\it Sum})$\cr}
\enddisplay
This cod takes $O(n^2)$ times; the analysis is exactly the same as the analysis
of Algorithm 2.

The algorithm we've seen so far inspect all possible pairs of starting and ending
values of subvectors and consider the sum of the numbers in that subvector.
Because there are $O(n^2)$ subvectors, any algorithm that inspects all such
values must take at least quadratic time. Can you think of a way to sidestep this
problem and achieve an algorithm that runs in less time?

\section A Divide-and-Conquer Algorithm

Our first subquadratic altorithm is complicated; if you get bogged down in its
details, you won't lose much by skipping to the next section. It is based on the
following divide-and-conquer schema:
\smallskip
{ \narrower\noindent\sl To solve a problem size $n,$ recursively solve two
  subproblems of size approximately $n/2,$ and combine their solutions to yield
  a solution to the complete problem.\par}
\smallskip
\noindent In this case the original problem deals with a vector of size $n,$
so the most natural way to divide it into subproblems is to create two
subvectors of approximately equal size, which we'll call $A$ and $B.$
$$
\mplibcode
input boxes
beginfig(0);
boxjoin(a.se=b.sw; a.ne=b.nw);
boxit.a(btex\strut\hbox to 1.6in{\strut\hfil$A$\hfil}etex); a.dx=0;
boxit.b(btex\strut\hbox to 1.6in{\strut\hfil$B$\hfil}etex); b.dx=0;
drawboxed(a,b);
endfig;
\endmplibcode
$$
We then recursively find the maximum subvectors in $A$ and $B,$ which we'll call
$M_A$ and $M_B.$
$$
\mplibcode
input boxes
beginfig(0);
boxjoin(a.se=b.sw; a.ne=b.nw);
boxit.a(btex\strut\hbox to 0.2in{\strut\hfil}etex); a.dx=0;
boxit.b(btex\strut\hbox to 0.7in{\strut\hfil$M_A$\hfil}etex); b.dx=0;
boxit.c(btex\strut\hbox to 0.7in{\strut\hfil}etex); c.dx=0;
boxit.d(btex\strut\hbox to 0.3in{\strut\hfil}etex); d.dx=0;
boxit.e(btex\strut\hbox to 0.5in{\strut\hfil$M_B$\hfil}etex); e.dx=0;
boxit.f(btex\strut\hbox to 0.8in{\strut\hfil}etex); f.dx=0;
drawboxed(a,b,c,d,e,f);
endfig;
\endmplibcode
$$
It is tempting to think that we have solved the problem because the maximum sum
bsubvector of the entire vector must be either $M_A$ or $M_B,$ and that is almost
right. In fact, the maximum is either entirely in $A,$ entirely in $B,$ or it
corsses the border between $A$ and $B;$ we'll call that $M_C$ for the maximum
{\it crossing\/} the border.
$$
\mplibcode
input boxes
beginfig(0);
boxjoin(a.se=b.sw; a.ne=b.nw);
boxit.a(btex\strut\hbox to 0.8in{\strut\hfil}etex); a.dx=0;
boxit.b(btex\strut\hbox to 1in{\strut\hfil$M_C$\hfil}etex); b.dx=0;
boxit.c(btex\strut\hbox to 1.4in{\strut\hfil}etex); c.dx=0;
drawboxed(a,b,c);
endfig;
\endmplibcode
$$
Thus our divide-and-conquer algorithm will compute $M_A$ and $M_B$ recursively,
compute $M_C$ by some other means, and then return the maximum of the three.

That description is almost enouth to write code. All we have left to describe
is how we'll handle small vectors and how we'll compute $M_C.$ The former is
easy: the maximum of a one-element vector is the only value in the vector or zero
if that number is negative, and the maximum of a zero-element vector as
previously defined to be zero. To compute $M_C$ we observe that its component in
$A$ is the largest subvector starting at the boundary and reaching into $A,$ and
similarly for its component in $B.$ Putting these facts together leads to the
following code for Algorithm 3, which is originally invoked by the procedure call
\begindisplay
\vbox{
\+${\it Answer}[0]\leftarrow {\it MaxSum}(1,n)$\cr
\+\bf recursive function ${\it MaxSum}(l,u)$\cr
\+\quad\bf if $l>u$ then\qquad {\tt /* Zero-element vector */}\cr
\+\qquad\bf return $0.0$\cr
\+\quad\bf if $l=u$ then\qquad {\tt /* One-element vector */}\cr
\+\qquad\bf return ${\it max}(0.0,x[l])$\cr
\+\cr
\+\quad$M\leftarrow (l+u)/2$\qquad\tt /* $A$ is $x[l\ltdot m]$, $B$ is
$x[m+1\ltdot u]$ */\cr
\+\quad\tt /* Find max crossing to left */\cr
\+\qquad${\it Sum}\leftarrow0.0;\ {\it MaxToLeft}\leftarrow0.0$\cr
\+\qquad\bf for $i\leftarrow m$ downto $l$ do\cr
\+\qquad\quad${\it Sum}\leftarrow{\it Sum}+x[i]$\cr
\+\qquad\quad${\it MaxToLeft}\leftarrow{\it max}({\it MaxToLeft},{\it Sum})$\cr
\+\quad\tt /* Find max crossing to right */\cr
\+\qquad${\it Sum}\leftarrow0.0;\ {\it MaxToRight}\leftarrow0.0$\cr
\+\qquad\bf for $i\leftarrow m+1$ downto $u$ do\cr
\+\qquad\quad${\it Sum}\leftarrow{\it Sum}+x[i]$\cr
\+\qquad\quad${\it MaxToRight}\leftarrow{\it max}({\it MaxToRight},{\it Sum})$\cr
\+\quad${\it MaxCrossing}\leftarrow{\it MaxToLeft}+{\it MaxToRight}$\cr}
\enddisplay
\begindisplay
\vbox{
%\+\cr
\+\quad${\it MaxInA}\leftarrow {\it MaxSum}(l,m)$\cr
\+\quad${\it MaxInB}\leftarrow {\it MaxSum}(m+1,u)$\cr
\+\quad\bf return ${\it max}({\it MaxCrossing},{\it MaxInA},{\it MaxSum})$\cr}
\enddisplay
The code is complicated and easy to get wrong, but it solves the problem in $O(n
\log n)$ time. There are a number of ways to prove this fact. An informal
argument observes that the algorithm does $O(n)$ work on each of $O(\log n)$
levels of recursion. The argument can be made more precise by the use of
recurrence relations. If $T(n)$ denotes the time to solve a problem of size $n,$
then $T(1)=O(1)$ and
\begindisplay
$T(n) = 2T(n/2)+O(n).$
\enddisplay
Problem 11 shows that this recurrence has the solution $T(n)=O(n\log n).$

\section A Scanning Algorithm

We'll now use the simplest kind of algorithm that operates on arrays: it starts
at the left end (element $x[1]$) and scans through to the right end (element
$x[n]$), keeping track of the maximum sum subvector seen so far. The maximum is
initially zero. Suppose that we've solved the problem for $x[1\ltdot l-1];$ how
can we extend that to a solution for the first $i$ elements? We use reasoning
similiar to that of the divide-and-conquer algorithm: the maximum sum in the
first $i$ elements is either the maximum sum in the first $i-1$ elements (which
we'll call {\it MaxSoFor\/}), or it is that of a subvector that ends in position
$i$ (which we'll call {\it MaxEndingHere\/}).
$$
\mplibcode
input boxes
beginfig(0);
boxjoin(a.se=b.sw; a.ne=b.nw);
boxit.a(btex\hbox to 0.4in{\strut\hfil}etex); a.dx=0;
boxit.b(btex\hbox to 0.8in{\strut\hfil\it MaxSoFar\hfil}etex); b.dx=0;
boxit.c(btex\hbox to 0.7in{\strut\hfil}etex); c.dx=0;
boxit.d(btex\hbox to 1.3in{\strut\hfil\it MaxEndingHere\hfil}etex); d.dx=0;
drawboxed(a,b,c,d);
label.bot(btex $i$ etex, d.se-(1mm,0));
endfig;
\endmplibcode
$$
Recomputing {\it MaxEndingHere\/} from scratch using code like that in Algorithm
3 yields yet another quadratic algorithm. We can get around this by using the
technique that led to Algorithm 2: instead of computing the maximum subvector
ending in position $i$ from scratch, we'll use the maximum subvector that ends
in position $i-1.$ This results in Algorithm 4.
\begindisplay
\vbox{
\+${\it MaxSoFar}\leftarrow0.0$\cr
\+${\it MaxEndingHere}\leftarrow0.0$\cr
\+\bf for $i\leftarrow1$ to $n$ do\cr
\+\quad\tt /* \cleartabs&\tt Invariant:$\ {\it MaxEndingHere}$ and {\it MaxSoFar}
  \cr
\+\quad&\tt are accurate for $x[1\ltdot i-1]$ */\cr
\+\quad${\it MaxEndingHere}\leftarrow{\it max}({\it MaxendingHere}+x[i],0.0)$\cr
\+\quad${\it MaxSoFar}\leftarrow{\it max}({\it MaxSoFar},
  {\it MaxEndingHere})$\cr}
\enddisplay
The key to understanding this program is the variable {\it MaxEndingHere\/}.
Before the first assignment statement in the loop, {\it MaxEndingHere\/} contains
the value of the maximum subvector ending in position $i-1;$ the assignment
statement modifies it to contain the value of the maximum subvector ending in
position $i.$ The statement increases it by the value $x[i]$ so long as doing so
keeps it positive; when it goes negative, it is reset to zero because the maximum
subvector ending at $i$ is the empty vector. Although the code is subtle, it is
short and fast: its run time is $O(n),$ so we'll refer to it as a linear
algorithm. David Gries systematically derives and verifies this algorithm in his
paper ``A Note on the Standard Strategy for Developing Loop Invariants and
Loops'' in the journal {\sl Science of Computer Programming 2,} pp. 207-214.

\section What Does It Matter?

So far I've played fast and loose with ``big-ohs''; it's time for me to come
clean and tell about the run times of the programs. I implemented the four
primary algorithms (all except Algorithm 2b) in the C language on a VAX-11/750,
timed them, and extrapolated the observed run times to achieve the following
table.
$$
\vbox to 5cm{\vfill
  \centerline{big table here}
  \vfill}
$$
This table makes a number of points. The most important is that proper algorithm
design can make a big difference in run time; that point is underscored by the
middle rows. The last two rows show how increases in problem size are related
to increases in run time.

Another important point is that when we're comparing cubic, quadratic, and linear
algorithms with one another, the constant factors of the programs don't matter
much. (The discussion of the $O(n!)$) algorithm in Section 2.4 shows that
constant factors matter even less in functions that grow faster than
polynomially.) To underscore this point, I conducted an experiment in which I
tried to make the constant factors of two algorithms differ by as much as
possible. To achieve a huge constant factor I implemented Algorithm 4 on a BASIC
interpreter on a Radio Shack TRS-80 Model III microcomputer. For the other end of
the spectrum, Eric Grosse and I implemented Algorithm 1 in fine-tuned FORTRAN on
a Cray-1 supercomputer. We got the disparity we wanted: the run time of the cubic
algorithm was measured as $3.0n^3$ nanoseconds, while the run time of the linear
algorithm was $19.5n$ milliseconds, or $19{,}500{,}000n$ nanoseconds. This table
shows how those expressions translate to times for various problem sizes.
\font\sc=cmcsc10
$$
\vbox{\offinterlineskip
\hrule
\vskip 1pt
\halign{&\vrule#&
  \strut\quad\hfil#\hfil\quad\cr
\multispan7\hrulefill\cr
\omit&\omit&height2pt&\omit&height2pt&\omit\cr
\omit&\omit&&\sc Cray-1,&&\sc Trs-80,&\omit\cr
\omit&$N$&&\sc Fortran,&&\sc Basic,&\omit\cr
\omit&\omit&&\sc Cubic Algorithm&&\sc Linear Algorithm&\omit\cr
\omit&\omit&height2pt&\omit&height2pt&\omit\cr
\multispan7\hrulefill\cr
\omit&\omit&height2pt&\omit&height2pt&\omit\cr
\omit&10&&3.0 microsecs&&200 millisecs& \omit\cr
\omit&100&&3.0 millisecs&&2.0 secs& \omit\cr
\omit&1000&&3.0 secs&&20 secs&\omit\cr
\omit&10{,}000&&49 mins&&3.2 mins& \omit\cr
\omit&100{,}000&&35 days&&32 mins&\omit\cr
\omit&1{,}000{,}000&&95 yrs&&5.4 hrs&\omit\cr
\omit&\omit&height2pt&\omit&height2pt&\omit\cr
\multispan7\hrulefill\cr}}
$$
The difference in constant factors of six and a half millian allowed the cubic
algorithm to start off faster, but the linear algorithm was bound to catch up.
The break-even point for the two algorithms is around 2{,}500, where each takes
about fifty seconds.
$$
\vbox to 4cm{\vfill
  \centerline{a graph here}
  \vfill}
$$


\section Principles

The history of the problem sheds light on the algorithm design techniques. The
problem arose in a pattern-matching procedure designed by Ulf Grenander of Brown
University in the two-dimensioanl form described in Problem 7. In that form, the
maximum sum subarray was the maximum likelihood estimator of a certain kind of
pattern in a digitized picture. Because teh two-dimentional problem required
too much time to solve, Grenander simplified ti to one dimension to gain insight
into its structure.

Grenander observed that the cubic time of Algorithm 1 was prohibitively slow, and
derived Algorithm 2. In 1977 he described the problem to Michael Shamos of
UNILOGIC, Ltd. (then of Carnegie-Mellon University) who overnight designed
Algorithm 3. When Shamos showed me the problem shortly thereafter, we thought
that it was probably the best possible; researchers had just shown that several
similar problems require time propotional to $n\log n.$ A few days later Shamos
described the problem and its history at a Catnegie-Mellon seminar attended by
statistician Jay Kadane, who designed Algorithm 4 within a minute. Fortunately,
we know that there is no faster algorithm: any correct algorithm must take $O(n)$
time.

Even though the one-dimensional problem is completely solved, Grenander's
original two-dimensional problem remained open eight years after it was posed, as
this book went press. Because of the computational expense of all known
algorithm, Grenander had to abandon that approach to the pattern-matching
problem. Readers who feel that the linear-time algorithm for the one-dimensional
problem is ``obvious'' are therefore urged to find an ``obvious'' algorithm for
Problem 7!

The algorithms in this story were never incorporated into a system, but they
illustrate important algorithm design techniques that have had substantial
impact on many systems (see Section 7.9).
\smallskip
\more
{\it Save state to avoid recomputations.}\enspace This simple form of dynamic
programming arose in Algorithm 2 and 4. By using space to store results, we avoid
using time to recompute them.
\smallskip
\more
{\it Preprocess information into data structures.}\enspace The {\it CumArray\/}
sutructure in Algorithm 2b allowed the sum of a subvector to be computed in just
a couple of operations.
\smallskip
\more
{\it Divide-and-conquer algorithms.}\enspace Algorithm 3 uses a simple form of
divide-and-conquer; textbooks on algorithm design describe more advanced forms.
\smallskip
\more
{\it Scanning algorithms.}\enspace Problems on arrays can often be solved by
asking ``how can I extend a solution for $x[1\ltdot i-1]$ to a solution for $x[1
\ltdot i]$?'' Algorithm 4 stores both the old snaswer and some auxsiliary data to
compute the new answer.
\smallskip
\more
{\it Cumulatives.}\enspace Algorithm 2b uses a cumulative table in which the $i^
{\rm th}$ element contains the sum of the first $i$ values of $x$; such tables
are common when dealing with ranges. In business data processing applications,
for instance, one find s the sales from March to October by substracting the
February year-to-date sales from the Octover year-to-date sales.
\smallskip
\more
{\it Lower bounds.}\enspace Algorithm designers sleep peacefully only when they
know their algorithms are the best possible; for this assurance, they must prove
a matching lower bound. The linear lower bound for this problem is the subject
of Problem 9; more complex lower bounds can be quite difficult.

\section Problems

\prob
Algorithm 3 and 4 use subtle code that is easy to get wrong. Use the program
verification techniques of Column 4 to argue the correctness of the code; specify
the loop invariants carefully.

\prob
Our analysis of the four algorithms was done only at the ``big-oh'' level of
detail. Analyze the number of {\it max} functions used by each algorithm as
exactly as possible; does this exercise give any insight into the running times
of the programs? How much space does each algorithm require?

\prob
We defined the maximum subvector of an array of negative numbers to be zero, the
sum of the empty subvector. Suppose that we had instead defined the maximum
subvector to be the value of the largest element; how would you change the
programs?

\prob
Suppose that we wished to find the subvector with the sum closest to zero rather
than that with maximum sum. What is the most efficient algorithm you can design
for this task? What algorithm design techinques are applicable? What if we
whished to find the subvector with the sum closest to a given real number $t$?

\prob
A turnpike consists of $n-1$ stretches of road between $n$ toll stations; each
stretch has an associated cost of travel. It is trivial to tell the cost of going
between any to station in $O(n)$ time using only an array of the costs or in
constant time using a table with $O(n^2)$ entries. Describe a data structure that
requires $O(n)$ space but allows the cost of any route to be computed in constant
time.

\prob
After the array $x[1\ltdot n]$ is initialized to zero, $n$ of the following
operations are performed
\begindisplay
\vbox{
\+\bf for $i\leftarrow l$ do $u$ do\cr
\+\quad $x[i]\leftarrow x[i]+v$\cr}
\enddisplay
where $l$, $u$ and $v$ are parameters of each operation ($l$ and $u$ are integers
satisfying $1\le l\le u\le n$ and $v$ is real). After the $n$ operations, the
values of $x[1]$ through $x[n]$ are reported in order .The method just sketched
requires $O(n^2)$ time. Can you find a faster algorithm?

\prob
In the maximum subarray problem we are given an $n\times n$ array of reals, and
we must find the maximum sum contained in any rectangular subarray. What is the
complexity of this problem?

\prob
Modify Algorithm 3 (the divide-and-conquer algorithm) to run in linear worst-case
tiem.

\prob
Prove that any correct algorithm for computing maximum subvectors must inspect
all $n$ inputs. (Algorithms for some problems may correctly ignore some inputs;
consider Saxe's algorithm in Solution 2.2 and Boyer and Moore's substring
searching algorithm in the October 1977 {\sl CACM\/}.)

\prob
Given integers $m$ and $n$ and the real vector $x[1\ltdot n],$ find the integer
$i(1\le l\le n-m)$ such that the sum $x[i]+\ldots+x[i+m]$ is nearest zero.

\prob
What is the solution of the recurrence $T(n)=2T(n/2)+CN$ when $T(1)=0$ and $n$ is
a power of two? Prove your result by mathematical induction. What if $T(1)=C$?

\section Further Reading

Only extensive study can put algorithm design techniques at your fingertips; most
programmers will get this only from a textbook on algorithms. {\sl Data
Structures and Algorithms\/} by Aho, Hopcroft and Ullman (published by Addison-%
Wesley in 1983) is an excellent undergraduate text. Chapter 10 on ``Algorithm
Design Techniques'' is especially relevant to this column.

\bye
