
\input xetexko.sty

\hangulfont="NanumMyeongjo:mapping=tex-text" at 10pt

% 기본 폰트를 한글을 포함하도록 다시 정의합니다.
\def\hfontsize#1{\ifx\empty#1\empty\else
 \edef\temp{\noexpand\hfontsizex\fontname\hangfnt\space\noexpand\nil}%
 \temp{#1}\fi}
\def\hfontsizex"#1" #2\nil#3{\font\hangfnt"#1" at #3}
\def\hfontname#1{\ifx\empty#1\empty\else
 \edef\temp{\noexpand\hfontnamex\fontname\hangfnt\space\noexpand\nil}%
 \temp{#1}\fi}
\def\hfontnamex"#1" #2\nil#3{\font\hangfnt"#3:mapping=tex-text" #2}
\def\hfont#1#2{\hfontname{#1}\hfontsize{#2}}

% 기본 글꼴 설정
\hfont{NanumMyeongjo}{10pt}
\expandafter\def\expandafter\tt\expandafter{\tt\hfontname{NanumGothic}}
\expandafter\def\expandafter\sl\expandafter{\sl\hfontname{NanumGothic}}
\expandafter\def\expandafter\it\expandafter{\it\hfontname{NanumGothic}}
\expandafter\def\expandafter\bf\expandafter{\bf\hfontname{NanumMyeongjo ExtraBold}}

% 각종 폰트를 한글을 포함하도록 재정의
\let\orgninerm=\ninerm
\def\ninerm{\hfont{NanumMyeongjo}{9pt}\orgninerm}
\let\orgeightrm=\eightrm
\def\eightrm{\hfont{NanumMyeongjo}{8pt}\orgeightrm}
\let\orgsc=\sc
\def\sc{\hfont{NanumMyeongjo}{8pt}\orgsc}
\let\orgtentex=\tentex
\def\tentex{\hfont{NanumGothic}{10pt}\orgtentex}

% FIFO
\def\fifo#1{\ifx\ofif#1\ofif\fi\process#1\fifo} 
\def\ofif#1\fifo{\fi}

% LIFO
\def\lifo#1#2\ofil{\ifx\empty#2%
  \empty\ofil\fi\lifo#2\ofil\process#1}
\def\ofil#1\ofil{\fi}

\parskip 3pt
\baselineskip 16pt
\abovedisplayskip=15pt plus 3pt minus 9pt
\belowdisplayskip=15pt plus 3pt minus 9pt

\endinput
