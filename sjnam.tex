\input kotex.sty

% 기본 글꼴 설정
\expandafter\def\expandafter\tt\expandafter{\tt\hfontname{outtzm}}
\expandafter\def\expandafter\sl\expandafter{\sl\hfontname{outbmm}}
\expandafter\def\expandafter\it\expandafter{\it\hfontname{outpgm}}
\expandafter\def\expandafter\bf\expandafter{\bf\hfontname{nanummjb}}

% 각종 폰트를 한글을 포함하도록 재정의
\font\titlefont=cmbx12 % scaled\magstep4 % title on the contents page
\let\otitlefont=\titlefont \def\titlefont{\hfont{nnmbrgtb}{at 13pt}\otitlefont}
\let\orgninerm=\ninerm \def\ninerm{\hfont{nanummjm}{at 9pt}\orgninerm}
\let\orgeightrm=\eightrm \def\eightrm{\hfont{nanummjm}{at 8pt}\orgeightrm}
\let\orgsc=\sc \def\sc{\hfont{nanumgtm}{at 9pt}\orgsc}

% FIFO
\def\fifo#1{\ifx\ofif#1\ofif\fi\process#1\fifo} 
\def\ofif#1\fifo{\fi}

% LIFO
\def\lifo#1#2\ofil{\ifx\empty#2%
  \empty\ofil\fi\lifo#2\ofil\process#1}
\def\ofil#1\ofil{\fi}

% 이미지 넣기 by DohyunKim
% http://www.ktug.org/xe/index.php?document_srl=202860&mid=KTUG_open_board
\def\picture#1#{%
  \def\next{%
    \unless\ifcsname PicKey #1\the\toks0\endcsname
      \immediate\pdfximage #1 {\the\toks0}%
      \expandafter\xdef\csname PicKey #1\the\toks0\endcsname{\the\pdflastximage}%
    \fi
    \expandafter\pdfrefximage\csname PicKey #1\the\toks0\endcsname\relax}%
  \afterassignment\next\toks0= }

\parskip 3pt
\baselineskip 15pt
\abovedisplayskip=15pt plus 3pt minus 9pt
\belowdisplayskip=15pt plus 3pt minus 9pt
