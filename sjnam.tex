\input kotex.sty

%% 기본 폰트를 한글을 포함하도록 다시 정의합니다.
%\def\hfontsize#1{\ifx\empty#1\empty\else
% \edef\temp{\noexpand\hfontsizex\fontname\hangfnt\space\noexpand\nil}%
% \temp{#1}\fi}
%\def\hfontsizex"#1" #2\nil#3{\font\hangfnt"#1" at #3}
%\def\hfontname#1{\ifx\empty#1\empty\else
% \edef\temp{\noexpand\hfontnamex\fontname\hangfnt\space\noexpand\nil}%
% \temp{#1}\fi}
%\def\hfontnamex"#1" #2\nil#3{\font\hangfnt"#3:mapping=tex-text" #2}
%\def\hfont#1#2{\hfontname{#1}\hfontsize{#2}}

% 기본 글꼴 설정
\expandafter\def\expandafter\tt\expandafter{\tt\hfontname{nanumgtm}}
\expandafter\def\expandafter\sl\expandafter{\sl\hfontname{nanummjmo}}
\expandafter\def\expandafter\it\expandafter{\it\hfontname{nanummjmo}}
\expandafter\def\expandafter\bf\expandafter{\bf\hfontname{nanummjb}}

% 각종 폰트를 한글을 포함하도록 재정의
\font\titlefont=cmr7 scaled\magstep4 % title on the contents page
\font\ttitlefont=cmtt10 scaled\magstep2 % typewriter type in title
\font\tentex=cmtex10 % TeX extended character set (used in strings)
\let\otitlefont=\titlefont \def\titlefont{\hfont{nanummjb}{at 13pt}\otitlefont}
\let\ottitlefont=\ttitlefont \def\ttitlefont{\hfont{nanumgtm}{at 13pt}\ottitlefont}
\let\orgninerm=\ninerm \def\ninerm{\hfont{nanummjm}{at 9pt}\orgninerm}
\let\orgeightrm=\eightrm \def\eightrm{\hfont{nanummjm}{at 8pt}\orgeightrm}
\let\orgsc=\sc \def\sc{\hfont{nanumgtm}{at 9pt}\orgsc}
\let\orgtentex=\tentex \def\tentex{\hfontname{nanumgtm}\orgtentex}
\let\orguppercase=\uppercase \def\uppercase{\sc\orguppercase}

% FIFO
\def\fifo#1{\ifx\ofif#1\ofif\fi\process#1\fifo} 
\def\ofif#1\fifo{\fi}

% LIFO
\def\lifo#1#2\ofil{\ifx\empty#2%
  \empty\ofil\fi\lifo#2\ofil\process#1}
\def\ofil#1\ofil{\fi}

\parskip 3pt
\baselineskip 15pt
\abovedisplayskip=15pt plus 3pt minus 9pt
\belowdisplayskip=15pt plus 3pt minus 9pt
