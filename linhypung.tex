\documentclass[12pt,b5paper]{article}

\usepackage[text={110mm,5.5in}]{geometry}

\usepackage{linhypung}

\pagestyle{empty}

\begin{document}

\begin{exercise}
  \question{백선}{살기 위해서 백은 어떻게 손질하면 좋을까? 착점은 한 곳이다.}
  \showquestiongoban{
    \white{r18,r17,s16,s15} 
    \black{q18,q17,o17,r16,r15,r14,s14,r11}} 
\end{exercise}

\begin{answer}
\begin{solutiontext}
\answertitle{삶}
\step 백 1의 꼬부림의 정해. 여러 가지 착점이 있을 것 같으나,
이 한 수이다. 흑 2로 치중하더라도 백 3으로 귀에서 눌러 확실히 살아
있다.
\step 계속하여 흑 1의 젖힘에는 백 2로 이으면 그만.
흑은 치중한 한 점을 이을 수 없다.
\end{solutiontext}
\begin{solutionfigure}
\white[1]{s18,t17,t18}
\stepgoban
\clear{s18,t17,t18}
\white{s18,t18}
\black{t17}
\black[1]{t15,s17}
\stepgoban
\end{solutionfigure}
\end{answer}


\begin{exercise}
  \question{흑선}{위의 `제1문'에서 백이 뛰어 가일수한 형태.\par 
  이렇게 되면 손질한 셈이 아니다.}
  \showquestiongoban{
    \white{r18,t18,r17,s16,s15} 
    \black{q18,q17,o17,r16,r15,r14,s14,r11}} 
\end{exercise}

\begin{answer}
\begin{solutiontext}
\answertitle{백 죽음}
\step 흑 1로 젖히는 착수가 좋으며, 이른바 젖혀서 죽이는
유형이다. 백 2로 이으면 흑 3으로 둘 뿐. 실은 흑 1로 먼저 3에 두어도
무방하지만, 대부분의 경우, 우선 젖힘부터 두는 것이 옳다고 기억해
두기 바란다.
\step 백 2로 누르면 흑 3으로 끊고, 5로 단수쳐 파호한다.
\end{solutiontext}
\begin{solutionfigure}
\black[1]{t15,s17,s19}
\stepgoban
\clear{t15,s17,s19}
\black[1]{t15,t16,s17,t17,r19}
\stepgoban
\end{solutionfigure}
\end{answer}


\begin{exercise}
  \question{흑선}{이것은 `제1문'의 백이 호구쳐서 이은 형태.\par 
  역시 손질이 정확하지 않으며, 수단의 여지가 있다.}
  \showquestiongoban{
    \white{r18,t17,r17,s16,s15} 
    \black{q18,q17,o17,r16,r15,r14,s14,r11}} 
\end{exercise}

\begin{answer}
\begin{solutiontext}
\answertitle{패}
\step 흑 1의 치중이 급소. 이 맥은 꼭 기억해 두기 바란다. 
백 2로 가로막으면 흑 3으로 나와, 무조건 백 죽음이다.
\step 따라서 백은 2로 받아 흑 3의 넘어감에 4로 눈을 갖게
되는데, 흑 5로 먹여쳐서 패로 한다. 흑 1로 2에 붙이는 것은
최악의 수. 백 1로 젖혀 살리고 만다.
\end{solutiontext}
\begin{solutionfigure}
\black[1]{s19,r19,s18}
\stepgoban
\clear{s19,r19,s18}
\black[1]{s19,s18,r19,t15,t18}
\stepgoban
\end{solutionfigure}
\end{answer}

\end{document}