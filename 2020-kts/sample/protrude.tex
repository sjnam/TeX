\directlua{
  local function check_lines (head)
  for line in node.traverse_id(node.id("hlist"), head) do
    local m = node.tail(line.list)
    while m.id \string~= 29 do
       m = m.prev
    end
    local cm
    if m.char == string.byte(",") then
       cm = node.copy(m)
       node.remove(line, m)
    end
    %if line.glue_order == 0 and line.glue_sign == 1 and line.glue_set > .2 then
      line.list = node.hpack(line.list, line.width, "exactly")
    %end
    if cm then
       texio.write_nl(string.char(m.char))
       node.insert_after(line, node.tail(line.list), cm)
    end
  end
  return head
end
callback.register("post_linebreak_filter", check_lines)}

\hsize 4in
This is a handbook about \TeX, a new typesetting system intended
for the creation
of beautiful books---and especially for books that contain a lot of
mathematics. By preparing a manuscript in \TeX\ format, you will be
telling a computer exactly how the manuscript is to be transformed into
pages whose typographic quality is comparable to that of the world's
finest printers; yet you won't need to do much more work than would be
involved if you were simply typing the manuscript on an ordinary
typewriter. In fact, your total work will probably be significantly less,
if you consider the time it ordinarily takes to revise a typewritten manuscript,
since computer text files are so easy to change and to reprocess.
This manual is intended for people who have never used \TeX\ before, as
well as for experienced \TeX\ hackers. In other words, it's supposed to
be a panacea that satisfies everybody, at the risk of satisfying nobody.
Everything you need to know about \TeX\ is explained
here somewhere, and so are a lot of things that most users don't care about.
If you are preparing a simple manuscript, you won't need to
learn much about \TeX\ at all; on the other hand, some
things that go into the printing of technical books are inherently
difficult, and if you wish to achieve more complex effects you
will want to penetrate some of \TeX's darker corners.

\bye
