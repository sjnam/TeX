\directlua{%
  local nsubtype = node.subtype
  local good_item = function (item)
     local isubtype = item.subtype
     if item.id == node.id("glue") and (isubtype == nsubtype("leftskip")
                             or isubtype == nsubtype("rightskip")
                             or isubtype == nsubtype("parfillskip")) then
        return false
     else
        return true
     end
  end

  local underline = function (head, order, ratio, sign)
     local item = head
     while item do
        if node.has_attribute(item, 100) and good_item(item) then
           local end_node = item
           while end_node.next and good_item(end_node.next) and
              node.has_attribute(end_node.next, 100) do
              end_node = end_node.next
           end
           local item_line = node.new("rule")
           item_line.depth = tex.sp("1.4pt")
           item_line.height = tex.sp("-1pt")
           item_line.width = node.dimensions(ratio, sign, order,
                                             item, end_node.next)
           local item_kern = node.new("kern", nsubtype("userkern"))
           item_kern.kern = -item_line.width % same as \llap in plain TeX
           node.insert_after(head, end_node, item_kern)
           node.insert_after(head, item_kern, item_line)
           item = end_node.next
        else
           item = item.next
        end
     end
  end

  get_lines = function (head)
     for line in node.traverse_id(node.id("hlist"), head) do
        underline(line.list, line.glue_order, line.glue_set, line.glue_sign)
     end
     callback.register("post_linebreak_filter", nil)
     return head
  end
}

\def\underline#1{%
  \leavevmode \attribute100 = 1 #1% set
  \attribute100 = -"7FFFFFFF% unset
  \directlua{callback.register("post_linebreak_filter", get_lines)}
}

\hsize 3in

\underline{This is a handbook about \TeX},
a new typesetting system intended for the creation
of \underline{beautiful} books---and especially for books that contain a lot of
mathematics. \underline{By preparing a manuscript in \TeX\ format, you will be
telling a computer exactly how the manuscript is to be transformed into
pages whose typographic quality is comparable to that of the world's
finest printers; yet you won't need to do much more work than would be
involved if you were simply typing the manuscript on an ordinary
typewriter.} In fact, your total work will probably be significantly less,
if you consider the time it ordinarily takes to revise a typewritten manuscript,
since computer text files are so easy to change and to reprocess.
This manual is intended for people who have never used \TeX\ before, as
well as for experienced \TeX\ hackers. In other words, it's supposed to
be a panacea that satisfies everybody, at the risk of satisfying nobody.
Everything you need to know about \TeX\ is explained
here somewhere, and so are a lot of things that most users don't care about.
If you are preparing a simple manuscript, you won't need to
learn much about \TeX\ at all; on the other hand, some
things that go into the printing of technical books are inherently
difficult, and if you wish to achieve more complex effects you
will want to penetrate some of \TeX's darker corners.

\bye
