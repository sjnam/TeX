\input luatexbase.sty

\directlua{%
  local color_push = node.new("whatsit", "pdf_colorstack")
  color_push.stack = 0
  color_push.command = 1
  local color_pop  = node.new("whatsit", "pdf_colorstack")
  color_pop.stack  = 0
  color_pop.command = 2

  textcolor = function (head)
     for line in node.traverse_id(node.id("hlist"),head) do
        local glue_ratio = 0
        if line.glue_order == 0 then
           if line.glue_sign == 1 then
              glue_ratio = .5 * math.min(line.glue_set, 1)
           else
              glue_ratio = -.5 * line.glue_set
           end
        else
           if line.glue_sign == 1 then
              glue_ratio = line.glue_set / (line.width/65536)
           end
        end
        color_push.data = .5 + math.min(0.5, glue_ratio) .. " g"

        local rule = node.new("rule")
        rule.width = line.width
        local p = line.list
        line.list = node.copy(color_push)
        node.flush_list(p)
        node.insert_after(line.list, line.list, rule)
        node.insert_after(line.list, node.tail(line.list), node.copy(color_pop))
     end
     luatexbase.remove_from_callback("post_linebreak_filter", "textcolor")
     return head
  end
}

\def\parabox#1{%
  \vbox{\hsize #1
    \noindent This is a handbook about
    \TeX, a new typesetting system intended for the creation
    of beautiful books---and especially for books that contain a lot of
    mathematics. By preparing a manuscript in \TeX\ format, you will be
    telling a computer exactly how the manuscript is to be transformed into
    pages whose typographic quality is comparable to that of the world's
    finest printers; yet you won't need to do much more work than would be
    involved if you were simply typing the manuscript on an ordinary
    typewriter. In fact, your total work will probably be significantly less,
    if you consider the time it ordinarily takes to revise a typewritten manuscript,
    since computer text files are so easy to change and to reprocess.}}

\def\textcolors#1{%
  \noindent {\tt\string\hsize=#1}\medskip
  \centerline{%
    \parabox{#1} \quad
    \directlua{luatexbase.add_to_callback("post_linebreak_filter",
                                           textcolor, "textcolor")}
    \parabox{#1}}}

\nopagenumbers

\textcolors{7cm}
\bigskip\bigskip
\textcolors{6cm}

\bye
