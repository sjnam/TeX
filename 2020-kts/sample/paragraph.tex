\nopagenumbers

\hsize 5in

\let\lua\directlua % Don't change this.
\catcode`\~=12
\lua{%
local HLIST = node.id("hlist")
local GLUE = node.id("glue")
local GLYPH = node.id("glyph")
local WHAT = node.id("whatsit")

function lineparse(line)

      local box_order=line.glue_order
      local box_ratio=line.glue_set
      local box_sign=line.glue_sign %0 = normal, 1 = stretching, 2 = shrinking
    % This loop uses a function to look for all lines of the typeset paragraph
    % This function scans all the items in a typeset line of the paragraph
	for n in node.traverse(line.head) do
       % We have seen some glue: calculate its size, draw a box
	   if n.id == GLUE then
            % glue parameters
			local gwidth=n.width %horizontal or vertical displacement
			local gstr=n.stretch %extra (+ve) displacement or stretch amount
			local gstrord=n.stretch_order %factor applied to stretch amount
			local gshr=n.shrink %extra (-ve) displacement or shrink amount
			local gshrord=n.shrink_order %factor applied to shrink amount
            local width=0
    		% This section is how TeX calculates glue settings
            if box_sign==0 then %(normal)
              	width = gwidth
             elseif box_sign==1 then %(stretch)
             	if gstrord == box_order then
               		width = gwidth + (box_ratio * gstr)
             	else 
             		width = gwidth
             	end
             elseif box_sign == 2 then %(shrink)
              	if gshrord == box_order then
            	  	width = gwidth - (box_ratio * gshr)
            	else
            	   	width = gwidth
            	end
             end %if box_sign
            % now calculate the glue's width in bp + draw a box.
            if width > 0 then % don't bother with zero width glues
            	local pdfn=node.new(WHAT, "pdf_literal")
           	 	local wp = (width/65536)*(72/72.27)
	  			pdfn.data=" 0.2 w 0 0 "..wp.." 5 re q 0 0 0.55 rg f  Q "
	            pdfn.mode=0
    			% insert into the typeset paragraph
				local np = n.prev
            	np.next=pdfn
            	pdfn.prev=np
            	pdfn.next=n
            	n.prev=pdfn
            end 
    	end % if n.id == GLUE
        
        % Within the typeset line we've seen another hbox
         if n.id == HLIST then
        	local nw = n.width
            local nd= n.depth
            local nh = n.height
            local ns = n.shift
            
         	local hw = ((nw/65536)*(72/72.27))
          	local hd = ((nd/65536)*(72/72.27))
          	local hh = ((nh/65536)*(72/72.27))
            local hs=  ((ns/65536)*(72/72.27))
        
 	        local pdfn=node.new(WHAT, "pdf_literal")
            local tot=hd+hh %total height + depth
            local ts = hd + hs %total height + any shift

            %pdfn.data=" 0.2 w 0 -"..ts.." "..hw.. " "..tot.." re q 0.9 0.0 0.0 rg f  Q"            
            pdfn.data=" 0.2 w 0 -"..ts.." "..hw.. " "..tot.." re q 0.8 g f  Q"
       		pdfn.mode=0
     		% insert into the typeset paragraph
 			local np = n.prev
             np.next=pdfn
             pdfn.prev=np
             pdfn.next=n
             n.prev=pdfn
                
         end % if n.id == HLIST then
         
         if n.id == GLYPH then
         
            local nw = n.width
            local nd=  n.depth
            local nh = n.height
            
         	local hw = ((nw/65536)*(72/72.27))
          	local hd = ((nd/65536)*(72/72.27))
          	local hh = ((nh/65536)*(72/72.27))
            local tot=hd+hh %total height + depth           
            local pdfn=node.new(WHAT, "pdf_literal")
            pdfn.data=" 0.2 w 0 -"..hd.." "..hw.. " "..tot.." re q 0.0 0.0 0.0 RG S  Q"
            pdfn.mode=0
      		% insert into the typeset paragraph
  			local np = n.prev
            if np ~= nil then
            	np.next=pdfn
            	pdfn.prev=np
           		pdfn.next=n
            	n.prev=pdfn
            elseif np == nil then
            %First glyph at start of the list: special steps needed... May not always work...?
                
                n.prev=pdfno
                pdfn.next=n 					
                line.head=pdfn  
                end
         end % if n.id == GLYPH then
         
         % Insert the line bounding box
    	 local lw = line.width
            local ld = line.depth
            local lh = line.height
            local lhw = ((lw/65536)*(72/72.27))
          	local lhd = ((ld/65536)*(72/72.27))
          	local lhh = ((lh/65536)*(72/72.27))
            local ltot=lhd+lhh
            local pdfn=node.new(WHAT, "pdf_literal")
            pdfn.data=" 0.1 w 0 -"..lhd.." "..lhw.. " "..ltot.." re q 0.9 g f  Q"
      		pdfn.mode=0
            %local ln=line.head.next
            pdfn.next=line.head
            line.head=pdfn
         
	end % for n in node.traverse
end % end of function

parsepara = function (head)
for line in node.traverse_id(HLIST, head) do
    lineparse(line)
end
return head
end
}

\lua{callback.register("post_linebreak_filter", parsepara)}

\nopagenumbers

\hsize 4in
For example, when \TeX{} typesets a paragraph of text and breaks it into a series of lines, it considers the paragraph's text as a sequence of boxes and uses the dimensions of those character boxes to find the best linebreaks. Each line of the paragraph is itself a box (containing other boxes---e.g., characters) and the typeset paragraph lines (boxes) are stacked (vertically) to produce the paragraph. Eventually, the largest box of all is produced: the typeset page. Clearly, this is a very simplified picture because you also need the ability to arbitrarily position those boxes and \TeX{} does this using so-called glue. Knuth commented (page 70 of The \TeX book) that ``glue'' probably should have been referred to as ``spring'' but the term glue was adopted early on and, to use Knuth's pun, it stuck.
\bye
