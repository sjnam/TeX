\documentclass[11pt]{beamer}

\usetheme{metropolis}

\usepackage{xspace}
\def\brotli{\textsc{Brotli}}
\newcommand{\gamename}{\textbf{\brotli}\xspace}

\title{lua-resty-brotli}
\subtitle{Lua bindings to \brotli\ for LuaJIT using FFI}
\date{\today}
\author{donald.knuth}

\begin{document}

\maketitle

\begin{frame}{Table of contents}
  \setbeamertemplate{section in toc}[sections numbered]
  \tableofcontents[hideallsubsections]
\end{frame}

\section{\brotli}

\begin{frame}[fragile]{\brotli}
  \gamename is a \alert{generic-purpose lossless compression} algorithm that compresses data
  using a combination of a modern variant of the LZ77 algorithm, Huffman coding and
  2nd order context modeling, with the best compression ratio.

  \gamename was first released in 2015 for off-line compression of
  \alert{web fonts.} The version of \brotli\ released in September 2015
  by the Google software engineers contained enhancements in generic
  lossless data compression, with particular emphasis on use for \alert{HTTP compression.}
\end{frame}

\begin{frame}{Comparison of Compression Algorithms}
  \vskip 3mm
  \hbox to\textwidth{\hss\includegraphics[height=7.5cm]{comp.jpg}\hss}
  \vskip-4mm
  \href{https://cran.r-project.org/web/packages/brotli/vignettes/brotli-2015-09-22.pdf}
    {\alert{\scriptsize Comparison of Brotli, Deflate, Zopfli, LZMA, LZHAM
and Bzip2 Compression Algorithms}}

\end{frame}

\section{\brotli\ in the Web}

\begin{frame}{Browser}
  \begin{description}
  \item [Chrome] has supported \brotli\ since version 49.
  \item [Edge] supports \brotli\ since version 15.
  \item [Firefox] implemented \brotli\ \alert{in} version 44.
  \item [Opera] supports \brotli\ since version 36.
  \item [Safari] no public commitment as of October 2016.
  \end{description}
\end{frame}

\begin{frame}{Web server}
  \begin{description}
  \item [Apache HTTP Server] \href{https://httpd.apache.org/docs/trunk/mod/mod_brotli.html}
    {in final stages of development}.
  \item [Microsoft IIS] no official support nor commitment to implement.
    There exists a community module that adds support.
  \item [Nginx] no official support nor commitment to implement.
    A ngx\_brotli module is provided by Google Inc.
  \item [Node.js] no official support but there exists a community module
  \end{description}
\end{frame}


\section{Nginx modules}

\begin{frame}{Nginx modules for \texttt{gzip}}
  \begin{itemize}
  \item \texttt{\textbf{ngx\_http\_\alert{gunzip}\_module}}
  \item \texttt{\textbf{ngx\_http\_\alert{gzip}\_module}}
  \item \texttt{\textbf{ngx\_http\_\alert{gzip\_static}\_module}}
  \end{itemize}
\end{frame}

\begin{frame}{Nginx modules for \gamename}
  \href{https://github.com/google/ngx_brotli}{\texttt{ngx\_brotli}}
  \begin{itemize}
  \item \texttt{\textbf{ngx\_http\_\alert{brotli}\_module}}
  \item \texttt{\textbf{ngx\_http\_\alert{brotli\_static}\_module}}
  \end{itemize}
\end{frame}

\section{Lua module}

\begin{frame}{Why lua module?}
  \texttt{ngx\_brotli} is good but \ldots
  \begin{itemize}
  \item Uses deprecated \gamename API.
  \item No module similiar to \texttt{\textbf{ngx\_http\_gunzip\_module}}
  \item No streaming compress or decompress
  \end{itemize}
  
  \bigskip

  \alert{\texttt{lua-resty-brotli}}, Lua bindings to google \brotli\ for Luajit using FFI.
\end{frame}

\section {Demo}

\begin{frame}{Scenario}
  \begin{itemize}
  \item The directory, \texttt{\textbf{html}} has only precompressed
    files with the
    \alert{\texttt{\textbf{.br}}} filename extension instead of reqular files.
  \item If a brower does not support \alert{\texttt{\textbf{br}}},
    it decompress on-the-fly.
  \item Compressing on-the-fly for dynamic contents.
  \item Streaming decompress the request compressed body
  \end{itemize}
\end{frame}

\begin{frame}[standout]
  Question?
\end{frame}

\end{document}
